%% PRE EDITION
\documentclass[a4paper]{article}
\usepackage[utf8]{inputenc}
\usepackage[T1]{fontenc}
\usepackage[french]{babel}
\usepackage{soul}
\usepackage[pdftex]{graphicx}

\usepackage{amsfonts}
\usepackage{amsthm}
\usepackage{amsmath}
\usepackage{amssymb}
\usepackage{mathrsfs}
\usepackage{booktabs}
\usepackage{siunitx}
\usepackage{thmtools}
\usepackage{ulem}

%% LAYOUT TITLE
\usepackage[explicit]{titlesec}
\usepackage{xcolor}
\usepackage{times}
\usepackage{tikz}
\usepackage{lipsum}
\titleformat{\subsection}
  {\color{blue!70}\large\sffamily\bfseries}
  {}
  {0em}
  {\colorbox{blue!10}{\parbox{\dimexpr\linewidth-2\fboxsep\relax}{\arabic{section}.\arabic{subsection}. #1}}}
  []

%% LAYOUT MATHS
\declaretheoremstyle[
    bodyfont=\normalfont\color{red},
    headfont=\color{red}
]{styleattention}

\declaretheoremstyle[
    spacebelow=1em
]{styleremarque}

\declaretheoremstyle[
    spaceabove=-6pt, 
    spacebelow=6pt, 
    headfont=\normalfont\bfseries, 
    bodyfont = \normalfont,
    postheadspace=1em, 
    qed=$\Box$, 
    headpunct={$\rhd$}
]{mystyle} 

\declaretheorem[thmbox=M,numberwithin=section,title=Définition]{definition}
\declaretheorem[thmbox=M,sibling=definition]{proposition}
\declaretheorem[thmbox=M,sibling=definition]{corollaire}
\declaretheorem[thmbox=M,sibling=definition,title=Théorème]{theoreme}
\declaretheorem[thmbox=M,sibling=definition]{lemme}
\declaretheorem[thmbox=M,sibling=definition,title=Propriété]{propriete}
\declaretheorem[thmbox=M,sibling=definition,title=Propriétés]{proprietes}
\declaretheorem[style=styleremarque,sibling=definition,title=Remarque]{remarque}
\declaretheorem[style=styleattention,title=À revoir]{Arevoir}
\declaretheorem[name={}, style=mystyle, unnumbered]{preuve}


\renewcommand\qedsymbol{$\blacksquare$}

%% NEW COMMAND
% solution
\newcommand{\y}{y}
% concentration monomers
\newcommand{\cmono}{c}
\newcommand{\pol}{a}
\newcommand{\dep}{b}
\newcommand{\mass}{\mathrm{M}}

\usepackage{geometry}
\geometry{hmargin=3cm,vmargin=2.5cm}

\usepackage{tabularx}
\usepackage{float}

\title{$v \in C^0$, $\epsilon \ne 0$}
\author{Cécile Della Valle}

%% DEBUT DE REDACTION
\begin{document}

\maketitle


%%%%%%%%%%%%%%%%%%%%%%%%%%%%%%%%%%%%%%%%%%%%%%%%%%%%%%%%%%%%%%%%%%%%%%%%%%%%%%%%%%%%%%%%%%%%%%%%%%%%%%%%%%%%%%%%%%%%%%%%%%%%%%%%%
%%%%%%%%%%%%%%%%%%%%%%%%%%%%%%%%%%%%%%%%%%%%%%%%%%%%%%%%%%%%%%%%%%%%%%%%%%%%%%%%%%%%%%%%%%%%%%%%%%%%%%%%%%%%%%%%%%%%%%%%%%%%%%%%%
%%%%%%%%%%%%%%%%%%%%%%%%%%%%%%%%%%%%%%%%%%%%%%%%%%%%%%%%%%%%%%%%%%%%%%%%%%%%%%%%%%%%%%%%%%%%%%%%%%%%%%%%%%%%%%%%%%%%%%%%%%%%%%%%%
%%%%%%%%%%%%%%%%%%%%%%%%%%%%%%%%%%%%%%%%%%%%%%%%%%%%%%%%%%%%%%%%%%%%%%%%%%%%%%%%%%%%%%%%%%%%%%%%%%%%%%%%%%%%%%%%%%%%%%%%%%%%%%%%%
%%%%%%%%%%%%%%%%%%%%%%%%%%%%%%%%%%%%%%%%%%%%%%%%%%%%%%%%%%%%%%%%%%%%%%%%%%%%%%%%%%%%%%%%%%%%%%%%%%%%%%%%%%%%%%%%%%%%%%%%%%%%%%%%%
\section{Existence de solution pour le problème direct}
%%%%%%%%%%%%%%%%%%%%%%%%%%%%%%%%%%%%%%%%%%%%%%%%%%%%%%%%%%%%%%%%%%%%%%%%%%%%%%%%%%%%%%%%%%%%%%%%%%%%%%%%%%%%%%%%%%%%%%%%%%%%%%%%%
%%%%%%%%%%%%%%%%%%%%%%%%%%%%%%%%%%%%%%%%%%%%%%%%%%%%%%%%%%%%%%%%%%%%%%%%%%%%%%%%%%%%%%%%%%%%%%%%%%%%%%%%%%%%%%%%%%%%%%%%%%%%%%%%%


Supposons $v$ une fonction connue, négative strictement 
(ce qui correspond pqr exemple au cas dans Lifschitz-Slyozov où $0<\dep -\mass $ ), 
et supposons $\epsilon>0$. 

On suppose que $v$ est bornée, de classe $C^1$ et sa dérivée est également bornée.
On s'intéresse à l'équation :


\begin{equation}
\label{eq:cas3}
\begin{cases}
 \displaystyle \frac{\partial y}{\partial t}
 + v(t) \frac{\partial y} {\partial x}  
 - \epsilon \frac{\partial^2 y} {\partial x^2}
 = 0  & \forall (x,t) \in [0,L] \times [0, \tau]\\
 y(x,0) = y_{0} (x) & \forall x \in [0,L] \\
 \displaystyle \frac{\partial y}{\partial t}|_{x=0}
 + v(t) \frac{\partial y} {\partial x}|_{x=0} = 0 & \forall t \in [0,\tau]\\
 y(L,t)=0 & \forall t \in [0,\tau]
\end{cases}
\end{equation}

Puisque $v<0$, pour faciliter la lecture, on pose $\dep (t) = - v(t) >0$.
De plus on pose $y=(u,m)$ avec $u$ solution de~\eqref{eq:cas3} et $u(0)=m$.

On définie $\tilde{y}=(w,\mu)$ tel que 
$(w,\mu)=\displaystyle (\frac{u}{\sqrt{b(t)} }, \frac{u(0)}{\sqrt{b(t)} }) = \frac{1}{\sqrt{b(t)}} y$.

Calculons la dynamique vérifiée par $w$ :

\[\begin{split}
\partial_t \displaystyle \frac{u}{\sqrt{b(t)}} & = \frac{1}{\sqrt{b(t)}} \partial_t u
                                                    -  \frac{\dot{b}(t)}{2 b(t)} \frac{u}{\sqrt{b(t)}}\\
												& = \frac{1}{\sqrt{b(t)}}
												   (b(t) \partial_x u 
												  +\epsilon \partial_{xx} u)
												  - \frac{\dot{b} (t) }{2b(t)} \frac{u}{\sqrt{b} }\\
												& = b(t) \partial_x  w
												  +\epsilon \partial_{xx} w 
												  - \frac{\dot{b} (t) }{2b(t) } w 
\end{split}\]

Calculons la dynamique vérifiée par $\mu$ :

\[
\begin{split}
	\displaystyle \partial_t \frac{m}{\sqrt{b(t)}} &= \frac{1}{\sqrt{b(t)}} \partial_t m
	                                                 - \frac{\dot{b}(t)}{2b(t)}\frac{m}{\sqrt{b(t)}}\\
												  &= \frac{1}{\sqrt{b(t)}} b(t) \partial_x u(0)
												  - \frac{\dot{b}(t)}{2b(t)} \frac{m}{\sqrt{b(t)}}\\
												  &= b(t) \partial_x w(0)
												  - \frac{\dot{b}(t)}{2b(t)} \mu
\end{split}
\]

On cherche donc la solution $\tilde{y}$ du système :

\begin{equation}
\label{eq:cas3b}
\begin{cases}
 \displaystyle \frac{\partial \tilde{y} }{\partial t}
 - b(t) \frac{\partial \tilde{y} }{\partial x}  
 - \epsilon \frac{\partial^2 \tilde{y}} {\partial x^2}
 =  - \frac{\dot{b} (t) }{2b(t) } \tilde{y}  & \forall (x,t) \in [0,L] \times [0, \tau]\\
 \displaystyle \tilde{y}(x,0) = \frac{y_{0}}{\sqrt{b(0)}} (x) & \forall x \in [0,L] \\
 \displaystyle \frac{\partial \tilde{y}}{\partial t}|_{x=0}
 - b(t) \frac{\partial \tilde{y}} {\partial x}|_{x=0} = - \frac{\dot{b}(t)}{2b(t)} \tilde{y}(0) & \forall t \in [0,\tau]\\
 \tilde{y}(L,t)=0 & \forall t \in [0,\tau]
\end{cases}
\end{equation}

On montrera par la suite que si $\tilde{y} \in \mathscr{W}$ 
est solution de~\eqref{eq:cas3b} 
alors $y = \sqrt{b(t)} \tilde{y}$ 
est solution de~\eqref{eq:cas3} dans $\mathscr{Y}$.


Pour la suite, on notera seulement $y$ la solution de~\eqref{eq:cas3b}.
On définit alors l'ensemble : 

	\[ W = \left\{ y \; | \; y = (w,\mu) \in H_R^1([0,L])\times \mathbb{R},
 	\; w(0)= \mu, \right\} \]
 
  Et 
   \[\mathscr{Y} = L^2(0,L)\times \mathbb{R}\] 
   
   avec la norme associée :

  \[\| y\|_{\mathscr{Y}}^2 = \int_0^L w^2 + \epsilon \mu^2 \]


 On cherche à définir l'opérateur $A(t)$ tel que :
 \[ \forall y \in D(A(t)) \subset \mathscr{Y}, \; \; 
 \left( \begin{array}{c}
 \dot{w}\\
 \dot{\mu}
 \end{array} \right)
 = A(t) \left( \begin{array}{c}
 w\\
 \mu\\
 \end{array} \right) 
 = \left(\begin{array}{c}
 b(t) \partial_x w + \epsilon \partial_{xx} w - \frac{\dot{b} (t) }{2b(t) } w \\
 b(t) \partial_x w(0) - \frac{\dot{b}(t)}{2b(t)} \mu
 \end{array}\right) \]
 
 Soit $a_t$ une forme bilinéaire telle que :
 \[ \forall (y_1,y_2) \in D(A), \; 
 a(t,y_1,y_2) \equiv - \langle A(t) y_1,y_2\rangle_{\mathscr{Y}} \]

 Il vient donc pour tout $(y_1,y_2) \in \mathscr{Y}$ :

 \[
 \begin{split}
 \displaystyle	\langle A(t) y_1,y_2\rangle_{\mathscr{Y}} 
	                   & =  \langle b(t) \partial_x w_1 + \epsilon \partial_{xx} w_1 
					  ,w_2\rangle
 					  + \epsilon \langle  b(t) \partial_x w_1(0) 
					  - \frac{\dot{b}(t)}{2b(t)} \mu_1,\mu_2 \rangle \\					  
                      &= \int_0^L \dep (t) (\partial_x w_1)w_2
                       + \int_0^L \epsilon (\partial_{xx} w_1)w_2
					    - \frac{\dot{b} (t) }{2b(t) } \int_0^L w_1w_2\\
					   & \; \; + \epsilon b(t) \partial_x w_1(0) \mu_2 
                       - \epsilon \frac{\dot{\dep}(t)}{2b(t)} \mu_1\mu_2\\					   
 					  & = \int_0^L \dep(t) (\partial_xw_1)w_2
 						  + [\epsilon (\partial_x w_1)w_2]_0^L 
 						  - \epsilon \int_0^L \partial_xw_1 \partial_xw_2
						  - \frac{\dot{b} (t) }{2b(t) } \int_0^L w_1w_2 \\
   					      & \; \; + \epsilon b(t) \partial_x w_1(0) \mu_2 
                          - \epsilon \frac{\dot{\dep}(t)}{2b(t)} \mu_1\mu_2\\
 						& = \int_0^L \dep (t) (\partial_xw_1)w_2
						   - \epsilon \partial_xw_1(0) \mu_2
						  -  \epsilon\int_0^L \partial_xw_1 \partial_xw_2
 					    - \frac{\dot{b} (t) }{2b(t) } \int_0^L w_1w_2 \\
 					   &\; \; + \epsilon b(t) \partial_x w_1(0) \mu_2 
                        - \epsilon \frac{\dot{\dep}(t)}{2b(t)} \mu_1\mu_2\\  						
  						& = \int_0^L \dep (t) (\partial_xw_1)w_2
						 -  \epsilon\int_0^L \partial_xw_1 \partial_xw_2
 					    - \frac{\dot{b} (t) }{2b(t) } \int_0^L w_1w_2 \\
 					   & \; \; + \epsilon (b(t)-1) \partial_x w_1(0) \mu_2 
                        - \epsilon \frac{\dot{\dep}(t)}{2b(t)} \mu_1\mu_2   
\end{split}
\]

\begin{lemme}
	Soit $b \in C^1([0,\tau])$, tel que $\dot(b)>0$.
 	On définit la famille,
	$a(t,\cdot, \cdot): D(A(t)) \times D(A(t)) \to \mathbb{R}$ par :
 	\[
 	\forall (y_1,y_2) \in D(A(t)) \times D(A(t))
	\]
	\begin{equation}
		\label{def:a3}
		\begin{split}
		 a(t,y_1,y_2) &=   \epsilon \int_0^L \partial_xw_1 \partial_xw_2
		                + \frac{\dot{b} (t) }{2b(t) } \int_0^L w_1w_2
		                - \int_0^L \dep (t) (\partial_xw_1)w_2 \\
						& \; \; - \epsilon (b(t)-1) \partial_x w_1(0) \mu_2
                        + \epsilon \displaystyle \frac{\dot{\dep}(t)}{2b(t)}\mu_1\mu_2
		\end{split}
	\end{equation}
	Alors, pour tout $t>0$, la forme bilinéaire $a(t, \cdot, \cdot)$ est $W$-$\mathscr{Y}$ coercive sur $D(A(t)$ (?)
\end{lemme}

\begin{preuve}
	Soient $y=(w,\mu)$ dans $D(A(t)) \subset {\mathscr{Y}}$, et $\lambda >0$,
	calculons la quantité :
	\[
	\begin{split}
		 \langle A(t) y,y \rangle_W + \lambda \|y\|_{\mathscr{Y}} 
		      & \geq \langle A(t) y,y \rangle_{\mathscr{Y}} + \lambda \|y\|_{\mathscr{Y}} \\
		      &=  \epsilon \int_0^L (\partial_xw)^2 
		       + \frac{\dot{b} (t) }{2b(t) } \int_0^L w^2
		       - \int_0^L b(t) (\partial_xw)w \\
			   & \; \; - \epsilon (b(t)-1) \partial_x w(0) \mu
			   + \epsilon \displaystyle \frac{\dot{\dep}(t)}{2b(t)}\mu^2 \\
               & \; \; + \lambda \int_0^L w^2 
			    +\lambda \epsilon \mu^2 \\
			  &= \epsilon \int_0^L (\partial_xw)^2 
		       + (\frac{\dot{b} (t) }{2b(t) } +\lambda) \int_0^L w^2
		       - b(t)\displaystyle \frac{1}{2} [w^2]_0^L
			   - \epsilon (b(t)-1) \partial_x w(0) \mu
               + \epsilon \displaystyle (\frac{\dot{\dep}(t)}{2b(t)} + \lambda )\mu^2 \\
			  &= \epsilon \int_0^L (\partial_xw)^2 
		       + (\frac{\dot{b} (t) }{2b(t) } +\lambda) \int_0^L w^2
		       + b(t)\displaystyle \frac{1}{2} \mu^2
			   - \epsilon (b(t)-1) \partial_x w(0) \mu
               + \epsilon \displaystyle (\frac{\dot{\dep}(t)}{2b(t)} + \lambda )\mu^2\\
			  &= \epsilon \int_0^L (\partial_xw)^2 
		       + (\frac{\dot{b} (t) }{2b(t) } +\lambda) \int_0^L w^2
			   - \epsilon (b(t)-1) \partial_x w(0) \mu
               + (b(t) +\epsilon \displaystyle \frac{\dot{\dep}(t)}{2b(t)} + \epsilon\lambda )\mu^2 
    \end{split}
	\]
		
\end{preuve}	

\newpage
%%%%%%%%%%%%%%%%%%%%%%%%%%%%%%%%%%%%%%%%%%%%%%%%%%%%%%%%%%%%%%%%%%%%%%%%%%% REFERENCES

\medskip

\bibliographystyle{unsrt}%Used BibTeX style is unsrt
\bibliography{20190206biblio}
	
\end{document}

\end{document}


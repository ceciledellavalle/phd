%% PRE EDITION
\documentclass[a4paper]{article}
\usepackage[utf8]{inputenc}
\usepackage[T1]{fontenc}
\usepackage[french]{babel}
\usepackage{soul}
\usepackage[pdftex]{graphicx}

\usepackage{amsfonts}
\usepackage{amsthm}
\usepackage{amsmath}
\usepackage{amssymb}
\usepackage{mathrsfs}
\usepackage{booktabs}
\usepackage{siunitx}
\usepackage{thmtools}
\usepackage{ulem}

%% LAYOUT TITLE
\usepackage[explicit]{titlesec}
\usepackage{xcolor}
\usepackage{times}
\usepackage{tikz}
\usepackage{lipsum}
\titleformat{\subsection}
  {\color{blue!70}\large\sffamily\bfseries}
  {}
  {0em}
  {\colorbox{blue!10}{\parbox{\dimexpr\linewidth-2\fboxsep\relax}{\arabic{section}.\arabic{subsection}. #1}}}
  []

%% LAYOUT MATHS
\declaretheoremstyle[
    bodyfont=\normalfont\color{red},
    headfont=\color{red}
]{styleattention}

\declaretheoremstyle[
    spacebelow=1em
]{styleremarque}

\declaretheoremstyle[
    spaceabove=-6pt, 
    spacebelow=6pt, 
    headfont=\normalfont\bfseries, 
    bodyfont = \normalfont,
    postheadspace=1em, 
    qed=$\Box$, 
    headpunct={$\rhd$}
]{mystyle} 

\declaretheorem[thmbox=M,numberwithin=section,title=Définition]{definition}
\declaretheorem[thmbox=M,sibling=definition]{proposition}
\declaretheorem[thmbox=M,sibling=definition]{corollaire}
\declaretheorem[thmbox=M,sibling=definition,title=Théorème]{theoreme}
\declaretheorem[thmbox=M,sibling=definition]{lemme}
\declaretheorem[thmbox=M,sibling=definition,title=Propriété]{propriete}
\declaretheorem[thmbox=M,sibling=definition,title=Propriétés]{proprietes}
\declaretheorem[style=styleremarque,sibling=definition,title=Remarque]{remarque}
\declaretheorem[style=styleattention,title=À revoir]{Arevoir}
\declaretheorem[name={}, style=mystyle, unnumbered]{preuve}


\renewcommand\qedsymbol{$\blacksquare$}

%% NEW COMMAND
% solution
\newcommand{\y}{y}
% concentration monomers
\newcommand{\cmono}{c}
\newcommand{\pol}{a}
\newcommand{\dep}{b}
\newcommand{\mass}{\mathrm{M}}
\newcommand{\Y}{\mathscr{Y}}
\newcommand{\Ak}{A_{\kappa}}
\newcommand{\yk}{y_{\kappa}}

\usepackage{geometry}
\geometry{hmargin=3cm,vmargin=2.5cm}

\usepackage{tabularx}
\usepackage{float}

\title{$v \in C^0$, $\epsilon \ne 0$}
\author{Cécile Della Valle}

%% DEBUT DE REDACTION
\begin{document}

\maketitle

%%%%%%%%%%%%%%%%%%%%%%%%%%%%%%%%%%%%%%%%%%%%%%%%%%%%%%%%%%%%%%%%%%%%%%%%%%%%%%%%%%%%%%%%%%%%%%%%%%%%%%%%%%%%%%%%%%%%%%%%%%%%%%%%%
\section{Introduction}
%%%%%%%%%%%%%%%%%%%%%%%%%%%%%%%%%%%%%%%%%%%%%%%%%%%%%%%%%%%%%%%%%%%%%%%%%%%%%%%%%%%%%%%%%%%%%%%%%%%%%%%%%%%%%%%%%%%%%%%%%%%%%%%%%

Supposons $v$ une fonction connue, négative strictement 
(ce qui correspond pqr exemple au cas dans Lifschitz-Slyozov où $0<\dep -\mass $ ), 
et supposons $\epsilon>0$. 

On suppose que $v$ est bornée, et pout tout $t \in[0,\tau]$, 
$ \dep \leq |v(t)| \leq v_{\infty} $
On s'intéresse à l'équation :


\begin{equation}
\label{eq:cas3}
\begin{cases}
 \displaystyle \frac{\partial y}{\partial t}
 + v(t) \frac{\partial y} {\partial x}  
 - \epsilon \frac{\partial^2 y} {\partial x^2}
 = 0  & \forall (x,t) \in [0,L] \times [0, \tau]\\
 y(x,0) = y_{0} (x) & \forall x \in [0,L] \\
 \displaystyle \frac{\partial y}{\partial t}|_{x=0}
 + v(t) \frac{\partial y} {\partial x}|_{x=0} = 0 & \forall t \in [0,\tau]\\
 y(L,t)=0 & \forall t \in [0,\tau]
\end{cases}
\end{equation}

Puisque $v<0$, pour faciliter la lecture, on pose $\dep (t) = - v(t) >0$.



%%%%%%%%%%%%%%%%%%%%%%%%%%%%%%%%%%%%%%%%%%%%%%%%%%%%%%%%%%%%%%%%%%%%%%%%%%%%%%%%%%%%%%%%%%%%%%%%%%%%%%%%%%%%%%%%%%%%%%%%%%%%%%%%%
\section{Approche par pénalisation de la forme variationnelle}
%%%%%%%%%%%%%%%%%%%%%%%%%%%%%%%%%%%%%%%%%%%%%%%%%%%%%%%%%%%%%%%%%%%%%%%%%%%%%%%%%%%%%%%%%%%%%%%%%%%%%%%%%%%%%%%%%%%%%%%%%%%%%%%%%


\subsection{Présentation du problème}


On cherche la forme variationnelle $A(t)$ qui sera notre candidat
du générateur infinitésimal de semi-groupe, on pose : 

\[D(A(t)) = \left\{ y \; | \; y = (u,m) \in H^1([0,L])\times \mathbb{R},
 \; u(0)=\sqrt{\dep (t)} m, \; u(L)=0 \right\} \]
 
 
 
  
 \begin{proposition}
  	Soit $b \in C^1([0,\tau])$, convexe, tel que $\; \dot{b}>0$
	(on rappelle que $\forall t>0, \;  \dep (t) = |v(t)|$). 
	
	On munit $\Y = L^2([0,L]\times \mathbb{R})$ de la norme :
  
    \[\| y\|_{\mathscr{Y}}^2 = \int_0^L u^2 + \epsilon m^2 \]
	
   	On définit la famille,
  	$a(t,\cdot, \cdot): D(A(t)) \times D(A(t)) \to \mathbb{R}$ par :
   	\[
   	\forall (y_1,y_2) \in D(A(t)) \times D(A(t))
  	\]
  	\begin{equation}
  		\label{def:a3}
  		 a(t,y_1,y_2) =   \epsilon \int_0^L \partial_xu_1 \partial_xu_2
  		                 - \int_0^L \dep (t) (\partial_xu_1)u_2
                          +\epsilon \displaystyle \frac{\dot{\dep}(t)}{\dep(t)}m_1m_2
  	\end{equation}
  \end{proposition}


\begin{preuve}
 On cherche à définir l'opérateur $A(t)$ tel que :
 \[ \forall y \in D(A(t)) \subset \mathscr{Y}, \; \; 
 \left( \begin{array}{c}
 \dot{u}\\
 \dot{m}
 \end{array} \right)
 = A(t) \left( \begin{array}{c}
 u\\
 m\\
 \end{array} \right) 
 = \left(\begin{array}{c}
 \dep(t) \partial_x u + \epsilon \partial_{xx} u\\
 \sqrt{\dep(t)} \partial_x u(0) - \displaystyle \frac{\dot{\dep}(t)}{\dep(t)}m
 \end{array}\right) \]
 
 Soit $a_t$ une forme bilinéaire telle que :
 \[ \forall (y_1,y_2) \in D(A), \; 
 a(t,y_1,y_2) \equiv - \langle A(t) y_1,y_2\rangle_{\mathscr{Y}} \]

 Il vient donc pour tout $(y_1,y_2) \in \mathscr{Y}$ :

 \[
 \begin{split}
 	\langle A(t) y_1,y_2\rangle_{\mathscr{Y}} 
	                   & =  \langle \dep (t) \partial_x u_1 + \epsilon \partial_{xx}u_1 ,u_2\rangle
 					  + \epsilon \langle \sqrt{\dep(t)} \partial_x u_1(0) 
					                     - \displaystyle \frac{\dot{\dep}(t)}{\dep(t)}m_1 ,m_2 \rangle \\	
                         &= \int_0^L \dep (t) (\partial_x u_1)u_2
                            + \int_0^L \epsilon (\partial_{xx} u_1)u_2
							+ \epsilon \sqrt{\dep(t)} \partial_x u_1(0)m_2 
                             - \epsilon \displaystyle \frac{\dot{\dep}(t)}{\dep(t)}m_1m_2\\
 						& =   \int_0^L \dep(t) (\partial_xu_1)u_2
 						   + [\epsilon (\partial_x u_1)u_2]_0^L 
 						  - \epsilon \int_0^L \partial_xu_1 \partial_xu_2
						+ \epsilon [\sqrt{\dep(t)}m_2]  \partial_x u_1(0)
                           - \epsilon \displaystyle \frac{\dot{\dep}(t)}{\dep(t)}m_1m_2\\
 						& = \int_0^L \dep (t) (\partial_xu_1)u_2
						   - \epsilon \partial_xu_1(0)u_2(0)
						    -  \epsilon\int_0^L \partial_xu_1 \partial_xu_2
   						+ \epsilon \partial_xu_1(0) u_2(0) 
                         - \epsilon \displaystyle \frac{\dot{\dep}(t)}{\dep(t)}m_1m_2\\
  						& = \int_0^L \dep (t) (\partial_xu_1)u_2
						 - \epsilon \int_0^L \partial_xu_1 \partial_xu_2
                          - \epsilon \displaystyle \frac{\dot{\dep}(t)}{\dep(t)}m_1m_2   
\end{split}
\]

\end{preuve}

\begin{remarque}
	La coercivité de cette forme variationnelle dépend du signe de $\dot{b}$.
\end{remarque}

\subsection{Pénalisation de la forme variationnelle}

On définit la forme linéaire ci-dessous :

\[ 
\begin{split}
	a_\kappa(y_1,y_2,t) &= -\langle \Ak(t)y_1, y_2 \rangle_{\Y}\\
	             &= \epsilon \int_0^L \partial_xu_1 \partial_xu_2
  		              - \int_0^L \dep (t) (\partial_xu_1)u_2
                      +\epsilon \displaystyle \frac{\dot{\dep}(t)}{\dep(t)}m_1m_2
				      +\epsilon \kappa^{-1}(u_1(0)-\sqrt{\dep}m_1)(u_2(0)-\sqrt{\dep}m_2)
\end{split}
\]

Et on munit $\Y = L^2([0,L]\times \mathbb{R})$ de la norme :
  
    \[\| y\|_{\mathscr{Y}}^2 = \int_0^L u^2 + \epsilon u(0)^2 + \epsilon m^2 \]

\vspace{0.3cm}
\underline{Etape 1 : Reconstruction du problème fort}

\[
\begin{split}
	a_\kappa(y_1,y_2,t) &= \epsilon \int_0^L \partial_xu_1 \partial_xu_2
  		            - \int_0^L \dep (t) (\partial_xu_1)u_2
                    +\epsilon \displaystyle \frac{\dot{\dep}(t)}{\dep(t)}m_1m_2
				    + \epsilon \kappa^{-1}(u_1(0)-\sqrt{\dep}m_1)(u_2(0)-\sqrt{\dep}m_2)\\
				 &= - \epsilon \int_0^L \partial_{xx}u_1 u_2 
				    + \epsilon \partial_xu_1(0)u_2(0)
					- \int_0^L \dep (t) (\partial_xu_1)u_2
					+\epsilon \displaystyle \frac{\dot{\dep}(t)}{\dep(t)}m_1m_2\\
					& +\epsilon \kappa^{-1}(u_1(0)-\sqrt{\dep}m_1)u_2(0)
					-\epsilon \kappa^{-1}(u_1(0)-\sqrt{\dep}m_1) \sqrt{\dep}m_2\\
   				 &= -  \int_0^L (\epsilon \partial_{xx}u_1 +\dep (t)\partial_xu_1)u_2 \\
   				 & + \epsilon (\partial_xu_1(0) + \kappa^{-1}(u_1(0)-\sqrt{\dep}m_1)) u_2(0)\\
   				 & - \epsilon (-\displaystyle \frac{\dot{\dep}(t)}{\dep(t)}m_1
   					+ \kappa^{-1}\sqrt{\dep}(u_1(0)-\sqrt{\dep}m_1) )m_2
\end{split}
\]

On peut donc réécrire cette forme bilinéaire avec l'opérateur $\Ak(t)$ :

 \[ \forall (y_1,y_2) \in D(\Ak (t)), \; 
 a_\kappa(t,y_1,y_2) \equiv - \langle A(t) y_1,y_2\rangle_{\mathscr{Y}} \]
 

et :
 
 \[ \forall y \in D(A_\kappa (t)) \subset \Y, \; \; 
 \left( \begin{array}{c}
 \dot{u}\\
 \dot{m}
 \end{array} \right)
 = \Ak (t) \left( \begin{array}{c}
 u\\
 m\\
 \end{array} \right) 
 = \left(\begin{array}{c}
 \epsilon \partial_{xx} u +\dep (t)\partial_xu )\\
- \displaystyle \frac{\dot{\dep}(t)}{\dep(t)}m 
+ \kappa^{-1}\sqrt{\dep}(u(0)-\sqrt{\dep}m)
 \end{array}\right) \]
 
 Le système fort s'écrit donc :
 
 \begin{equation}
 \label{eq:cas3b}
 \begin{cases}
  \partial_t u -\epsilon \partial_{xx} u - \dep (t)\partial_x u 
   = 0 & \forall (x,t) \in [0,L] \times [0, \tau]\\
  u(x,0) = u_{0} (x) & \forall x \in [0,L] \\
  \partial_t m + \displaystyle \frac{\dot{\dep}(t)}{\dep(t)}m 
   + \kappa^{-1} \dep m = \kappa^{-1}\sqrt{\dep}u(0) 
   & \forall t \in [0,\tau]\\
  u(L,t)=0 & \forall t \in [0,\tau]
 \end{cases}
 \end{equation}
 
 QUESTION :

 Comment retrouve-t-on l'équation :
 
 \[  \partial_xu(0) + \kappa^{-1} u(0) = \kappa^{-1} \sqrt{\dep}m \] ?
 
 
 Dans le domaine $D(\Ak (t))$ ?
 
 \[D(\Ak(t) = \{ (u,m) \in H_R^1([0,L])\times \mathbb{R} | \; \;
                   \partial_xu(0) + \kappa^{-1} u(0) = \kappa^{-1} \sqrt{\dep}m \} \]

\vspace{0.3cm}
\underline{Etape 2 : Convergence de de la solution pénalisée}

Soit $y$ solution du problème $\dot{y}=A(t)y$ 
et $\yk$ solution de $\dot{\yk} = \Ak(t) \yk$.
On pose $\tilde{y}= y-\yk $. 

Calculons la quantité $\Ak(t) y$ :

\[
\Ak(t) y =  
\left(\begin{array}{c}
 \epsilon \partial_{xx} u +\dep (t)\partial_xu )\\
- \displaystyle \frac{\dot{\dep}(t)}{\dep(t)}m 
+ \kappa^{-1}\sqrt{\dep}(u(0)-\sqrt{\dep}m)
 \end{array}\right) 
 =
 \left(\begin{array}{c}
 0 \\
 - \displaystyle \frac{\dot{\dep}(t)}{\dep(t)}m 
  \end{array}\right) 
\]

Donc $\tilde{y}$ vérifie l'équation :

\[ \dot{\tilde{y}} = \Ak(t) \tilde{y} 
                  + \left(\begin{array}{c}
                         0 \\
                         - \displaystyle \frac{\dot{\dep}(t)}{\dep(t)} m 
                       \end{array}\right)
\]


QUESTION :

Comment assurer que $\tilde{y} \rightarrow 0$ ?

Comment gérer le fait que $y$ n'appartient pas à $D(\Ak (t))$ avec la définition qui fait intervenir $m$ ?

Il me semble qu'on ne fait pas intervenir le fait que $\yk$ vérifie $\partial_xu(0) + \kappa^{-1} u(0) = \kappa^{-1} \sqrt{\dep}m$ ?



\vspace{0.3cm}
\underline{Etape 3 : Coercivité de l'opérateur}

\[
\displaystyle a_\kappa(y,y,t) = \epsilon \int_0^L (\partial_xu)^2 
  		            + \dep (t) \frac{1}{2}u(0)^2
                    + \epsilon  \frac{\dot{\dep}(t)}{\dep(t)}m^2
				    + \epsilon \kappa^{-1}(u(0)-\sqrt{\dep}m)^2
\]

La coercivité de $a_\kappa$ est donc encore liée au signe de la dérivée de $\dot{b}$.


                            


\newpage
%%%%%%%%%%%%%%%%%%%%%%%%%%%%%%%%%%%%%%%%%%%%%%%%%%%%%%%%%%%%%%%%%%%%%%%%%%% REFERENCES

\medskip

\bibliographystyle{unsrt}%Used BibTeX style is unsrt
\bibliography{20190206biblio}
	
\end{document}



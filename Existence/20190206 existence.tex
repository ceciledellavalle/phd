%% PRE EDITION
\documentclass[a4paper]{article}
\usepackage[utf8]{inputenc}
\usepackage[T1]{fontenc}
\usepackage[french]{babel}
\usepackage{soul}
\usepackage[pdftex]{graphicx}

\usepackage{amsfonts}
\usepackage{amsthm}
\usepackage{amsmath}
\usepackage{amssymb}
\usepackage{mathrsfs}
\usepackage{booktabs}
\usepackage{siunitx}
\usepackage{thmtools}
\usepackage{ulem}

%% LAYOUT
\declaretheoremstyle[
    bodyfont=\normalfont\color{red},
    headfont=\color{red}
]{styleattention}

\declaretheoremstyle[
    spacebelow=1em
]{styleremarque}

\declaretheoremstyle[
    spaceabove=-6pt, 
    spacebelow=6pt, 
    headfont=\normalfont\bfseries, 
    bodyfont = \normalfont,
    postheadspace=1em, 
    qed=$\Box$, 
    headpunct={$\rhd$}
]{mystyle} 

\declaretheorem[thmbox=M,numberwithin=section,title=Définition]{definition}
\declaretheorem[thmbox=M,sibling=definition]{proposition}
\declaretheorem[thmbox=M,sibling=definition]{corollaire}
\declaretheorem[thmbox=M,sibling=definition,title=Théorème]{theoreme}
\declaretheorem[thmbox=M,sibling=definition]{lemme}
\declaretheorem[thmbox=M,sibling=definition,title=Propriété]{propriete}
\declaretheorem[thmbox=M,sibling=definition,title=Propriétés]{proprietes}
\declaretheorem[style=styleremarque,sibling=definition,title=Remarque]{remarque}
\declaretheorem[style=styleattention,title=À revoir]{Arevoir}
\declaretheorem[name={}, style=mystyle, unnumbered]{preuve}


\renewcommand\qedsymbol{$\blacksquare$}

%% NEW COMMAND
% solution
\newcommand{\y}{y}
% concentration monomers
\newcommand{\cmono}{c}
\newcommand{\pol}{a}
\newcommand{\dep}{b}
\newcommand{\mass}{\mathrm{M}}

\usepackage{geometry}
\geometry{hmargin=3cm,vmargin=2.5cm}

\usepackage{tabularx}
\usepackage{float}

\title{To be determined}
\author{Cécile Della Valle}

%% DEBUT DE REDACTION
\begin{document}

\maketitle

%%%%%%%%%%%%%%%%%%%%%%%%%%%%%%%%%%%%%%%%%%%%%%%%%%%%%%%%%%%%%%%%%%%%%%%%%%%%%%%%%%%%%%%%%%%%%%%%%%%%%%%%%%%%%%%%%%%%%%%%%%%%%%%%%
%%%%%%%%%%%%%%%%%%%%%%%%%%%%%%%%%%%%%%%%%%%%%%%%%%%%%%%%%%%%%%%%%%%%%%%%%%%%%%%%%%%%%%%%%%%%%%%%%%%%%%%%%%%%%%%%%%%%%%%%%%%%%%%%%
%%%%%%%%%%%%%%%%%%%%%%%%%%%%%%%%%%%%%%%%%%%%%%%%%%%%%%%%%%%%%%%%%%%%%%%%%%%%%%%%%%%%%%%%%%%%%%%%%%%%%%%%%%%%%%%%%%%%%%%%%%%%%%%%%
%%%%%%%%%%%%%%%%%%%%%%%%%%%%%%%%%%%%%%%%%%%%%%%%%%%%%%%%%%%%%%%%%%%%%%%%%%%%%%%%%%%%%%%%%%%%%%%%%%%%%%%%%%%%%%%%%%%%%%%%%%%%%%%%%
%%%%%%%%%%%%%%%%%%%%%%%%%%%%%%%%%%%%%%%%%%%%%%%%%%%%%%%%%%%%%%%%%%%%%%%%%%%%%%%%%%%%%%%%%%%%%%%%%%%%%%%%%%%%%%%%%%%%%%%%%%%%%%%%%
\section{Introduction}
%%%%%%%%%%%%%%%%%%%%%%%%%%%%%%%%%%%%%%%%%%%%%%%%%%%%%%%%%%%%%%%%%%%%%%%%%%%%%%%%%%%%%%%%%%%%%%%%%%%%%%%%%%%%%%%%%%%%%%%%%%%%%%%%%
%%%%%%%%%%%%%%%%%%%%%%%%%%%%%%%%%%%%%%%%%%%%%%%%%%%%%%%%%%%%%%%%%%%%%%%%%%%%%%%%%%%%%%%%%%%%%%%%%%%%%%%%%%%%%%%%%%%%%%%%%%%%%%%%%

Soit $\tau>0$,
soit $v$ une fonction continue, $v \in C^0([0,\tau])$ strictement négative.
On souhaite démontrer l'existence d'une solution dans $C^0([0,\tau],\mathscr{Y})$ du problème de Cauchy par la théorie des semi-groupes :

\begin{equation}
\label{eq:general}
\begin{cases}
 \displaystyle \frac{\partial y}{\partial t}
 + v(t) \frac{\partial y} {\partial x}  
 - \epsilon \frac{\partial^2 y} {\partial x^2}
 = 0  & \forall (x,t) \in [0,L] \times [0, \tau]\\
 y(x,0) = y_{0} (x) & \forall x \in [0,L] \\
 \displaystyle \frac{\partial y}{\partial t}|_{x=0}
 + v(t) \frac{\partial y} {\partial x}|_{x=0} = 0 & \forall t \in [0,\tau]\\
 y(L,t) = 0 & \forall t \in [0,\tau]
\end{cases}
\end{equation}

Dans une première partie, on s'intéresse aux cas particuliers ($v$ est une fonction constante négative, et $\epsilon$ est nul), pour progressivement revenir sur ces hypothèses simplificatrices.

Dans une deuxième partie, on cherche démontrer l'existence et l'unicité de solutions
de l'équation de Riccati associée à l'observation de moments de la solution.
Ce problème mal posé sera régularisé par une méthode de Tikhonov.

Enfin la dernière partie portera sur l'étude de la résolution numérique 
du filtre de Kalman.

\newpage
%%%%%%%%%%%%%%%%%%%%%%%%%%%%%%%%%%%%%%%%%%%%%%%%%%%%%%%%%%%%%%%%%%%%%%%%%%%%%%%%%%%%%%%%%%%%%%%%%%%%%%%%%%%%%%%%%%%%%%%%%%%%%%%%%
%%%%%%%%%%%%%%%%%%%%%%%%%%%%%%%%%%%%%%%%%%%%%%%%%%%%%%%%%%%%%%%%%%%%%%%%%%%%%%%%%%%%%%%%%%%%%%%%%%%%%%%%%%%%%%%%%%%%%%%%%%%%%%%%%
%%%%%%%%%%%%%%%%%%%%%%%%%%%%%%%%%%%%%%%%%%%%%%%%%%%%%%%%%%%%%%%%%%%%%%%%%%%%%%%%%%%%%%%%%%%%%%%%%%%%%%%%%%%%%%%%%%%%%%%%%%%%%%%%%
%%%%%%%%%%%%%%%%%%%%%%%%%%%%%%%%%%%%%%%%%%%%%%%%%%%%%%%%%%%%%%%%%%%%%%%%%%%%%%%%%%%%%%%%%%%%%%%%%%%%%%%%%%%%%%%%%%%%%%%%%%%%%%%%%
%%%%%%%%%%%%%%%%%%%%%%%%%%%%%%%%%%%%%%%%%%%%%%%%%%%%%%%%%%%%%%%%%%%%%%%%%%%%%%%%%%%%%%%%%%%%%%%%%%%%%%%%%%%%%%%%%%%%%%%%%%%%%%%%%
\section{Existence de solution pour le problème direct}
%%%%%%%%%%%%%%%%%%%%%%%%%%%%%%%%%%%%%%%%%%%%%%%%%%%%%%%%%%%%%%%%%%%%%%%%%%%%%%%%%%%%%%%%%%%%%%%%%%%%%%%%%%%%%%%%%%%%%%%%%%%%%%%%%
%%%%%%%%%%%%%%%%%%%%%%%%%%%%%%%%%%%%%%%%%%%%%%%%%%%%%%%%%%%%%%%%%%%%%%%%%%%%%%%%%%%%%%%%%%%%%%%%%%%%%%%%%%%%%%%%%%%%%%%%%%%%%%%%%




%%%%%%%%%%%%%%%%%%%%%%%%%%%%%%%%%%%%%%%%%%%%%%%%%%%%%%%%%%%%%%%%%%%%%%%%%%%%%%%%%%%%%%%%%%%%%%%%%%%%%%%%%%%%%%%%%%%%%%%%%%%%%%%%%
\subsection{$v=$cste, $\epsilon = 0$}
%%%%%%%%%%%%%%%%%%%%%%%%%%%%%%%%%%%%%%%%%%%%%%%%%%%%%%%%%%%%%%%%%%%%%%%%%%%%%%%%%%%%%%%%%%%%%%%%%%%%%%%%%%%%%%%%%%%%%%%%%%%%%%%%%



On suppose dans un premier temps que $v$ est une constante, et on suppose,
sans perte de généralité, 
que cette constante est négative $v<0$.
De plus on suppose que la constante $\epsilon$ est nulle.

L'équation~\eqref{eq:general} devient :

\begin{equation}
\label{eq:cas0}
\begin{cases}
 \displaystyle \frac{\partial y}{\partial t}
 + v \frac{\partial y} {\partial x}  
 = 0  & \forall (x,t) \in [0,L] \times [0, \tau]\\
 y(x,0) = y_{0} (x) & \forall x \in [0,L] \\
 y(L,t) = 0 & \forall t \in [0,\tau]
\end{cases}
\end{equation}

Soit $\mathscr{Y} = L^2([0,L])$ l'espace de Banach et l'opérateur $A$ sur $D(A)$ tel que :

\begin{equation} 
	\label{def:A0}
	\forall y \in D(A), \; \; Ay = -v \partial_x y
\end{equation}

\[ D(A)=\left\{y \; | \; y \; absoluement \; continue \; sur \; [0,L],\; y(L) =0| \right\}\]


\begin{proposition}
	\label{prop:cas0}
	L'opérateur $A: D(A) \to \mathscr{Y}$ défini par~\eqref{def:A0}
	est générateur d'un semi-groupe $C_0$ de contraction dans $\mathscr{Y}$.
\end{proposition}

\begin{preuve}
	On souhaite appliquer le théorème de Lumer-Phillips, 
	il nous faut donc démontrer que $A$ possède les trois propriétés suivantes :
\begin{itemize}
	\item $(i)$ $A$ est un opérateur fermé et $\overline{D(A)} = \mathscr{Y}$ ;
	\item $(ii)$ $A$ est dissipatif ;
	\item $(iii)$ il existe $\lambda_0$ tel que $\lambda_0 - A : D(A) \to \mathscr{Y}$ est surjectif.
\end{itemize}

\vspace{0.3cm}
$(i)$

L'opérateur A est défini sur l'ensemble $D(A)$ des fonctions absoluement continues sur $[0,L]$ qui s'annulent en $L$. 

En effet, $y$ est absolument continue sur un compact $\Omega$ si et seulement si pour presque tout $x \in \Omega$, $y$ est dérivable en $x$ de dérivée $y'\in L^1(\Omega)$ et
pour tout $x$ de $\Omega$ : $y(x)=y(0)+\int_0^x y'(s)ds$ 
(d'après le chapitre 7 du livre de W. Rudin \cite{WRudin}).

Sur cet ensemble, l'opérateur dérivation est fermée, densément défini.
(démonstration ? très difficile).

\vspace{0.3cm}
$(ii)$

Montrons que $A$ est dissipatif.
Soit $y \in \mathscr{Y}$, 
calculons la quantité :

\[ \begin{split}
\langle Ay,y \rangle_{\mathscr{Y}} &= \langle v \partial_x y, y \rangle_{\mathscr{Y}}  \\
                     &= v \int_0^L y(z) \partial_x y(z) d(z) \\
					 & = v[y(z)^2]_{z=0}^L - v \int_0^\infty y(z) \partial_x y(z) d(z)
\end{split} \]

Soit :

\[ \langle Ay,y \rangle_{\mathscr{Y}} = -\frac{1}{2}vy(0)^2 \leq 0 \]

Donc $\forall$ $y \in \mathscr{Y}$, $\langle Ay,y \rangle_{\mathscr{Y}} \leq 0$.


\vspace{0.3cm}
$(iii)$

Soit $y_1 \in \mathscr{Y}$ et $\lambda >0$.

On cherche une solution $y$ tel que $(\lambda -A)y = y_1$.

Alors on a de façon équivalente que $y$ est solution de :
\[ \begin{cases}
\lambda y - v y' = y_1 \\
y(L) = 0
\end{cases}\]

donc, pour tout $y_1 \in \mathscr{Y}$, cette équation possède une unique solution donnée par la formule de Duhamel :

\[ y(x) =  \int_x^L e^{\lambda/v (x-x')}y_1(x')dx' \]

(On peut par ailleurs vérifier que $ \| y \|_{\mathscr{Y}} \leq 1/ \lambda \|y_1 \|_{\mathscr{Y}}$).

Donc d'après $(i)$,$(ii)$ et $(iii)$, on peut appliquer le théorème de Lumer-Phillips, et $A$ est le générateur infinitésimal d'un semi-group $C_0$.
On a donc l'existence d'une solution de~\eqref{eq:cas0} 
dans $C^0([0,\tau],L^2([0,L])$.
	
\end{preuve}

Dans ce cas simple, on peut donner une formule explicite de l'opérateur de semi-group $C_0$ :

\begin{lemme}
	\label{lem:cas0}
	Pour tout $t\in[0.\tau]$, le semi-groupe $C_0$ généré par $A$ 
	l'opérateur infinitésimal associé à l'équation~\eqref{eq:cas0} est la translation :
	\[ \forall y \in D(A(t)), \; \; S(s)y = y(x - vs) \chi_{0\leq x-vs \leq L}\]
\end{lemme}

\begin{theoreme}[Existence et unicité]
	Soit soit $\tau>0$ et $v<0$,
	soit $\mathscr{Y}$ un espace de Banach, et le  
	de générateur infinitésimal du semi-groupe $C_0$ sur $\mathscr{Y}$
	défini par ~\eqref{def:A0}.
	Alors le problème de Cauchy~\eqref{eq:cas0} a une unique solution dans $C^0([0,tau],\mathscr{Y})$ qui s'écrit :
	\[ \forall s\in[0,\tau] \; \; y(s) = S(s)y_0 \]
	où $S$ est le semi-groupe d'évolution donné par lemme~\ref{lem:cas0}.
\end{theoreme}

%%%%%%%%%%%%%%%%%%%%%%%%%%%%%%%%%%%%%%%%%%%%%%%%%%%%%%%%%%%%%%%%%%%%%%%%%%%%%%%%%%%%%%%%%%%%%%%%%%%%%%%%%%%%%%%%%%%%%%%%%%%%%%%%%
\subsection{$v \in C^0$, $\epsilon = 0$}
%%%%%%%%%%%%%%%%%%%%%%%%%%%%%%%%%%%%%%%%%%%%%%%%%%%%%%%%%%%%%%%%%%%%%%%%%%%%%%%%%%%%%%%%%%%%%%%%%%%%%%%%%%%%%%%%%%%%%%%%%%%%%%%%%


On suppose cette fois que $v \in C^0$ est une fonction donnée, et de plus que $v$ ne s'annule pas. 
Sans perdre de généralité on suppose que $\forall t \in [0,\tau]$ on a $v(t)<0$.

\begin{equation}
\label{eq:cas1}
\begin{cases}
 \displaystyle \frac{\partial y}{\partial t}
 + v(t) \frac{\partial y} {\partial x}  
 = 0  & \forall (x,t) \in [0,L] \times [0, \tau]\\
 y(x,0) = y_{0} (x) & \forall x \in [0,L]\\
 y(L,0) = 0 & \forall t \in [0,\tau]
\end{cases}
\end{equation}

Soit $\mathscr{Y} = L^2([0,\infty))$ l'espace de Banach et l'opérateur d'évolution $A(t)$ définit sur $D(A(t))$ tel que :

\begin{equation}
	\label{def:A1}
	\forall y \in D(A(t)), \; \; A(t)y = -v(t)\partial_x y 
\end{equation}

Avec 

\[\forall t \in [0,\tau], \; D(A(t)) = 
\left\{y \; | \; y \; absoluement \; continue \; sur \; [0,L],\; y(L) =0| \right\}\]

\begin{proposition}
	Soit $\mathscr{Y}$ un espace de Banach, pour $t>0$, 
	l'opérateur $A(t)$ est le générateur infinitésimal de semi-groupe $C_0$,
	noté $S_t$.
\end{proposition}

\begin{preuve}
	Pour $t>0$ fixé, on est ramené au cas de la proposition~\ref{prop:cas0}.
\end{preuve}

\begin{proposition}
	\label{prop:stable0}
	Soit soit $\tau>0$, soit $v \in C^0([0,\tau])$ telle que $\forall t \in [0,\tau]$ on a $v(t)<0$.
	Soit $\mathscr{Y}$ un espace de Banach, la famille $(A(t))_{t \in [0,\tau]}$ 
	définie par~\eqref{def:A1}
	de générateurs infinitésimals de semi-groupes $C_0$ sur $\mathscr{Y}$ est stable. 
\end{proposition}

\begin{preuve}
	Rappelons les conditions nécessaires pour que  $(A(t))_{t \in [0,\tau]}$  
	soit une famille stable :
	il existe $M>1$ et $\omega>0$ tels que pout tout $t \in [0,\tau]$ 
	
	\[ 
	\begin{cases}
		(i) \; \; ] \omega; +\infty[ \subset \rho(A(t)) & \\
		(ii) \; \; \|  \displaystyle \prod_{j=1}^{k} R(\lambda:A(t_j))\| \leq M(\lambda-\omega)^{-k} & \forall \; t_0<..<t_j<..<t_k
	\end{cases}
	\]
	
	La condition $(i)$ est immédiatement vérifée. En effet, puisque pour tout $t>0$
	chaque générateur infinitésimal $A(t)$ vérifie $ \mathbb{R}^{+*} \subset \rho(A(t)) $ 
	(en particulier, dans notre cas puisque pour tout $t>0$, pour tout $\lambda >0$, 
	on a que $\lambda - A$ est un opérateur surjectif, donc $\mathbb{R} \subset \rho(A(t)) $).
	
	Pour la condition $(ii)$ calculons explicitement la norme d'un opérateur $R(\lambda:A(t_j))$ :
	
	\[
	\begin{split}
		\displaystyle \|R(\lambda:A(t_j))y_1 \|^2 & = \int_0^L |\int_x^L e^{\lambda/v(t_j)(x-x')}y_1(x')dx'|^2dx \\
		                                          & \leq \int_0^L (\int_x^L e^{2\lambda/v(t_j)(x-x')}dx') (\int_x^L] y_1(x')^2 dx') dx \\
												  & \leq \|y_1\|^2 \int_0^L (\int_x^L e^{2\lambda/v(t_j)(x-x')}dx')dx \\
												  & \leq \|y_1\|^2 \int_0^L \displaystyle \frac{-v(t_j)}{2\lambda}[e^{2\lambda/v(t_j)(x-x')}]_x^L dx \\
												  & \leq \|y_1\|^2 \displaystyle \frac{-v(t_j)}{2\lambda} \int_0^L (e^{2\lambda/v(t_j)(x-L)}-e^{2\lambda/v(t_j)x}) dx \\
												  & \leq \|y_1\|^2 \displaystyle |\frac{v(t_j)}{2\lambda}|^2(2e^{2\lambda/v(t_j)L}-1)
	\end{split}
	\]
	
	
	
	Sachant que $v$ est une fonction continue donnée, on pose $V = \|v\|_{\infty}$ 
	et on choisit les constantes de stabilité :
	
	\begin{equation}
		\begin{cases}
			M = 1 \\
			\omega = \frac{2}{V}e^{-\lambda/v(t_j)L}
		\end{cases}
	\end{equation}
	
	Alors :
	
	\[
	\begin{split}
		\displaystyle \|R(\lambda:A(t_j)) \| &\leq  \frac{V}{2\lambda} e^{\lambda/v(t_j)L}\\
		                                     &\leq \frac{ V e^{\lambda/v(t_j)L} \omega}{2\lambda \omega}\\
											 &\leq M \frac{1}{|\lambda - \omega|}
	\end{split}
	\]
	
 	Donc pour tout $j>0$ on a l'inégalité :
	
 	\begin{equation}
 		\displaystyle \|R(\lambda:A(t_j)) \|^2 \leq M \frac{1}{|\lambda - \omega|}
 	\end{equation}
	
	Ce qui donne l'inégalité $(ii)$.
\end{preuve}

\begin{proposition}
	\label{prop:cas1}
	Soit soit $\tau>0$, soit $v \in C^0([0,\tau])$ telle que $\forall t \in [0,\tau]$ on a $v(t)<0$.
	Soit $\mathscr{Y}$ un espace de Banach, la famille $(A(t))_{t \in [0,\tau]}$ 
	définie par ~\eqref{def:A1}
	de générateurs infinitésimals de semi-groupes $C_0$ sur $\mathscr{Y}$.
	Il existe une unique solution $U(t,s)$, $0\leq s \leq t \leq \tau $, qui satisfait :
	\begin{itemize}
		\item $\|U(t,s)\| \leq M e^{\omega(t-s)}$ pour tout $0\leq s \leq t\leq \tau$ ;
		\item $\partial_t^+ U(t,s)v|_{t=s} = A(sv)$ pour $v\in mathscr{Y}$, $0\leq s\leq \tau$ ;
		\item $\partial_s U(t,s)v = - U(t,s)A(s)v$ pour $v\in mathscr{Y}$, $0\leq s\leq t\leq \tau$.
	\end{itemize}
\end{proposition}

\begin{preuve}
	On souhaite appliquer le théorème du chap.5 de Pazy.
	On rappelle les hypothèses que doit vérifier la famille $(A(t))_{t \in [0,\tau]}$ pour appliquer 
	le théorème 3.1 page 135 livre de Pazy \cite{APazy}.
	\begin{itemize}
		\item $(i)$ $(A(t))$ est une famille d'opérateur stable pour les constantes $M$ et $\omega$ ;
		\item $(ii)$ $\mathscr{H}$ est $A(t)$-admissible pour $t \in [0,\tau]$ et la famille $(\tilde{A}(t))$
		de $A(t)$ dans $\mathscr{H}$ est une famille stable de $\mathscr{H}$ 
		pour les constantes $\tilde{M}$ et $\tilde{\omega}$ ;
		\item $(iii)$ Pour tout $t \in[0,\tau]$, $\mathscr{H} \subset D(A(t))$, $A(t)$ est un opérateur borné
		de $\mathscr{H}$ dans $\mathscr{Y}$ et $t \to A(t)$ est continue 
		pour la norme des opéraeurs bornés $\| \|_{\mathscr{H}\to \mathscr{Y}}$.
	\end{itemize}
	
	\vspace{0.3cm}
	La condition $(i)$ a été démontrée par la proposition ~\ref{prop:stable0}.
	
	\vspace{0.3cm}
	Démontrons maintenant la condition $(ii)$.
	On pose :
	\[\mathscr{H} = \left\{ y \; | \; y\in H^1([0,L]), \; y(L)=0 \right\} \]
	 
	  et la norme associée à cet espace $\| \cdot \|_{\mathscr{H}}$ est définie de la façon suivante :
	\[ \forall y \in \mathscr{H}, \; \|y\|_{\mathscr{H}} = \|y\|_{\mathscr{Y}} + \|y'\|_{\mathscr{Y}} \]
	
	Et de plus on définie la norme des applications bornées de $\mathscr{H}$ dans $\mathscr{Y}$:
	
	\[ \forall A \in B(\mathscr{H},\mathscr{Y}), 
	\; \| A \|_{\mathscr{H}\to \mathscr{Y}} =\sup_{y\in\mathscr{H}} \frac{\|Ay\|_{\mathscr{H}}}{\|y\|_{\mathscr{H}}} \]
	
	Alors $\mathscr{H}$ est fermé dans $\mathscr{Y}$ pour la norme $\| \|_{\mathscr{H}}$.
	On a de plus une formule explicite pour tout $t\in [0,\tau]$ :
	
	\begin{lemme}
		Pour tout $t\in[0,\tau]$, le semi-groupe $C_0$ généré par $(A(t)= - v(t)\partial_x \;)_{t \in [0,\tau]}$
		avec $v \in C^0([0,\tau])$ est la translation :
		\[ \forall y \in D(A(t)), \; \; U_t(s)y = y(x - v(t)s)\chi_{0\leq x-v(t)s \leq L} \]
	\end{lemme}
	
	Donc $\mathscr{H}$ sous-espace de $\mathscr{Y}$ est $A(t)$-admissible 
	puisque $\mathscr{H}$ est un espace invariant par la translation $U_t$
	et la restrinction de $U_t$ à $\mathscr{H}$ est un semi-groupe $C^0$ de $\mathscr{H}$.
	
	Enfin, chaque opérateur $A(t)$ est borné sur $\mathscr{H}$ ce qui achève la démonstration puisque $(iii)$ est également vérifiée.
\end{preuve}

\begin{theoreme}
	Soit soit $\tau>0$, soit $v \in C^0([0,\tau])$ telle que $\forall t \in [0,\tau]$ on a $v(t)<0$.
	Soit $\mathscr{Y}$ un espace de Banach, la famille $(A(t))_{t \in [0,\tau]}$ 
	définie par ~\eqref{def:A1}
	de générateurs infinitésimals de semi-groupes $C_0$ sur $\mathscr{Y}$.
	Alors le problème de Cauchy~\eqref{eq:cas1} a une unique solution dans $C^0([0,tau],\mathscr{Y})$ qui s'écrit :
	\[ \forall t\in[0,\tau] \; \; y(t) = U(t,0)y_0 \]
	où $U$ est le semi-groupe d'évolution donné par \ref{prop:cas1}.
\end{theoreme}

%%%%%%%%%%%%%%%%%%%%%%%%%%%%%%%%%%%%%%%%%%%%%%%%%%%%%%%%%%%%%%%%%%%%%%%%%%%%%%%%%%%%%%%%%%%%%%%%%%%%%%%%%%%%%%%%%%%%%%%%%%%%%%%%%
\subsection{$v =$ cste, $\epsilon \ne 0$}
%%%%%%%%%%%%%%%%%%%%%%%%%%%%%%%%%%%%%%%%%%%%%%%%%%%%%%%%%%%%%%%%%%%%%%%%%%%%%%%%%%%%%%%%%%%%%%%%%%%%%%%%%%%%%%%%%%%%%%%%%%%%%%%%%

Supposons $v=$ cste et $v<0$, supposons $\epsilon>0$. 
On s'intéresse à l'équation :


\begin{equation}
\label{eq:cas2}
\begin{cases}
 \displaystyle \frac{\partial y}{\partial t}
 + v \frac{\partial y} {\partial x}  
 - \epsilon \frac{\partial^2 y} {\partial x^2}
 = 0  & \forall (x,t) \in [0,L] \times [0, \tau]\\
 y(x,0) = y_{0} (x) & \forall x \in [0,L] \\
 \displaystyle \frac{\partial y}{\partial t}|_{x=0}
 + v \frac{\partial y} {\partial x}|_{x=0} = 0 & \forall t \in [0,\tau]\\
 y(L,t)=0 & \forall t \in [0,\tau]
\end{cases}
\end{equation}

Puisque $v<0$, pour facilioter la lecture, on pose $\dep = - v >0$.

On remarque que la principale difficulté par rapport à ce qui a été développé 
précédemment est la condition au bord en $x=0$.
Pour gérer cette condition, on introduit l'espace :

\[\mathscr{Y} = \left\{ y \; | \; y = (u,m) \in H^1([0,L])\times \mathbb{R},
 \; u(0)=m, \; u(L)=0 \right\} \]
 
 Avec la normne associée 
 \[\| y\|_{\mathscr{Y}}^2 = \int_0^L u^2 + \displaystyle \frac{\epsilon}{\dep} m^2 \]


On cherche à définir l'opérateur $A$ tel que :
\[ \forall y \in D(A) \subset \mathscr{Y}, \; \; 
\left( \begin{array}{c}
\dot{u}\\
\dot{m}
\end{array} \right)
= A \left( \begin{array}{c}
u\\
m\\
\end{array} \right) 
= \left(\begin{array}{c}
\dep \partial_x u + \epsilon \partial_{xx} u\\
\dep \partial_x u(0)
\end{array}\right) \]

Soit $a$ une forme bilinéaire telle que :
\[ \forall (y_1,y_2) \in D(A), \; a(y_1,y_2) = \langle A y_1,y_2\rangle_{\mathscr{Y}} \]

Il vient donc pour tout $(y_1,y_2) \in \mathscr{Y}$ :

\[
\begin{split}
	\langle A y_1,y_2\rangle_{\mathscr{Y}} & = 
	                  \langle \dep \partial_x u_1 + \epsilon \partial_{xx}u_1 ,u_2\rangle
					  + \displaystyle \frac{\epsilon}{\dep} \langle \dep \partial_x u_1(0),m_2\rangle \\	
                        &= \int_0^L b (\partial_xu_1)u_2
                           + \int_0^L \epsilon (\partial_{xx} u_1)u_2
                           + \displaystyle \frac{\epsilon}{\dep} \dep \partial_xu_1(0)m_2\\
						& =   \int_0^L b (\partial_xu_1)u_2
						   + [\epsilon (\partial_x u_1)u_2]_0^L 
						  - \int_0^L \partial_xu_1 \partial_xu_2
						   + \epsilon \partial_xu_1(0)m_2\\
						& = \int_0^L b (\partial_xu_1)u_2
						  -\epsilon m_2 \partial_x u_1(0) 
						 - \int_0^L \epsilon \partial_xu_1 \partial_xu_2
						   + \epsilon \partial_xu_1(0)m_2\\
   						& = \int_0^L b (\partial_xu_1)u_2 
   						 - \int_0^L \epsilon \partial_xu_1 \partial_xu_2
\end{split}
\]

 
\begin{proposition}
	\label{prop:cas2}
	L'opérateur $A: D(A) \to \mathscr{Y}$ défini par :
	\begin{equation} 
		\label{def:A2}
	\forall (y_1,y_2) \in D(A), \; 
	\langle A y_1,y_2\rangle_{\mathscr{Y}} =\int_0^L b (\partial_xu_1)u_2
                           -  \int_0^L \epsilon \partial_xu_1 \partial_xu_2
	\end{equation}
	est générateur d'un semi-groupe $C_0$ de contraction dans $\mathscr{Y}$.
\end{proposition}						

\begin{preuve}

On souhaite appliquer le théorème de Lumer-Philipps, 
on reprend les étapes de la preuve de la propriété~\ref{prop:cas0}.

$(i)$
L'opérateur A est défini sur l'ensemble $D(A)$ des fonctions $H^1[0,L] \cap \mathscr{Y}$. 


$(ii)$
Montrons que $A$ est dissipatif.
Soit $y \in \mathscr{Y}$, 
calculons la quantité :

\[ \begin{split}
\langle Ay,y \rangle_{\mathscr{Y}} &= \int_0^L b (\partial_x u)u
                           - \epsilon \int_0^L (\partial_x u)^2\\
						           &= \int_0^L \displaystyle \frac{b}{2} \partial_x u^2
                           - \epsilon \int_0^L (\partial_x u)^2 \\
						           & = \displaystyle \frac{b}{2} [u^2]_0^L
                           - \epsilon \int_0^L (\partial_x u)^2 \\
	   						           & = - \displaystyle \frac{b}{2} u(0)^2
	                              - \int_0^L \epsilon (\partial_x u)^2
\end{split} \]

Soit :

\[
	\langle Ay,y \rangle_{\mathscr{Y}} = - \displaystyle \frac{b}{2} u(0)^2
	                              - \epsilon \int_0^L (\partial_x u)^2 
\]

Donc $\forall$ $y \in \mathscr{Y}$, $\langle Ay,y \rangle_{\mathscr{Y}} \leq 0$.


\vspace{0.3cm}
$(iii)$

Soit $y_1=(u_1,m_1) \in \mathscr{Y}$ et $\lambda >0$.

On cherche une solution $y$ tel que $(\lambda -A)y = y_1$.

Alors pour tout $y_2=(u_2,m_2)$, $y$ est solution de :
\begin{equation}
	\label{eq:surj2}
\langle \lambda y-Ay,y_2 \rangle_{\mathscr{Y}} = \langle y_1,y_2 \rangle_{\mathscr{Y}}
\end{equation}

Soit :
\[
\lambda (\int_0^L u u_2 + \displaystyle \frac{\epsilon}{\dep}mm_2)
- \int_0^L b (\partial_xu)u_2 
+  \int_0^L \epsilon \partial_xu \partial_xu_2
= \int_0^L u_1 u_2 + \displaystyle \frac{\epsilon}{\dep} m_1 m_2
\]

On choisit $u_2\in H_0^1([0,L])$, donc $m_2=0$, et on cherche $u$ solution du problème variationnel :
\[\forall u_2 \in H_0^1([0,L]), \;
\lambda  \int_0^L u u_2
- \int_0^L b (\partial_xu)u_2 
+ \int_0^L \epsilon \partial_xu \partial_xu_2 
= \int_0^L u_1 u_2 \]

On montre que 
\[ \tilde{a}(u,u_2) = \lambda  \int_0^L u u_2 
- \int_0^L b (\partial_xu)u_2 
+\int_0^L \epsilon \partial_xu \partial_xu_2 \]

 est bilinéaire, continue, coercive sur $H_0^1$. Et $L(u_2)= \int_0^L u_1 u_2$
est une forme linéaire continue sur $H_0^1$.
Donc d'après le théorème de Lax-Milgram il existe une unique fonction $u_0$ de $H_0^1(0,L)$ solution.

Alors pour $m_2 \ne 0$, il vient par ailleurs :

\[ \displaystyle \lambda \frac{\epsilon}{\dep}mm_2 = \displaystyle \frac{\epsilon}{\dep}m_1m_2 \]

Ce qui définit un unique $m$ pour $\lambda \ne 0$.

On pose alors $y = (u_0 + m_1/\lambda, m_1/\lambda)$ 
qui nous donne l'unique élément de $\mathscr{Y}$
tel que~\eqref{eq:surj2} soit vérifiée.





donc, pour tout $y_1 \in \mathscr{Y}$, cette équation possède une unique solution 

Donc d'après $(i)$,$(ii)$ et $(ii)$, on peut appliquer le théorème de Lumer-Phillips, et $A$ est le générateur infinitésimal d'un semi-group $C_0$.
On a donc l'existence d'une solution de~\eqref{eq:cas2} dans $L^2([0,+\infty)\times[0,\tau])$.

\end{preuve}

\begin{remarque}
Le caractère dissipatif peut être démontré par une égalité d'énergie.
En effet, on multiplie par $y$ la solution forte de l'équation~\eqref{eq:general} 
et on intègre entre $0$ et $L$, et il vient :


\begin{equation}
	\label{energie0}
	 \int_0^L y \partial_t y 
+ \int_0^L -\dep y \partial_x y 
+ \int_0^L -\epsilon \partial_{xx} y =0
\end{equation}

Or,

\[
\begin{split}
\displaystyle \int_0^L -\dep y \partial_x y &= \frac{-\dep}{2} \int_0^L \partial_x y^2\\
                                        &=  \frac{\dep}{2} y(0,\cdot)^2
\end{split}
\]

Et,

\[
\begin{split}
	\int_0^L -\epsilon \partial_{xx} y & = -\epsilon [y \partial_x y]_0^L 
                                    + \epsilon \int_0^L (\partial_x y)^2 \\
									& = +\epsilon y(0,\cdot) \partial_x y|_{x=0} 
									+ \epsilon \int_0^L (\partial_x y)^2 \\
									& = (\epsilon/-\dep) y(0,\cdot) \partial_t y|_{x=0} 
									+ \epsilon \int_0^L(\partial_x y)^2\\
									& = (\epsilon/2\dep) \frac{\mathrm{d}}{\mathrm{d}t} y(0,\cdot)^2
									+ \epsilon \int_0^L (\partial_x y)^2
\end{split}
\]



Alors l'équation~\eqref{energie0} devient :

\[
\displaystyle \int_0^L \frac{1}{2}\partial_t y^2 
+ \frac{\dep}{2} y(0,\cdot)^2
+ (\epsilon/2\dep) \frac{\mathrm{d}}{\mathrm{d}t} y(0,\cdot)^2
+ \epsilon \int_0^L (\partial_x y)^2 =0
\]

Donc :

\begin{equation}
	\displaystyle \frac{\mathrm{d}}{\mathrm{d}t}
	[\int_0^L y^2 
	+(\epsilon/\dep) y(0,\cdot)^2]
	= -2\epsilon\int_0^L (\partial_x y)^2 - \dep y(0,\cdot)^2
\end{equation}

On constate donc que la solution $y$ de ~\eqref{eq:cas2} est également solution 
d'une équation de dissipation d'énergie.

\end{remarque}

\begin{theoreme}[Existence et unicité]
	Soit soit $\tau>0$ et $v<0$,
	soit $\mathscr{Y}$ un espace de Banach, et le  
	de générateur infinitésimal du semi-groupe $C_0$ sur $\mathscr{Y}$
	défini par~\eqref{def:A2}.
	Alors le problème de Cauchy~\eqref{eq:cas2} a une unique solution dans $C^0([0,tau],\mathscr{Y})$ qui s'écrit :
	\[ \forall s\in[0,\tau] \; \; y(s) = S(s)y_0 \]
	où $S$ est le semi-groupe d'évolution donné par \ref{lem:cas0}.
\end{theoreme}





\newpage
%%%%%%%%%%%%%%%%%%%%%%%%%%%%%%%%%%%%%%%%%%%%%%%%%%%%%%%%%%%%%%%%%%%%%%%%%%%%%%%%%%%%%%%%%%%%%%%%%%%%%%%%%%%%%%%%%%%%%%%%%%%%%%%%%
%%%%%%%%%%%%%%%%%%%%%%%%%%%%%%%%%%%%%%%%%%%%%%%%%%%%%%%%%%%%%%%%%%%%%%%%%%%%%%%%%%%%%%%%%%%%%%%%%%%%%%%%%%%%%%%%%%%%%%%%%%%%%%%%%
%%%%%%%%%%%%%%%%%%%%%%%%%%%%%%%%%%%%%%%%%%%%%%%%%%%%%%%%%%%%%%%%%%%%%%%%%%%%%%%%%%%%%%%%%%%%%%%%%%%%%%%%%%%%%%%%%%%%%%%%%%%%%%%%%
%%%%%%%%%%%%%%%%%%%%%%%%%%%%%%%%%%%%%%%%%%%%%%%%%%%%%%%%%%%%%%%%%%%%%%%%%%%%%%%%%%%%%%%%%%%%%%%%%%%%%%%%%%%%%%%%%%%%%%%%%%%%%%%%%
%%%%%%%%%%%%%%%%%%%%%%%%%%%%%%%%%%%%%%%%%%%%%%%%%%%%%%%%%%%%%%%%%%%%%%%%%%%%%%%%%%%%%%%%%%%%%%%%%%%%%%%%%%%%%%%%%%%%%%%%%%%%%%%%%
\section{Existence Riccati et estimateur de Kalman}
%%%%%%%%%%%%%%%%%%%%%%%%%%%%%%%%%%%%%%%%%%%%%%%%%%%%%%%%%%%%%%%%%%%%%%%%%%%%%%%%%%%%%%%%%%%%%%%%%%%%%%%%%%%%%%%%%%%%%%%%%%%%%%%%%
%%%%%%%%%%%%%%%%%%%%%%%%%%%%%%%%%%%%%%%%%%%%%%%%%%%%%%%%%%%%%%%%%%%%%%%%%%%%%%%%%%%%%%%%%%%%%%%%%%%%%%%%%%%%%%%%%%%%%%%%%%%%%%%%%


%%%%%%%%%%%%%%%%%%%%%%%%%%%%%%%%%%%%%%%%%%%%%%%%%%%%%%%%%%%%%%%%%%%%%%%%%%%%%%%%%%%%%%%%%%%%%%%%%%%%%%%%%%%%%%%%%%%%%%%%%%%%%%%%%
\subsection{$v =$ cste, $\epsilon \ne 0$}
%%%%%%%%%%%%%%%%%%%%%%%%%%%%%%%%%%%%%%%%%%%%%%%%%%%%%%%%%%%%%%%%%%%%%%%%%%%%%%%%%%%%%%%%%%%%%%%%%%%%%%%%%%%%%%%%%%%%%%%%%%%%%%%%%

On cherche l'adjoint de l'opérateur $A$ sur $\mathscr{Y}$ tel que :

\begin{equation}
	\begin{cases}
		\mathscr{Y} = \left\{ y \; | \; y = (u,m) \in H^1([0,L])\times \mathbb{R},
 \; u(0)=m, \; u(L)=0 \right\} \\
        \forall y \in D(A), \; \;
		\left( \begin{array}{c}
		\dot{u}\\
		\dot{m}
		\end{array} \right)
		= A \left( \begin{array}{c}
		u\\
		m\\
		\end{array} \right) 
		= \left(\begin{array}{c}
		\dep \partial_x u + \epsilon \partial_{xx} u\\
		\dep \partial_x u(0)
		\end{array}\right)
	\end{cases}
\end{equation}

Donc pour tout $y_1$ et $y_2$ de $D(A)$ on calcule la quantité suivante :

\[ \begin{split}
\langle Ay_1,y_2 \rangle_{\mathscr{Y}} &= \int_0^L \dep (\partial_x u_1)u_2
                                     - \int_0^L \epsilon (\partial_x u_1)(\partial_x u_2)\\
					&= [\dep u_1u_2]_0^L
						- \int_0^L \dep u_1(\partial_x u_2)
						-[\epsilon u_1 (\partial_x u_2)]_0^L
						+ \int_0^L \epsilon u_1(\partial_xx u_2)\\
					&= -\dep m_1m_2
						- \int_0^L \dep u_1(\partial_x u_2)
						+\epsilon m_1 \partial_x u_2(0)
						+ \int_0^L \epsilon u_1(\partial_xx u_2)\\
					&= \int_0^L u_1 [- \dep \partial_x u_2 + \epsilon \partial_xx u_2]
						+ \displaystyle \frac{\epsilon}{\dep} m_1 
						[-\frac{\dep^2}{\epsilon} m_2 + \dep \partial_x u_2(0)]\\
					&= \langle y_1, A^* y_2 \rangle_{\mathscr{Y}}
\end{split} \]

On en déduit l'adjoint $A^*$ sur $\mathscr{Y}$ associé à la norme $\| \cdot \|_{\mathscr{Y}}$:

\begin{equation}
	\begin{cases}
		\mathscr{Y} = \left\{ y \; | \; y = (u,m) \in H^1([0,L])\times \mathbb{R},
 \; u(0)=m, \; u(L)=0 \right\} \\
        \forall y \in D(A), \; \;
	A \left( \begin{array}{c}
		u\\
		m\\
		\end{array} \right) 
		= \left(\begin{array}{c}
		- \dep \partial_x u + \epsilon \partial_xx u\\
		-\displaystyle \frac{\dep^2}{\epsilon} m + \dep \partial_x u(0)
		\end{array}\right)
	\end{cases}
\end{equation}



\newpage
%%%%%%%%%%%%%%%%%%%%%%%%%%%%%%%%%%%%%%%%%%%%%%%%%%%%%%%%%%%%%%%%%%%%%%%%%%%%%%%%%%%%%%%%%%%%%%%%%%%%%%%%%%%%%%%%%%%%%%%%%%%%%%%%%
%%%%%%%%%%%%%%%%%%%%%%%%%%%%%%%%%%%%%%%%%%%%%%%%%%%%%%%%%%%%%%%%%%%%%%%%%%%%%%%%%%%%%%%%%%%%%%%%%%%%%%%%%%%%%%%%%%%%%%%%%%%%%%%%%
%%%%%%%%%%%%%%%%%%%%%%%%%%%%%%%%%%%%%%%%%%%%%%%%%%%%%%%%%%%%%%%%%%%%%%%%%%%%%%%%%%%%%%%%%%%%%%%%%%%%%%%%%%%%%%%%%%%%%%%%%%%%%%%%%
%%%%%%%%%%%%%%%%%%%%%%%%%%%%%%%%%%%%%%%%%%%%%%%%%%%%%%%%%%%%%%%%%%%%%%%%%%%%%%%%%%%%%%%%%%%%%%%%%%%%%%%%%%%%%%%%%%%%%%%%%%%%%%%%%
%%%%%%%%%%%%%%%%%%%%%%%%%%%%%%%%%%%%%%%%%%%%%%%%%%%%%%%%%%%%%%%%%%%%%%%%%%%%%%%%%%%%%%%%%%%%%%%%%%%%%%%%%%%%%%%%%%%%%%%%%%%%%%%%%
\section{Discrétisation}
%%%%%%%%%%%%%%%%%%%%%%%%%%%%%%%%%%%%%%%%%%%%%%%%%%%%%%%%%%%%%%%%%%%%%%%%%%%%%%%%%%%%%%%%%%%%%%%%%%%%%%%%%%%%%%%%%%%%%%%%%%%%%%%%%
%%%%%%%%%%%%%%%%%%%%%%%%%%%%%%%%%%%%%%%%%%%%%%%%%%%%%%%%%%%%%%%%%%%%%%%%%%%%%%%%%%%%%%%%%%%%%%%%%%%%%%%%%%%%%%%%%%%%%%%%%%%%%%%%%





\newpage
%%%%%%%%%%%%%%%%%%%%%%%%%%%%%%%%%%%%%%%%%%%%%%%%%%%%%%%%%%%%%%%%%%%%%%%%%%%%%%%%%%%%%%%%%%%%%%%%%%%%%%%%%%%%%%%%%%%%%%%%%%%%%%%%%
%%%%%%%%%%%%%%%%%%%%%%%%%%%%%%%%%%%%%%%%%%%%%%%%%%%%%%%%%%%%%%%%%%%%%%%%%%%%%%%%%%%%%%%%%%%%%%%%%%%%%%%%%%%%%%%%%%%%%%%%%%%%%%%%%
%%%%%%%%%%%%%%%%%%%%%%%%%%%%%%%%%%%%%%%%%%%%%%%%%%%%%%%%%%%%%%%%%%%%%%%%%%%%%%%%%%%%%%%%%%%%%%%%%%%%%%%%%%%%%%%%%%%%%%%%%%%%%%%%%
%%%%%%%%%%%%%%%%%%%%%%%%%%%%%%%%%%%%%%%%%%%%%%%%%%%%%%%%%%%%%%%%%%%%%%%%%%%%%%%%%%%%%%%%%%%%%%%%%%%%%%%%%%%%%%%%%%%%%%%%%%%%%%%%%
%%%%%%%%%%%%%%%%%%%%%%%%%%%%%%%%%%%%%%%%%%%%%%%%%%%%%%%%%%%%%%%%%%%%%%%%%%%%%%%%%%%%%%%%%%%%%%%%%%%%%%%%%%%%%%%%%%%%%%%%%%%%%%%%%
\section{Analyse numérique}
%%%%%%%%%%%%%%%%%%%%%%%%%%%%%%%%%%%%%%%%%%%%%%%%%%%%%%%%%%%%%%%%%%%%%%%%%%%%%%%%%%%%%%%%%%%%%%%%%%%%%%%%%%%%%%%%%%%%%%%%%%%%%%%%%
%%%%%%%%%%%%%%%%%%%%%%%%%%%%%%%%%%%%%%%%%%%%%%%%%%%%%%%%%%%%%%%%%%%%%%%%%%%%%%%%%%%%%%%%%%%%%%%%%%%%%%%%%%%%%%%%%%%%%%%%%%%%%%%%%

\newpage
%%%%%%%%%%%%%%%%%%%%%%%%%%%%%%%%%%%%%%%%%%%%%%%%%%%%%%%%%%%%%%%%%%%%%%%%%%% REFERENCES

\medskip

\bibliographystyle{unsrt}%Used BibTeX style is unsrt
\bibliography{20190206biblio}
	
\end{document}

\end{document}


%% PRE EDITION
\documentclass[a4paper]{article}
\usepackage[utf8]{inputenc}
\usepackage[T1]{fontenc}
\usepackage[french]{babel}
\usepackage{soul}
\usepackage[pdftex]{graphicx}

\usepackage{amsfonts}
\usepackage{amsthm}
\usepackage{amsmath}
\usepackage{amssymb}
\usepackage{mathrsfs}
\usepackage{booktabs}
\usepackage{siunitx}
\usepackage{thmtools}
\usepackage{ulem}

%% LAYOUT
\declaretheoremstyle[
    bodyfont=\normalfont\color{red},
    headfont=\color{red}
]{styleattention}

\declaretheoremstyle[
    spacebelow=1em
]{styleremarque}

\declaretheoremstyle[
    spaceabove=-6pt, 
    spacebelow=6pt, 
    headfont=\normalfont\bfseries, 
    bodyfont = \normalfont,
    postheadspace=1em, 
    qed=$\Box$, 
    headpunct={$\rhd$}
]{mystyle} 

\declaretheorem[thmbox=M,numberwithin=section,title=Définition]{definition}
\declaretheorem[thmbox=M,sibling=definition]{proposition}
\declaretheorem[thmbox=M,sibling=definition]{corollaire}
\declaretheorem[thmbox=M,sibling=definition,title=Théorème]{theoreme}
\declaretheorem[thmbox=M,sibling=definition]{lemme}
\declaretheorem[thmbox=M,sibling=definition,title=Propriété]{propriete}
\declaretheorem[thmbox=M,sibling=definition,title=Propriétés]{proprietes}
\declaretheorem[style=styleremarque,sibling=definition,title=Remarque]{remarque}
\declaretheorem[style=styleattention,title=À revoir]{Arevoir}
\declaretheorem[name={}, style=mystyle, unnumbered]{preuve}


\renewcommand\qedsymbol{$\blacksquare$}

%% NEW COMMAND
% solution
\newcommand{\y}{y}
% concentration monomers
\newcommand{\cmono}{c}
\newcommand{\pol}{a}
\newcommand{\dep}{b}
\newcommand{\mass}{\mathrm{M}}

\usepackage{geometry}
\geometry{hmargin=3cm,vmargin=2.5cm}

\usepackage{tabularx}
\usepackage{float}

\title{Réunion du 01/02}
\author{Cécile Della Valle}

%% DEBUT DE REDACTION
\begin{document}

\maketitle

%%%%%%%%%%%%%%%%%%%%%%%%%%%%%%%%%%%%%%%%%%%%%%%%%%%%%%%%%%%%%%%%%%%%%%%%%%%%%%%%%%%%%%%%%%%%%%%%%%%%%%%%%%%%%%%%%%%%%%%%%%%%%%%%%
%%%%%%%%%%%%%%%%%%%%%%%%%%%%%%%%%%%%%%%%%%%%%%%%%%%%%%%%%%%%%%%%%%%%%%%%%%%%%%%%%%%%%%%%%%%%%%%%%%%%%%%%%%%%%%%%%%%%%%%%%%%%%%%%%
%%%%%%%%%%%%%%%%%%%%%%%%%%%%%%%%%%%%%%%%%%%%%%%%%%%%%%%%%%%%%%%%%%%%%%%%%%%%%%%%%%%%%%%%%%%%%%%%%%%%%%%%%%%%%%%%%%%%%%%%%%%%%%%%%
%%%%%%%%%%%%%%%%%%%%%%%%%%%%%%%%%%%%%%%%%%%%%%%%%%%%%%%%%%%%%%%%%%%%%%%%%%%%%%%%%%%%%%%%%%%%%%%%%%%%%%%%%%%%%%%%%%%%%%%%%%%%%%%%%
%%%%%%%%%%%%%%%%%%%%%%%%%%%%%%%%%%%%%%%%%%%%%%%%%%%%%%%%%%%%%%%%%%%%%%%%%%%%%%%%%%%%%%%%%%%%%%%%%%%%%%%%%%%%%%%%%%%%%%%%%%%%%%%%%
\section{Existence}
%%%%%%%%%%%%%%%%%%%%%%%%%%%%%%%%%%%%%%%%%%%%%%%%%%%%%%%%%%%%%%%%%%%%%%%%%%%%%%%%%%%%%%%%%%%%%%%%%%%%%%%%%%%%%%%%%%%%%%%%%%%%%%%%%
%%%%%%%%%%%%%%%%%%%%%%%%%%%%%%%%%%%%%%%%%%%%%%%%%%%%%%%%%%%%%%%%%%%%%%%%%%%%%%%%%%%%%%%%%%%%%%%%%%%%%%%%%%%%%%%%%%%%%%%%%%%%%%%%%

Soit $L>0$ et $\tau>0$,
soit $v$ une fonction continue, $v \in C^0([0,\tau])$, 
On souhaite démontrer l'existence d'une solution de :

\begin{equation}
\label{eq:general}
\begin{cases}
 \displaystyle \frac{\partial y}{\partial t}
 + v(t) \frac{\partial y} {\partial x}  
 - \epsilon \frac{\partial^2 y} {\partial x^2}
 = 0  & \forall (x,t) \in [0,L] \times [0, \tau]\\
 y(x,0) = y_{0} (x) & \forall x \in [0,L] \\
 1_{v(t)>0}y(0,t) = 0 & \forall t \in [0,\tau] \\
 1_{v(t)<0}y(L,t) = 0 & \forall t \in [0,\tau] \\
\end{cases}
\end{equation}


%%%%%%%%%%%%%%%%%%%%%%%%%%%%%%%%%%%%%%%%%%%%%%%%%%%%%%%%%%%%%%%%%%%%%%%%%%%%%%%%%%%%%%%%%%%%%%%%%%%%%%%%%%%%%%%%%%%%%%%%%%%%%%%%%
\subsection{$v=$cste, $\epsilon = 0$}
%%%%%%%%%%%%%%%%%%%%%%%%%%%%%%%%%%%%%%%%%%%%%%%%%%%%%%%%%%%%%%%%%%%%%%%%%%%%%%%%%%%%%%%%%%%%%%%%%%%%%%%%%%%%%%%%%%%%%%%%%%%%%%%%%



On suppose dans un premier temps que $v$ est une constante, et on suppose,
sans perte de généralité, 
que cette constante est positive $v>0$.
De plus on suppose que la constante $\epsilon$ est nulle.

L'équation~\eqref{eq:general} devient :

\begin{equation}
\label{eq:cas0}
\begin{cases}
 \displaystyle \frac{\partial y}{\partial t}
 + v \frac{\partial y} {\partial x}  
 = 0  & \forall (x,t) \in [0,L] \times [0, \tau]\\
 y(x,0) = y_{0} (x) & \forall x \in [0,L] \\
 y(0,t) = 0 & \forall t \in [0,\tau] \\
\end{cases}
\end{equation}

Soit $\mathscr{Y} = L^2([0,L])$ l'espace de Banach et l'opérateur $A$ sur $D(A)$ tel que :
\[ \forall y \in D(A), \; \; Ay = -v \partial_x y \]

On souhaite démontrer la propriété suivante :

\begin{proposition}
	\label{prop:cas0}
	L'opérateur $A: D(A) \to \mathscr{Y}$ est générateur d'un semi-groupe $C_0$ de contraction dans $\mathscr{Y}$.
\end{proposition}

\begin{preuve}
	On souhaite appliquer le théorème de Lumer-Phillips, 
	il nous faut donc démontrer que $A$ possède les trois propriétés suivantes :
\begin{itemize}
	\item (1) $A$ est un opérateur fermé et $\bar{D{A}} = \mathscr{Y}$ ;
	\item (2) $A$ est dissipatif ;
	\item (3) il existe $\lambda_0$ tel que $\lambda_0 - A : D(A) \to \mathscr{Y}$ est surjectif.
\end{itemize}

\vspace{0.3cm}
(1)

L'opérateur A est défini sur l'ensemble $D(A)$ des fonctions absoluement continues sur $[0,L]$ qui s'annulent en $0$. 
On peut par exemple utilisee lemme suivant :

\begin{lemme}
	Une fonction $y$ de $\mathscr{Y}$ est absolument continue sur $[0,L]$ si et seulement si pour presque tout $x \in [0,L]$, 
	$y$ est dérivable en $x$ de dérivée $y' \in L^2([0,L])$ et 
	\[ y(x) = y(0) + \int_0^x y' \]
\end{lemme}

Sur cet ensemble, l'opérateur dérivation est fermée, densément défini.

\vspace{0.3cm}
(2)

Montrons que $A$ est dissipatif.
Soit $y \in \mathscr{Y}$, 
calculons la quantité :

\[ \begin{split}
\langle Ay,y \rangle_{\mathscr{Y}} &= \langle v \partial_x y, y \rangle_{\mathscr{Y}}  \\
                     &= v \int_0^L y(z) \partial_x y(z) d(z) \\
					 & = v[y(z)^2]_{z=0}^L - v \int_0^L y(z) \partial_x y(z) d(z)
\end{split} \]

Soit :

\[ \langle Ay,y \rangle_{\mathscr{Y}} = -\frac{1}{2}vy(0)^2 \leq 0 \]

Donc $\forall$ $y \in \mathscr{Y}$, $\langle Ay,y \rangle_{\mathscr{Y}} \leq 0$.


\vspace{0.3cm}
(3)

Soit $y_1 \in \mathscr{Y}$ et $\lambda >0$.

On cherche une solution $y$ tel que $(\lambda -A)y = y_1$.

Alors on a de façon équivalente que $y$ est solution de :
\[ \begin{cases}
\lambda y - v y' = y_1 \\
y(0) = 0
\end{cases}\]

donc, pour tout $y_1 \in \mathscr{Y}$, cette équation possède une unique solution donnée par la formule de Duhamel :

\[ y(x) = - \int_0^x e^{\lambda/v (x-x')}y_1(x')dx' \]

(On peut par ailleurs vérifier que $ \| y \|_{\mathscr{Y}} \leq 1/ \lambda \|y_1 \|_{\mathscr{Y}}$).

Donc d'après (1),(2) et (3), on peut appliquer le théorème de Lumer-Phillips, et $A$ est le générateur infinitésimal d'un semi-group $C_0$.
On a donc l'existence d'une solution de~\eqref{eq:cas0}.
	
\end{preuve}


%%%%%%%%%%%%%%%%%%%%%%%%%%%%%%%%%%%%%%%%%%%%%%%%%%%%%%%%%%%%%%%%%%%%%%%%%%%%%%%%%%%%%%%%%%%%%%%%%%%%%%%%%%%%%%%%%%%%%%%%%%%%%%%%%
\subsection{$v \in C^0$, $\epsilon = 0$}
%%%%%%%%%%%%%%%%%%%%%%%%%%%%%%%%%%%%%%%%%%%%%%%%%%%%%%%%%%%%%%%%%%%%%%%%%%%%%%%%%%%%%%%%%%%%%%%%%%%%%%%%%%%%%%%%%%%%%%%%%%%%%%%%%


On suppose cette fois que $v \in C^0$ et de plus que $v$ ne change pas de signe. 
Sans perdre de généralité on suppose que $\forall t \in [0,\tau]$ on a $v(t)>0$

\begin{equation}
\label{eq:cas1}
\begin{cases}
 \displaystyle \frac{\partial y}{\partial t}
 + v(t) \frac{\partial y} {\partial x}  
 = 0  & \forall (x,t) \in [0,L] \times [0, \tau]\\
 y(x,0) = y_{0} (x) & \forall x \in [0,L] \\
 y(0,t) = 0 & \forall t \in [0,\tau] \\
\end{cases}
\end{equation}

Soit $\mathscr{Y} = L^2([0,L])$ l'espace de Banach et l'opérateur d'évolution $A(t)$ définit sur $D(A(t))$ tel que :
\[ \forall y \in D(A(t)), \; \; A(t)y = -v (t)\partial_x y \]

\begin{proposition}
	Pour $t>0$, l'opérateur $A(t)$ est le générateur infinitésimal du semi-groupe $C_0$ sur $\mathscr{Y}$,
	noté $S_t$.
\end{proposition}

\begin{preuve}
	Pour $t>0$ fixé, on est ramené au cas de la proposition~\ref{prop:cas0}.
\end{preuve}

\begin{proposition}
	Il existe une unique solution $U(t,s)$, $0\leq s \leq t \leq \tau $, qui satisfait :
	\begin{itemize}
		\item 
		\item 
		\item 
	\end{itemize}
\end{proposition}

\begin{preuve}
	On souhaite appliquer le théorème du chap.5 de Pazy.
	
	\begin{lemme}
		Pour tout $t>0$, le semi-groupe $C_0$ généré par $A(t)$ est la translation :
		\[ \forall y \in D(A(t)), \; \; S_t(s)y = y(x - \theta(s)) \chi_{0<x - \theta(s)<L} \]
		avec $\theta (s) = v(t)s$.
		
	\end{lemme}
	
	Vérifions les hypothèses.
	
	\vspace{0.3cm}
	(1)
	Montrons que $(A(t))_{t \in [0,\tau]}$ est stable.
	
	D'après le théorème de Hille-Yoshida, la demi droite $(0; +\infty)$ est incluse dans l'ensemble résolvent de $A(t)$ pour tout $t>0$.
	
\end{preuve}

\end{document}


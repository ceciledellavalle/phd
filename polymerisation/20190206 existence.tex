%% PRE EDITION
\documentclass[a4paper]{article}
\usepackage[utf8]{inputenc}
\usepackage[T1]{fontenc}
\usepackage[french]{babel}
\usepackage{soul}
\usepackage[pdftex]{graphicx}

\usepackage{amsfonts}
\usepackage{amsthm}
\usepackage{amsmath}
\usepackage{amssymb}
\usepackage{mathrsfs}
\usepackage{booktabs}
\usepackage{siunitx}
\usepackage{thmtools}
\usepackage{ulem}

%% LAYOUT
\declaretheoremstyle[
    bodyfont=\normalfont\color{red},
    headfont=\color{red}
]{styleattention}

\declaretheoremstyle[
    spacebelow=1em
]{styleremarque}

\declaretheoremstyle[
    spaceabove=-6pt, 
    spacebelow=6pt, 
    headfont=\normalfont\bfseries, 
    bodyfont = \normalfont,
    postheadspace=1em, 
    qed=$\Box$, 
    headpunct={$\rhd$}
]{mystyle} 

\declaretheorem[thmbox=M,numberwithin=section,title=Définition]{definition}
\declaretheorem[thmbox=M,sibling=definition]{proposition}
\declaretheorem[thmbox=M,sibling=definition]{corollaire}
\declaretheorem[thmbox=M,sibling=definition,title=Théorème]{theoreme}
\declaretheorem[thmbox=M,sibling=definition]{lemme}
\declaretheorem[thmbox=M,sibling=definition,title=Propriété]{propriete}
\declaretheorem[thmbox=M,sibling=definition,title=Propriétés]{proprietes}
\declaretheorem[style=styleremarque,sibling=definition,title=Remarque]{remarque}
\declaretheorem[style=styleattention,title=À revoir]{Arevoir}
\declaretheorem[name={}, style=mystyle, unnumbered]{preuve}


\renewcommand\qedsymbol{$\blacksquare$}

%% NEW COMMAND
% solution
\newcommand{\y}{y}
% concentration monomers
\newcommand{\cmono}{c}
\newcommand{\pol}{a}
\newcommand{\dep}{b}
\newcommand{\mass}{\mathrm{M}}

\usepackage{geometry}
\geometry{hmargin=3cm,vmargin=2.5cm}

\usepackage{tabularx}
\usepackage{float}

\title{Réunion du 01/02}
\author{Cécile Della Valle}

%% DEBUT DE REDACTION
\begin{document}

\maketitle

%%%%%%%%%%%%%%%%%%%%%%%%%%%%%%%%%%%%%%%%%%%%%%%%%%%%%%%%%%%%%%%%%%%%%%%%%%%%%%%%%%%%%%%%%%%%%%%%%%%%%%%%%%%%%%%%%%%%%%%%%%%%%%%%%
%%%%%%%%%%%%%%%%%%%%%%%%%%%%%%%%%%%%%%%%%%%%%%%%%%%%%%%%%%%%%%%%%%%%%%%%%%%%%%%%%%%%%%%%%%%%%%%%%%%%%%%%%%%%%%%%%%%%%%%%%%%%%%%%%
%%%%%%%%%%%%%%%%%%%%%%%%%%%%%%%%%%%%%%%%%%%%%%%%%%%%%%%%%%%%%%%%%%%%%%%%%%%%%%%%%%%%%%%%%%%%%%%%%%%%%%%%%%%%%%%%%%%%%%%%%%%%%%%%%
%%%%%%%%%%%%%%%%%%%%%%%%%%%%%%%%%%%%%%%%%%%%%%%%%%%%%%%%%%%%%%%%%%%%%%%%%%%%%%%%%%%%%%%%%%%%%%%%%%%%%%%%%%%%%%%%%%%%%%%%%%%%%%%%%
%%%%%%%%%%%%%%%%%%%%%%%%%%%%%%%%%%%%%%%%%%%%%%%%%%%%%%%%%%%%%%%%%%%%%%%%%%%%%%%%%%%%%%%%%%%%%%%%%%%%%%%%%%%%%%%%%%%%%%%%%%%%%%%%%

Soit $L>0$ et $\tau>0$,
soit $v$ une fonction continue, $v \in C^0([0,\tau])$, 
On souhaite démontrer l'existence d'une solution de :

\begin{equation}
\label{eq:general}
\begin{cases}
 \displaystyle \frac{\partial y}{\partial t}
 + v(t) \frac{\partial y} {\partial x}  
 - \epsilon \frac{\partial^2 y} {\partial x^2}
 = 0  & \forall (x,t) \in [0,L] \times [0, \tau]\\
 y(x,0) = y_{0} (x) & \forall x \in [0,L] \\
 1_{v(t)>0}y(0,t) = 0 & \forall t \in [0,\tau] \\
 1_{v(t)<0}y(L,t) = 0 & \forall t \in [0,\tau] \\
\end{cases}
\end{equation}


\vspace{0.5cm}

\underline{$v=$cste, $\epsilon = 0$}

On suppose dans un premier temps que $v$ est une constante, et on suppose,
sans perte de généralité, 
que cette constante est positive $v>0$.
De plus on suppose que la constante $\epsilon$ est nulle.

L'équation~\eqref{eq:general} devient :

\begin{equation}
\label{eq:cas0}
\begin{cases}
 \displaystyle \frac{\partial y}{\partial t}
 + v \frac{\partial y} {\partial x}  
 = 0  & \forall (x,t) \in [0,L] \times [0, \tau]\\
 y(x,0) = y_{0} (x) & \forall x \in [0,L] \\
 y(0,t) = 0 & \forall t \in [0,\tau] \\
\end{cases}
\end{equation}

Soit $\mathscr{Y} = L^2([0,L])$ l'espace de Banach et l'opérateur $A$ sur cet espace tel que :
\[ \forall y \in D(A), \; \; Ay = -v \partial_x y \]

On souhaite démontrer la propriété suivante :

\begin{proposition}
	L'opérateur $A: D(A) \to \mathscr{Y}$ est générateur d'un semi-groupe $C_0$ de contraction dans $\mathscr{Y}$.
\end{proposition}

\begin{preuve}
	On souhaite appliquer le théorème de Hille-Yoshida, 
	il nous faut donc démontrer que $A$ possède les deux propriétés suivantes :
\begin{itemize}
	\item (1) $A$ est un opérateur fermé et $\bar{D{A}} = \mathscr{Y}$ ;
	\item (2) l'ensemble résolvant de A contient la demi-droite $]0; + \infty)$ et on a 
	$\forall \lambda > 0 \; \| (\lambda - A)^{-1} \|_{L({\mathscr{Y}})} \leq 1/\lambda $
\end{itemize}

(1)

L'opérateur A est défini sur l'ensemble $D(A)$ des fonctions absoluement continues sur $[0,L]$ qui s'annulent en $0$. 
On peut par exemple utilisee lemme suivant :

\begin{lemme}
	Une fonction $y$ de $\mathscr{Y}$ est absolument continue sur $[0,L]$ si et seulement si pour presque tout $x \in [0,L]$, 
	$y$ est dérivable en $x$ de dérivée $y' \in L^2([0,L])$ et 
	\[ y(x) = y(0) + \int_0^x y' \]
\end{lemme}

Sur cet ensemble, l'opérateur dérivation est fermée, densément défini.

(2)

Soit $y_1 \in \mathscr{Y}$ tel aue $(\lambda -A)y = y_1$, alors on a de façon équivalente :
\[ \begin{cases}
\lambda y - v y' = y_1 \\
y(0) = 0
\end{cases}\]

Pour tout $\lambda \in \mathbb{R}$, cette équation possède une unique solution donnée par la formule de Duhamel :
\[ y(x) = - \int_0^x e^{\lambda/v (x-x')}y_1(x')dx' \]

et on vérifie que $ \| y \|_{\mathscr{Y}} \leq 1/ \lambda \|y_1 \|_{\mathscr{Y}}$.

\[
\begin{split}
	\|y\|_{\mathscr{Y}}^2 &= \int_0^L |\int_0^x e^{\lambda/v (x-x')}y_1(x')dx' |^2 dx \\
                        & \leq (\int_0^L e^{2 \lambda/v x} dx) (\int_0^L |\int_0^x e^{-2\lambda/v x'} y_1(x') dx' |^2 dx \\
						& \leq \frac{v}{2 \lambda}(e^{\frac{2L\lambda}{v}}-1) (\int_0^L e^{-2\lambda/v x'}dx' \int_0^L |y_1(x')|^2 dx') \\
						& \leq \frac{v}{2 L \lambda}^2(e^{\frac{2L\lambda}{v}}-1)^2 \|y_1 \|_{\mathscr{Y}} \\
						& \leq \frac{1}{\lambda} \|y_1 \|_{\mathscr{Y}} 
\end{split}
\]
	
\end{preuve}

On obtient donc l'existence de solution de~\eqref{eq:cas0}.

\underline{$v \in C^0$, $\epsilon = 0$}

On suppose cette fois que $v \in C^0$ et de plus que $v$ ne change pas de signe. 
Sans perdre de généralité on suppose que $\forall t \in [0,\tau]$ on a $v(t)>0$

\begin{equation}
\label{eq:cas0}
\begin{cases}
 \displaystyle \frac{\partial y}{\partial t}
 + v(t) \frac{\partial y} {\partial x}  
 = 0  & \forall (x,t) \in [0,L] \times [0, \tau]\\
 y(x,0) = y_{0} (x) & \forall x \in [0,L] \\
 y(0,t) = 0 & \forall t \in [0,\tau] \\
\end{cases}
\end{equation}

\end{document}


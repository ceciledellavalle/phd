%% PRE EDITION
\documentclass[a4paper]{article}
\usepackage[utf8]{inputenc}
\usepackage[T1]{fontenc}
\usepackage[french]{babel}
\usepackage{soul}
\usepackage[pdftex]{graphicx}

\usepackage{amsfonts}
\usepackage{amsthm}
\usepackage{amsmath}
\usepackage{amssymb}
\usepackage{mathrsfs}
\usepackage{booktabs}
\usepackage{siunitx}

%% LAYOUT
\newtheorem{theorem}{Theorem}[section]
\newtheorem{corollary}{Corollary}[theorem]
\newtheorem{lemma}[theorem]{Lemma}
\newtheorem{prop}{Proposition}[section]
\theoremstyle{definition}
\newtheorem{definition}{Definition}[section]
\theoremstyle{remark}
\newtheorem*{remark}{Remark}
\renewcommand\qedsymbol{$\blacksquare$}

%% NEW COMMAND
\newcommand{\mass}{\mathrm{M}}
\newcommand{\pol}{p}
\newcommand{\dep}{d}

\usepackage{geometry}
\geometry{hmargin=3cm,vmargin=2.5cm}

\usepackage{tabularx}
\usepackage{float}

\title{Premier résultat sur la polymérisation}
\author{Cécile Della Valle}

%% DEBUT DE REDACTION
\begin{document}

\maketitle

%%%%%%%%%%%%%%%%%%%%%%%%%%%%%
\section{Introduction}

Afin d'élargir les résultats sur le cas linéaire de la dépolimérisation uniquement, 
nous remplaçons le coefficient de réaction constant par une vitesse de réaction totale. 
Le problème reste linéaire tout en introduisant de la polymérisation. Expérimentalement, 
cette vitesse totale moyenne résulte en effet de la somme des vitesses de polymérisation et de dépolymérisation, 
elle est connue par la mesure du moment d'ordre 1. 

Dans une première partie nous rappelons des résultats sur le problème direct de Lifschitz-Slyosov, 
sur l'espace fonctionnel des solutions et sur leur comportement asymptotique. 
L'objectif est en autre de déterminer sous quelles conditions la vitesse totale de réaction converge.

Nous rappelons dans une deuxième partie les résultats obtenus sur l'observabilité pour la seule dépolymérisation. 
Nous chercherons à étendre ce résultat  au cas où le coefficient de dépolymérisation n'est pas constant en fonction de x.

Dans une troisième partie, nous introduirons la polymérisation. 
La vitesse de réaction totale ne dépent alors que du temps. 
Nous considérons deux cas : le cas où la vitesse de polymérisation est positive et le cas où elle est négative.

Cette étude nous incite à considérer des cas où le coefficent de dépolymérisation est à croissance bornée.

%%%%%%%%%%%%%%%%%%%%%%%%%%%%%%

\section{Notation}

\begin{itemize}
\item $y$ -- fonction de la concentration des polymères en fonction de leur taille x ;
\item $c$ -- fonction de concentration du monomère
\item $v(t)$ -- vitesse totale de réaction (dépolymérisation et polymérisation) ;
\item $\bar{v}(t)$ -- vitesse totale de réaction moyenne quand les coefficients de réaction dépendent de la taille x ;
\item $\pol$ -- coefficient de polymérisation ;
\item $\dep$ -- coefficient de dépolymérisation ;
\item $\mu_i$ -- moment d'ordre i de la fonction y.
\end{itemize}

%%%%%%%%%%%%%%%%%%%%%%%%%%%%%%

\section{Objectif}

On se propose d'étudier le système linéaire issu du modèle de Lifshitz-Slyosov où les vitesses de réaction totale, somme de la vitesse de polymérisation et dépolimérisation, est connue et ne dépend que du temps. En effet, cette vitesse se déduit de la mesure du moment d'ordre 1, noté $\mu_1$, des polymères :
\[
\mu_1 = \int_0 ^\infty x y(x,t) dx
\]
Ainsi le modèle de Lifshitz-Slyosov s'écrit pour $y(x,t)$ la concentration de polymères de taille $x$ en temps $t$ :

\begin{equation}
\label{eq:general}
\begin{cases}
 \displaystyle \frac{\partial y}{\partial t}+ v(t) \frac{\partial y} {\partial x}  = 0  \\
 y(x,0) = y_{0} (x) 
\end{cases}
\end{equation}

où la vitesse $v(t)$ est calculée par conservation de la masse totale $\rho$, des coefficients de polymérisation $k_{pol}$ et $k_{dep}$:

\[
v(t) = \pol(\rho - \int_0 ^\infty x y(x,t) dx)-\dep
\]

\underline{Questions}

\underline{Asymptotes et vitesse de réaction} On constate que si $v(t)<0$, $\forall t > t^*$, 
alors on retrouve le cas de la dépolymérisation déjà étudié par un changement de variable pour $t > t^*$. 
On peut donc se demander si un tel régime existe.

\underline{Conditions aux limites} Dans le cadre de la dépolymérisation on a pu imposer une taille $l$ limite, 
au-dessus de laquelle la concentration en polymère est nulle. 
En effet, pour un système physique de polymères totalement polymérisés, 
la taille des polymères dans l'équation discrète de Becker-Döring ne saurait dépasser la taille d'un polymère 
qui aurait aggloméré l'ensemble des monomères. O
r cette limite physique n'est plus vraie pour Lifshitz-Slyozov, 
par exemple avec un coefficient $\dep$ strictement décroissant en fonction de la taille. 
Quelles sont les conséquences pour la solution puis pour l'observabilité de se restreindre à un domaine borné 
$[0, \tau] \times [0,l]$ ?

\underline{Espaces fonctionnels} Les solutions dde l'équation de Lifschitz-Slyosov sont en général contenu dans $L^1$, o
r la théorie de l'assimilation de données s'applique sur $L^2$. 
Quelles conditions supplémentaires doit-on imposer pour assurer l'existence de solution dans $L^2$ ? 
Par exemple si un régime asymptotique existe pour lequel $v(t)<0$ ou $v(t)>0$, $\forall t > t^*$, 
alors par changement de variable on peut se ramener à une équation de transport 
pour laquelle la solution dams $L^2$ est assurée par l'espace fonctionnel de la condition initiale.


%%%%%%%%%%%%%%%%%%%%%%%%%%%%%
%%%%%%%%%%%%%%%%%%%%%%%%%%%%%
%%%%%%%%%%%%%%%%%%%%%%%%%%%%%

\section{Problème direct - Lifschitz-Slyozov}

%%%%%%%%%%%%%%%%%%%%%%%%%%%%%

On considère l'étude d'un cas de polymérisation et dépolymérisation où ces coefficients de réaction dépendent de la taille du polymère $x$.
Nous nous intéressons au problème direct suivant pour un système de Lifchitz-Slyozov pour un coefficient de polymérisation normalisé à $1$ et pour un coefficient de dépolymérisation $\dep \in C^1$. La nucléation est ici considérée comme nulle et il n'existe pas de taille de polymères maximums.

Nous rappelons ici les résultats de \cite{CalvoPerthame} pour les équations suivantes :

\begin{equation}
	\label{eq:lsc}
\begin{cases}
	\displaystyle \frac{\mathrm{d}c}{\mathrm{d}t} = -c(t) \int_0^\infty y(x,t)dx + \int_0^\infty \dep(x) y(x,t)dx \\
	(c(t)-d(0))y(0,t)\mathbb{I}_{c(t)-\dep(0) > 0} = 0 \\
	\displaystyle \frac{\partial y}{\partial t}+ \frac{\partial}{\partial x}((c(t)-\dep(x))y) = 0 \\
	y(0,x) = y_0(x)
\end{cases}
\end{equation}





%%%%%%%%%%%%%%%%%%%%%%%%%%%%%
\subsection{Asymptotes}
%%%%%%%%%%%%%%%%%%%%%%%%%%%%%

\subsubsection{$\dep$ décroissant strictement en fonction de la taille $x$}
%%%%%%%%%%%%%%%%%%%%%%%%%%%%%

Dans le cas où l'on considère que $\dep$ décroit strictement, il a déjà été prouvé \cite{Niethammer} que les asymptotes de l'équation sont réduites à deux cas :
\begin{itemize}
\item $\mass<\dep(0)$ :  la dépolymérisation est totale et $c$ tend vers la masse totale du système $\mass$.
\item $\dep(0) \leq \mass$ : le moment d'ordre zéro noté $\mu_0$ tend vers 0 et la taille moyenne tend vers $\infty$.
\end{itemize}

Mais ce résultat n'est plus valable pour $\dep = cste$.

\subsubsection{Asymptotes pour $\dep$ à croissance bornée}
%%%%%%%%%%%%%%%%%%%%%%%%%%%%%

Par la suite on suppose qu'il existe $\alpha$ et $\beta$ tels que $ \forall x \in \mathbb{R}_+$, 

$0 < \alpha \leq \dep' (x) \leq \beta$. 

Cette hypothèse est justifiée par le fait que l'hypothèse de décroissance stricte de $\dep$ ne reflète pas les comportements asymptotiques attendus pour des polymères de type fibrille amyloïde (notamment la convergence vers $+\infty$ de la taille du fibrille moyen).

\vspace{0.5cm}

\underline{Cas 1 :} $c_0<\dep(0)$

\begin{prop} 
Soit $(c,y) \in C_b^1(\mathbb{R}_+)\times C(\mathbb{R}_+,L^1(1+x^2)dx)$  une fonction positive solution de l'équation ~\eqref{eq:lsc} telle que la condition initiale sur la concentration de monomères vérifie $dep(0)>c_0$, alors :
\begin{itemize}
	\item Si $\dep(0) < \mass$ alors $\exists t^* \in (0,\infty)$ tel que $c(t^*)= \dep(0)$ et $\frac{\mathrm{d}c}{\mathrm{d}t} (t^*)>0$
	\item Si $\mass \leq \dep(0) $ alors $\|\mass - c(t)\| \leq (\mass-c_0)e^{-\alpha t}$
\end{itemize}
\end{prop}

La preuve n'est plus valable dans le cas où $\dep$ est une constante, on ne peut plus minorer le terme $\dep$ en fonction de $\alpha$ (et on voit d'ailleurs aisément dans le second cas que que si $\alpha =0 $ on perd la convergence, la dépolymérisation est totale et l'inégalité devient triviale). Néanmoins ces inégalités sont intéressantes si on veut étudier le cas où le coefficient de dépolymérisation n'est pas constant.

\vspace{0.5cm}

\underline{Cas 2:} $\dep(0) \leq c_0 $

La dépolymérisation n'est plus majoritaire en $t=0$ et $x=0$ et il faut considérer la nucléation.

On définit la carctéristique :

\[
\begin{cases}
\displaystyle \frac{\mathrm{d}}{\mathrm{d}s} X(s;x,t)= c(s) - \dep(X(s;x,t))\\
X(t;x,t) = x
\end{cases}
\]

\begin{prop} 
Soit $(c,y) \in C_b^1(\mathbb{R}_+)\times C(\mathbb{R}_+,L^1(1+x^2)dx)$  une fonction positive solution de l'équation ~\eqref{eq:lsc} telle que la condition initiale sur la concentration de monomères vérifie $dep(0) \leq c_0$, et il existe une unique solution $\bar{x}>0$ tel que $\mass = \mu_0 \bar{s} +\dep{\bar{x}}$.
Alors :
\begin{itemize}
\item Quelque soit $s \geq 0$, $\int_0^\infty \|X(s,t) -x\|^2 y(x,t)\mathrm{d}x \leq e^{-2\alpha t}\int_0^\infty \|s -x\|^2y_0(x)\mathrm{d}x$
\item $lim_{t \rightarrow \infty}X(s,t) = \bar{x}$
\item $lim_{t \rightarrow \infty}c(t) = \bar{c} = \dep{\bar{x}}$
\end{itemize}

Enfin, la solution $y$ converge exponentiellement vers la fonction Dirac centrée sur $\bar{x}$ de hauteur $\mu_0$ au sens de la distance de Wasserstein.
\end{prop}





%%%%%%%%%%%%%%%%%%%%%%%%%%%%%
\subsection{Espace fonctionnel}
%%%%%%%%%%%%%%%%%%%%%%%%%%%%%

On remarque dans les résultats précédents l'espace des solutions est inclu dans $L^1$ ou $L^1(1+x^2)dx$. On se place dans le cas où $\exists t^*$ tel que $v(t)<0$ pour $t>t^*$. On pourra se ramener facilement au cas $v(t)<0$.

On écrit l'équation ~\eqref{eq:lsc} sour la forme suivante :

\begin{equation}
	\label{eq:lscbis}
\begin{cases}
	\displaystyle \frac{\mathrm{d}}{\mathrm{d}t} (\int_0^\infty x y(x,t)dx + c(t)) = 0 \\
	c(t) - \dep = \mass - \mu_1(t) - dep \\
	(c(t)-\dep(0))y(0,t)\mathbb{I}_{c(t)-\dep(0) > 0} = 0 \\
	\displaystyle \frac{\partial y}{\partial t}+ (c(t)-\dep) \frac{\partial}{\partial x}) = 0 \\
	y(x,t^*) = y_0(x)
\end{cases}
\end{equation}

\begin{prop} 
	Soit $t^* < \tau$ tel aue $v(t)< - \epsilon$ pour tout $t \in [t^*, \tau]$, et $y_0 \in L^2(0,\infty)$, alors le problème de Cauchy~\eqref{eq:lscbis} admet une unique solution dans $C^0([t^*,tau],L^2(0,\infty))$.
\end{prop}

On pose le changement de variable suivant :

 \begin{equation}
	\begin{array}{ccccc}
	\Theta & : & [0,\infty] \times [t^*,\tau]& \to & [0,\infty] \times [0,T] \\
	      &    &        (x,t)             & \to &  (x, \int_{t^*}^t v(s)ds)\\
	\end{array}
\end{equation}

On pose $\theta (t) = \int_{t^*}^t v(s)ds$, et comme $v$ est strictement décroissante, $\Theta$ est un difféomorphisme. Alors $y$ est solution de l'équation~\eqref{eq:lsc} si et seulement si elle est solution de :

\begin{equation}
	\label{eq:lstrsp}
\begin{cases}
	v(\tilde{t}) = c(\tilde{t}) - \dep \\
	\displaystyle \partial _{\tilde{t}} \tilde{y}(\cdot,\tilde{t}) = - \partial_{x} \tilde{y}(\cdot, \tilde{t})\\
	\tilde{y}(x,0) = y_0 \circ \theta^{-1}
\end{cases}
\end{equation}

On retrouve ainsi l'existence, l'unicité de la solution de ~\eqref{eq:lstrsp}.

Peut-on élargir le résultat pour $v(x,t)$ ?

Sous $0 < \alpha \leq \dep' (x) \leq \beta$, on en déduit que la fonction $\dep$ est Lipschitzienne. Alors par la méthode des caractéristiques et le théorème de Cauchy-Lipschitz l'équation ~\eqref{eq:lstrsp2} admet une unique solution et la norme est conservée. L'équation~\eqref{eq:lscbis} peut s'écrire sous la forme :

\begin{equation}
	\label{eq:lstrsp2}
\begin{cases}
	\displaystyle \frac{\partial y}{\partial t}+ v(x,t)\frac{\partial y}{\partial x} + \dep '(x) y = 0 \\
	v(0,t)y(0,t)\mathbb{I}_{v(0,t) > 0} = 0 \\
	y(x,0) = y_0(x)
\end{cases}
\end{equation}





%%%%%%%%%%%%%%%%%%%%%%%%%%%%%
\subsection{Synthèse}
%%%%%%%%%%%%%%%%%%%%%%%%%%%%%

\begin{tabular}{|p{2cm}|p{2cm}| p{2cm}|p{2cm}|p{2cm}|p{2cm}|}
\hline
\multicolumn{2}{|c|}{$\dep \searrow $}&\multicolumn{3}{|c|}{$0<\alpha \leq \dep '(x) \leq \beta$} & $\dep ' = 0$\\
\hline
 $\dep (0)>\mass$ &$\mass \geq \dep(0)$&\multicolumn{2}{|c|}{$c_0 < \dep $} & $ \dep \leq c(0)$&       \\
 \hline 
                  &                    &$\dep (0)>\mass$ &$\mass \geq \dep(0)$ &       (*)  &        \\ 
\hline
\hline
$c\rightarrow \mass$&$c\rightarrow 0 $&   ramené         & $c\rightarrow \mass$ & convergence          & $ \mu_0 \rightarrow 0$ \\ 
 ($v<0 $)          & $ \mu_0 \rightarrow 0$&    à (*)    &  ($v<0 $)            &  vers un             &  $\mu_1 \rightarrow 0$ \\ 
                   & $ \mu_1 \rightarrow 0$& en temps    &                     &     Dirac             &        \\ 
				  & $\mu_1/\mu_0 \rightarrow \infty$&fini &                    &$c(t) \rightarrow c^*$ &        \\ 
				  & ($ v >0 $ )             &             &                    &                       &        \\ 
\hline
\hline
\end{tabular}



%%%%%%%%%%%%%%%%%%%%%%%%%%%%%
%%%%%%%%%%%%%%%%%%%%%%%%%%%%%%%%%%
%%%%%%%%%%%%%%%%%%%%%%%%%%%%%
\section{Assimilation de données pour dépolymérisation}
%%%%%%%%%%%%%%%%%%%%%%%%%%%%%




%%%%%%%%%%%%%%%%%%%%%%%%%%%%%
\subsection{Problème inverse}
%%%%%%%%%%%%%%%%%%%%%%%%%%%%%

Nous rappelons les résultats obtenus pour l'équation linéaire de dépolymérisation ($\pol =0$). L'équation~\eqref{eq:general} devient :

\begin{equation}
	\label{eq:depol}
	\begin{cases}
	    \frac{\partial y}{\partial t}-\dep \frac{\partial y} {\partial x}  = 0  \\
	    y(x,0) = y_{0} (x)\\
		y(l,t) = 0
	\end{cases}
\end{equation}

Dans ce cas la condition initiale peut s'écrire en fonction d'un moment d'ordre $n>0$ :

\[
\begin{cases}
\frac{\mathrm{d} \mu_n }{\mathrm{d}t} = -b n \mu_{n-1} \\
\frac{\mathrm{d} \mu_0 }{\mathrm{d}t} = -b y_0(bt)
\end{cases}
\]


On en déduit donc $y_0$ en fonction du moment d'ordre $n$.

\[
y_0(x) = \frac{1}{n! (-b)^{n+1}} \frac{\mathrm{d}^{n+1} \mu_n }{\mathrm{d}t^{n+1}}(\frac{x}{b})
\]

Par cette relation on peut déployer différentes méthodes (notamment la suite régularisante) pour obtenir un observateur optimal. 
On aimerait donc pouvoir obtenir une formule similaire pour notre problème où $v(t)$, au lieu d'être une constante égale à $-b$ est cette fois une fonction du premier moment $\mu_1$.


%%%%%%%%%%%%%%%%%%%%%%%%%%%%%
\subsection{Observabilité}
%%%%%%%%%%%%%%%%%%%%%%%%%%%%%

On suppose que la solution $y$ de l'équation~\eqref{eq:depol} est à support compact inclus dans $[0,l] \times [0,\tau]$. On définit les opérateurs de mesures $\Psi_n$ pour $n \geq 0$:

 \begin{equation}
	\begin{array}{ccccc}
	\Psi_n & : & L^2([0,l]) & \to & L^2([0,\tau]) \\
	 & & y_0 & \mapsto & t \to \int_0^l x^n y_0(x-\theta(t)) dx\\
	\end{array}
\end{equation}

On note les espaces vectoriels normés $\mathscr{X} = L^2 (0,l)$ et $\mathscr{Y} = L^2 (0,l) \times L^2 (0,\tau)$, alors $\Psi_n$ appartient à $L^2(X,L^2((0,\tau),Y))$. Le système~\eqref{eq:depol} est dit observable s'il existe une constante telle que :

\begin{equation}
	\label{obs}
	\forall y_0 \in \mathscr{X}, \quad \| \Psi_n(y_0)\|_{\mathscr{Y}}^2 \geq k_n \|y_0\|^2_{\mathscr{X}}
\end{equation}

Pour $y$ solution de ~\eqref{eq:depol}, cette inégalité n'est pas toujours vérifiée et le système n'est pas exactement observable.

Pour la démonstration, on peut exhiber un contre exemple sous la forme d'une suite $(y_0^m)_{m \in \mathbb{N}}$ telle que :

\[
\begin{cases}
	\|y_0^m\|^2_{L^2} \underset{m\to+\infty}{\nrightarrow 0} \\
	\| \Psi_n (y_0^m)\|_{L^2}^2 \underset{m\to+\infty}{\rightarrow 0}
\end{cases}
\]
	
En effet,

\[ 
\begin{split}
	\| \Psi_n (y_0^m)\|_{L^2}^2  &= \int_0^\tau  |\int_0^l x^n y^m(x,t) dx |^2 dt \\
	                             & \leq l^{2n}  \int_0^\tau (\int_0^l y^m(x,t)dx)^2dt \\
								 & \leq l^{2n} \int_0^\tau (\int_{bt}^l y_0^m(\xi)d\xi)^2dt \\
								 & \leq l^{2n} \int_0^\tau (\int_0^l y_0^m(\xi)d\xi)^2 dt
\end{split}
\]

Il suffit donc de trouver une fonction $y_0$ telle que :

\[
\begin{cases}
	\int_0^l (y_0^m(x))^2 dx \underset{m\to+\infty}{\nrightarrow 0} \\
	\int_0^l y_0^m(\xi)d\xi \underset{m\to+\infty}{\rightarrow 0}
\end{cases}
\]

C'est le cas par exemple de la fonction :

\[
\begin{cases}
	2n^3 & 0 <x < \frac{1}{2n^2}\\
	2n - 2n^3x & \frac{1}{2n^2} <x<\frac{1}{n^2}\\
	0 & \frac{1}{n^2} <x<l
\end{cases}
\]


En conclusion, on remarque qu'on peut étendre cette démontsration dès lors que la fonction $y$ reste dans le support compact $[0,l] \times [0,\tau]$.

Cependant, si l'observabilité n'est pas exacte dans $L^2$, elle l'est en revanche pour l'espace de Sobolev $H^{n+1}$
		
%%%%%%%%%%%%%%%%%%%%%%%%%%%%%
%%%%%%%%%%%%%%%%%%%%%%%%%%%%%
%%%%%%%%%%%%%%%%%%%%%%%%%%%%%%%%%%%%
\section{Vitesse de réaction dépendante du temps}
%%%%%%%%%%%%%%%%%%%%%%%%%%%%%

Le modèle que nous allons étudiez est le modèle de polymérisation-dépolymérisation suivant :

\begin{equation}
		\label{eq:poldepol}
		\begin{cases}
			\displaystyle \frac{\partial y}{\partial t}+ v(t) \frac{\partial y} {\partial x}  = 0 & (x,t) \in [0,L] \times [0, \tau] \\
             y(x,0) = y_{0} (x) \\
			 v(t) = M - \int_0^L y(x,t)dx - \dep \\
			 v(t)y(0,t)\mathbb{I}_{v(t) > 0} = 0 \\
		\end{cases}
\end{equation}

Et on ajoute une hypothèse sur la condition initiale $y_0$ :

\begin{equation}
	\label{hyp:compact}
	\exists l>0 \; t.q. \; \forall t \lea \tau, \; supp \, (y_{0}) \subset [0,l]\subset [0,L-\int_{0}^t v(s)ds]
\end{equation}


\begin{definition}
	On appelle y une solution classique de l'équation~\eqref{eq:poldepol} 
	sur un domaine $\Omega$ ouvert de $[0,L] \times [0, \tau]$ 
	si c'est une fonction $C^1$ de $x$ et de $t$ dans $\Omega$ 
	et si elle satisfait~\eqref{eq:poldepol} point par point dans $\Omega$.
\end{definition}

\begin{definition} 
	Pour toute solution classique de l'équation~\eqref{eq:poldepol}, on définit la fonction $\theta$ de la façon suivante : 
	\[ \theta (t) = \int_0^tv(t')dt'\].
\end{definition}

\begin{lemma}
	La fonction $\theta$ associée à une solution classique de l'équation~\eqref{eq:poldepol} est dans $C^2$.
\end{lemma}

\begin{proof}
	Soit $y$ une solution classique de l'équation~\eqref{eq:poldepol}. Alors en particulier $y \in C^1(\Omega)$.
	On pose $\Omega = \Omega_x \times \Omega_t$ 
	Donc la fonction $t \to v(t)$ appartient à $C^1(\Omega_t)$ et donc $\theta$ appartient à $C^2(\Omega_t)$  
\end{proof}


Par la suite, on cherche à distinguer deux cas : 
le cas où $\theta(t)<0$, il s'agit de la dépolymérisation, 
déjà connu et résolu sur le plan théorique, 
et $\theta(t)>0$ pour lequel $\mu_0$ est une constante.

ET ON VA VOIR SOUS QUELLES CONDITIONS C'EST RESPECTE !!

\subsection{Dépolymérisation - $v(t) < 0$}

On est ramené au cas traité dans la section précédente.

\subsection{Polymérisation - $v(t) > 0$}

On souhaite étudier dans cette partie le cas d'une vitesse positive. 
Sans perdre de généralité, 
on pose que le coefficient de polymérisation est égal à 1, 
soit $\pol =1$.

\begin{prop}
	\label{pol-moment}
	Soit $y$ solution de l'équation de transport :
	\begin{equation}
		\label{eq:pol}
		\begin{cases}
			\displaystyle \frac{\partial y}{\partial t}+ v(t) \frac{\partial y} {\partial x}  = 0 & (x,t) \in [0,L] \times [0, \tau] \\
             y(x,0) = y_{0} (x) 
			 v(t) = M - \int_0^L y(x,t)dx - \dep
		\end{cases}
	\end{equation}
	
	On suppose de plus que :
	\begin{itemize}
		\item $\exists t^* \; t.q. \; \forall t \geq t^* \; v(t) > \mu \geq 0$ ;
		\item $\exists l>0 \; t.q. \; supp \, (y_{0}) \subset [0,l]\subset [0,L-\int_{t^*}^t v(s)ds]$ ; 
	\end{itemize}
	
	Alors le moment d'ordre $n$ d'une solution de~\eqref{eq:pol} défini par :
	\[\mu_n (t) = \int_0^L x^n y(x,t) dx \]
	ne dépend que du temps et du moment d'ordre $n=0$.
\end{prop}

Sans perte de généralité, on pose par la suite $t^* = 0$. On pourra ensuite démontrer la proposition \ref{pol-moment} pour un $t^*$ quelconque par le changement de variable suivant $T: (x,t) \to (x,t^*)$ sachant que $T$ est un $C^\infty$-difféomorphisme.

Pour démontrer cette proposition, montrons d'abord le lemme suivant :

\begin{lemma}
	\label{pol-sol}
	Soit $y$ solution de l'équation ~\eqref{eq:pol}, et sous les hypothèses de la proposition \ref{pol-moment}, 
	
	alors la solution $(x,t) \to y(x,t)$ existe, elle est unique et $\forall (x,y) \in [0,L] \times [0, \tau]$, 
	\[y(x,t) = y_0(x-\int_{0}^t v(s)ds) \]
	
	et notamment $\forall n \geq 0, \; \forall t>0$, on a 
	\[\mu_n(t) = \int_0^L x^n y_0(x-\int_{0}^t v(s)ds) \, dx \].

\end{lemma}

\begin{proof}
	(lemme \ref{pol-sol})
	
	A DEMONTRER EXISTENCE UNICITE PAR LE POLY M1 EQ DE TRANSPORT - METHODE DES CARACTERISTIQUES
\end{proof}

La solution de l'équation~\eqref{eq:poldepol} 
par la méthode des caractéristiques s'écrit donc :

\[ y(x,t) = y_0(x- \theta(t)) \]

On définit l'application de $\Psi_n$ pour $n \geq 0$:

 \begin{equation}
	\begin{array}{ccccc}
	\Psi_n & : & L^2([0,l]) & \to & L^2([0,\tau]) \\
	 & & y_0 & \mapsto & t \to \int_0^L x^n y_0(x-\theta(t)) dx\\
	\end{array}
\end{equation}
  
Alors, toujours sous l'hypothèse~\eqref{hyp:compact}, on peut également effectué le changement de variable $x =x'+\theta(t)$ et
  
\[ \Psi_n (y_0) (t) = \int_0^L x^n y_0(x) dx\ = \int_{-\theta(t)}^{L - \theta(t)} (x'+\theta(t))^ny_0(x')dx'\]
  
On montre par intégration par partie pour $n>0$:
  
\[
\begin{split}
	\frac{\mathrm{d} \Psi_n (y_0) }{\mathrm{d}t} &= \frac{\mathrm{d}}{\mathrm{d}t}\int_0^L x^n y_0(x-\theta(t)) dx \\
	                                             &= \int_0^L x^n \frac{\partial}{\partial t}y_0(x-\theta(t)) dx \\
												 &= \int_0^L x^n (-v(t))y_0'(x-\theta(t)) dx \\
												 &= -v(t)[x^n y_0(x-\theta(t))]_0^L + v(t) n \int_0^l x^{n-1} y_0(x-\theta(t)) dx\\
\end{split}
\]

Or, d'après l'hypothèse~\eqref{hyp:compact}, la fonction $y_0$ est nulle sur $[L-\theta(t);L]$ et ce $\forall t >0$. Donc :

\[\frac{\mathrm{d} \Psi_n (y_0) }{\mathrm{d}t}= v(t) n \mu_{n-1}\]

De plus pour $n=0$ :

\[ 
\begin{split}
\frac{\mathrm{d} \Psi_0 (y_0) }{\mathrm{d}t} &= \frac{\mathrm{d} \mu_n }{\mathrm{d}t} \\
                                             &= \frac{\mathrm{d}}{\mathrm{d}t}\int_0^l y_0(x-\theta(t)) dx \\
	                                         &= \int_0^l \frac{\partial}{\partial t}y_0(x-\theta(t)) dx \\
											 &= \int_0^l -v(t)y_0'(x-\theta(t)) dx \\
											 &= -v(t)[y_0(x-\theta(t))]_0^l  \\
											 &= + v(t) y_0(-\theta(t)) 
\end{split}
\]

Donc, pour une solution $y$ qui s'écrit sous la forme $y(x,t) = y_0(x- \theta(t))$ 
avec $y_0$ vérifiant l'hypothèse~\eqref{hyp:compact}, 
alors $y$ est solution de~\eqref{eq:poldep} 
et ses moments vérifient l'équation suivante :

\begin{equation}
	\label{mmt}
\begin{cases}
\displaystyle \frac{\mathrm{d} \mu_n }{\mathrm{d}t} = v(t) n \mu_{n-1} & n>0 \\
\displaystyle \frac{\mathrm{d} \mu_0 }{\mathrm{d}t} = + v(t) y_0(-\theta(t))
\end{cases}
\end{equation}


Nous allons donc pouvoir démontrer la proposition.

\begin{proof}
	(proposition \ref{pol-moment})
	
	On se propose de résoudre pour $t>0$:
	\begin{equation}
		\label{mmtpol}
	\begin{cases}
	\displaystyle \frac{\mathrm{d} \mu_n }{\mathrm{d}t} = (\pol(\mass - \mu_1)-\dep)n \mu_{n-1} & n>0 \\
	\displaystyle \frac{\mathrm{d} \mu_0 }{\mathrm{d}t} = 0
	\end{cases}
	\end{equation}
	
	Nous allons démontrer la proposition par récurrence. 

	\underline{$n=1$} :

	\begin{equation}
		\begin{cases}
			\mu_1 (t) = (\mass - \dep/\pol) - (\mass - \dep/\pol)e^{-\pol \mu_0 (t-t^*)} \\
			v(t) = \pol( \mass - \mu_1 (t)) - \dep = v(t^*)e^{-\pol \mu_0 (t-t^*)}\\
			\theta (t) = \theta(t^*) + \int_{t^*}^t v(t')dt' = \frac{v(t^*)}{\pol \mu_0} e^{-\pol \mu_0 t^*}(1-e^{-\pol \mu_0 (t-t^*)})
		\end{cases}
	\end{equation}
	
	\vspace{0.5}
	
	\underline{$n$ 	\Rightarrow $n+1$}
\end{proof}


Donc, si $\mu_0$ est une constante, la masse va entièrement polymériser jusqu'à ce que $\mu_1(t)$ tende vers $\mass-\dep/\pol$. D'après la partie précédente, on a comme hypothèse sous jacente que $\dep \leq \mass$ (sinon la vitesse devient négative car l'état asymptotique du système est la dépolymérisation totale). Toute la masse est polymérisée à l'infini et $v(t)$ tend vers $0$. On note que ce résultat est en contracdiction avec l'hypothèse de bord pour $x=l$ qui nous a permis d'obtenir ces équations, cependant celles-ci ne perdent pas de leurs généralités pour $l= \infty$. 


   
%%%%%%%%%%%%%%%%%%%%%%%%%%%%%%%%%
%%%%%%%%%%%%%%%%%%%%%%%%%%%%%%%%%
%%%%%%%%%%%%%%%%%%%%%%%%%%%%%%%%%

\section{Conclusion}
%%%%%%%%%%%%%%%%%%%%%%%%%%%%%%%%%

Les précédents calculs nous ont permis de mettre en valeur le fait que, bien que le problème soit linéaire, 
la polymérisation ajoute une difficulté à traiter : le domaine ne peut plus être considéré comme borné, sauf à travailler en temps 
fini. 
Dans ce dernier cas, nous avons montré que les méthodes développées pour la dépolymérisation ne peuvent s'appliquer.
De plus, si le domaine n'est plus borné, l'espace fonctionnel des solutions à considéré n'est plus $L^2$,
alors qu'il s'agit de l'espace fonctionnel de la théorie du contrôle et de l'assimilation de données.

Une façon de contourner ce problème pourrait être de définir un coefficient de dépolymérisation 
variable en fonction la taille $x$ des polymères :

\begin{equation}
	\label{coeff}
	\begin{cases}
		\pol = 1 \\
		0 < \alpha \leq \dep' (x) \leq \beta & \forall x \in \mathbb{R}_+ 
	\end{cases}
\end{equation}

On reprend l'équation~\eqref{eq:lstrsp2} où l'on suppose que $ \dep '(x) = \alpha $:

\begin{equation}
	\label{coeff}
	\begin{cases}
		\displaystyle \frac{\partial y}{\partial t}+ v(x,t)\frac{\partial y}{\partial x} = - \alpha y \\
		v(0,t)y(0,t)\mathbb{I}_{v(0,t) > 0} = 0 \\
		y(x,0) = y_0(x)
	\end{cases}
\end{equation}

Cette équation semble plus difficile à résoudre. On conserve l'écriture d'une solution analytique 
par la méthode des caractéristiques et la formule de Duhamel :

\[
\begin{cases}
\displaystyle \frac{\mathrm{d}}{\mathrm{d}s} X(s;x,t)= (\mass - \mu_1(s)) - \dep(X(s;x,t))\\
X(t;x,t) = x
\end{cases}
\]

Alors $(s;x,t) \to X(s;x,t)$ est solution de :

\[
\begin{cases}
\displaystyle \frac{\partial X}{\partial x} = exp \, (\alpha (s-t))\\
\\
\displaystyle \frac{\partial X}{\partial t} = - ((\mass - \mu_1(t)) - \alpha x ) \, exp \, (\alpha (s-t))\\
\end{cases}
\]

Et la solution s'écrit en fonction de $X$ :

\[
y(x,t) = y(X(s;x,t),s) exp \, (- \alpha s)
\]



On conserver également le caractère $L^2$ sous certaine condition (démonstration à faire).
Mais on perd une asymptote de fuite à l'infini qui se produisait dans la partie précédente.

Attention, on n'a plus de relation simple entre les moments $\mu$ (pour $n>0$):

\[ 
\begin{split}
	\frac{\mathrm{d} \mu_n }{\mathrm{d}t} & = \int_0^l x^{n} \frac{\partial }{\partial t} y(x,t) dx \\
	                                      & = - n \int_0^l x^{n-1} v(x,t) y(x,t)) dx \\
										  & = -n \int_0^l x^{n-1} ((\mass - \mu_1(t)) - \alpha x )y(x,t) dx \\
										  & = -n (\mass - \mu_1(t)) \mu_{n-1} + n \alpha \mu_{n}
\end{split}	
\]

et

\[ 
\begin{split}
	\frac{\mathrm{d} \mu_0 }{\mathrm{d}t} & = \int_0^l \frac{\partial }{\partial t} y(x,t) dx \\
	                                      & = - \int_0^l \frac{\partial }{\partial x} v(x,t) y(x,t) dx \\
										  & = - [(\mass - \mu_1(t)) - \alpha x )y(x,t)]_0^l \\
										  & = (\mass - \mu_1(t))y(0,t)
\end{split}	
\]

Soit :

\begin{equation}
	\label{moment}
	\begin{cases}
		\displaystyle \frac{\mathrm{d} \mu_n }{\mathrm{d}t} = -n (\mass - \mu_1(t)) \mu_{n-1} + n \alpha \mu_{n} & n >0 \\
		\\
		\displaystyle \frac{\mathrm{d} \mu_0 }{\mathrm{d}t} = (\mass - \mu_1(t))y(0,t)
	\end{cases}
\end{equation}

On voit que l'on a pu généraliser le résultat précédent : si on polymérise sur un espace borné, 
ie si $v(0,t) >0$ et $ v(0,t)y(0,t)\mathbb{I}_{v(0,t)>0} =0$,
 alors on a une contradiction et une polymérisation vers l'infini.
			 

%%%%%%%%%%%%%%%%%%%%%%%%%%%%%%%%%%%%%%%%%%%%%%%%%%%%%%%%%%%%%%%%%%%%%%%%%%% REFERENCES

\medskip

\bibliographystyle{unsrt}%Used BibTeX style is unsrt
\bibliography{20190106biblio}
	
\end{document}

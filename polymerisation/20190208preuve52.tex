%% PRE EDITION
\documentclass[a4paper]{article}
\usepackage[utf8]{inputenc}
\usepackage[T1]{fontenc}
\usepackage[french]{babel}
\usepackage{soul}
\usepackage[pdftex]{graphicx}

\usepackage{amsfonts}
\usepackage{amsthm}
\usepackage{amsmath}
\usepackage{amssymb}
\usepackage{mathrsfs}
\usepackage{booktabs}
\usepackage{siunitx}
\usepackage{thmtools}

%% LAYOUT
\declaretheoremstyle[
    bodyfont=\normalfont\color{red},
    headfont=\color{red}
]{styleattention}

\declaretheoremstyle[
    spacebelow=1em
]{styleremarque}

\declaretheoremstyle[
    spaceabove=-6pt, 
    spacebelow=6pt, 
    headfont=\normalfont\bfseries, 
    bodyfont = \normalfont,
    postheadspace=1em, 
    qed=$\Box$, 
    headpunct={$\rhd$}
]{mystyle} 

\declaretheorem[thmbox=M,numberwithin=section,title=Définition]{definition}
\declaretheorem[thmbox=M,sibling=definition]{proposition}
\declaretheorem[thmbox=M,sibling=definition]{corollaire}
\declaretheorem[thmbox=M,sibling=definition,title=Théorème]{theoreme}
\declaretheorem[thmbox=M,sibling=definition]{lemme}
\declaretheorem[thmbox=M,sibling=definition,title=Propriété]{propriete}
\declaretheorem[thmbox=M,sibling=definition,title=Propriétés]{proprietes}
\declaretheorem[style=styleremarque,sibling=definition,title=Remarque]{remarque}
\declaretheorem[style=styleattention,title=À revoir]{Arevoir}
\declaretheorem[name={}, style=mystyle, unnumbered]{preuve}


\renewcommand\qedsymbol{$\blacksquare$}

%% NEW COMMAND
\newcommand{\mass}{\mathrm{M}}
\newcommand{\pol}{a}
\newcommand{\dep}{b}

\usepackage{geometry}
\geometry{hmargin=3cm,vmargin=2.5cm}

\usepackage{tabularx}
\usepackage{float}

\title{Retour sur la propriété 5.2}
\author{Cécile Della Valle}

%% DEBUT DE REDACTION
\begin{document}

\maketitle


%%%%%%%%%%%%%%%%%%%%%%%%%%%%%%%%%%%%%%%%%%%%%%%%%%%%%%%%%%%%%%%%%%%%%%%%%%%%%%%%%%%%%%%%%%%%%%%%%%%%%%%%%%%%%%%%%%%%%%%%%%%%%%%%%
%%%%%%%%%%%%%%%%%%%%%%%%%%%%%%%%%%%%%%%%%%%%%%%%%%%%%%%%%%%%%%%%%%%%%%%%%%%%%%%%%%%%%%%%%%%%%%%%%%%%%%%%%%%%%%%%%%%%%%%%%%%%%%%%%
%%%%%%%%%%%%%%%%%%%%%%%%%%%%%%%%%%%%%%%%%%%%%%%%%%%%%%%%%%%%%%%%%%%%%%%%%%%%%%%%%%%%%%%%%%%%%%%%%%%%%%%%%%%%%%%%%%%%%%%%%%%%%%%%%
\section{Introduction}
%%%%%%%%%%%%%%%%%%%%%%%%%%%%%%%%%%%%%%%%%%%%%%%%%%%%%%%%%%%%%%%%%%%%%%%%%%%%%%%%%%%%%%%%%%%%%%%%%%%%%%%%%%%%%%%%%%%%%%%%%%%%%%%%%
%%%%%%%%%%%%%%%%%%%%%%%%%%%%%%%%%%%%%%%%%%%%%%%%%%%%%%%%%%%%%%%%%%%%%%%%%%%%%%%%%%%%%%%%%%%%%%%%%%%%%%%%%%%%%%%%%%%%%%%%%%%%%%%%%

Ce document est consacré à la démonstration de la propriété 
du comportement en temps long
(ci après notée ~\ref{prop:v}), 
qui étudie les variations de la fonction $v \in C^0([0,\tau])$ 
(vitesse du système de Lifschitz-Slyozov) 
en fonction des conditions initiales et des paramètres du système.

Nous rappelons dans un premier temps le système étudié et les équations vérifiées par les moments de la solution.

Puis, dans un deuxième temps, nous énnonçons la proposition et procédons à la démonstration 
(à noter que ce document est transitoire et que la preuve reste pour l'heure incomplète).


%%%%%%%%%%%%%%%%%%%%%%%%%%%%%%%%%%%%%%%%%%%%%%%%%%%%%%%%%%%%%%%%%%%%%%%%%%%%%%%%%%%%%%%%%%%%%%%%%%%%%%%%%%%%%%%%%%%%%%%%%%%%%%%%%
%%%%%%%%%%%%%%%%%%%%%%%%%%%%%%%%%%%%%%%%%%%%%%%%%%%%%%%%%%%%%%%%%%%%%%%%%%%%%%%%%%%%%%%%%%%%%%%%%%%%%%%%%%%%%%%%%%%%%%%%%%%%%%%%%
%%%%%%%%%%%%%%%%%%%%%%%%%%%%%%%%%%%%%%%%%%%%%%%%%%%%%%%%%%%%%%%%%%%%%%%%%%%%%%%%%%%%%%%%%%%%%%%%%%%%%%%%%%%%%%%%%%%%%%%%%%%%%%%%%
\section{Notation}
%%%%%%%%%%%%%%%%%%%%%%%%%%%%%%%%%%%%%%%%%%%%%%%%%%%%%%%%%%%%%%%%%%%%%%%%%%%%%%%%%%%%%%%%%%%%%%%%%%%%%%%%%%%%%%%%%%%%%%%%%%%%%%%%%
%%%%%%%%%%%%%%%%%%%%%%%%%%%%%%%%%%%%%%%%%%%%%%%%%%%%%%%%%%%%%%%%%%%%%%%%%%%%%%%%%%%%%%%%%%%%%%%%%%%%%%%%%%%%%%%%%%%%%%%%%%%%%%%%%

\begin{itemize}
\item $y$ -- fonction de la concentration des polymères en fonction de leur taille x ;
\item $c$ -- fonction de concentration du monomère ;
\item $v(t)$ -- vitesse totale de réaction (dépolymérisation et polymérisation) ;
\item $\theta(t)$ -- fonction intégrale de la vitesse $\theta(t)=\int_0^t v(s)ds$ ;
\item $\pol$ -- coefficient de polymérisation, par la suite on posera $\pol=1$;
\item $\dep$ -- coefficient de dépolymérisation ;
\item $\mu_i$ -- moment d'ordre i de la fonction y, soit $\mu_i = \int_0^L x^i y(x,t)dx$.
\end{itemize}



%%%%%%%%%%%%%%%%%%%%%%%%%%%%%%%%%%%%%%%%%%%%%%%%%%%%%%%%%%%%%%%%%%%%%%%%%%%%%%%%%%%%%%%%%%%%%%%%%%%%%%%%%%%%%%%%%%%%%%%%%%%%%%%%%
%%%%%%%%%%%%%%%%%%%%%%%%%%%%%%%%%%%%%%%%%%%%%%%%%%%%%%%%%%%%%%%%%%%%%%%%%%%%%%%%%%%%%%%%%%%%%%%%%%%%%%%%%%%%%%%%%%%%%%%%%%%%%%%%%
%%%%%%%%%%%%%%%%%%%%%%%%%%%%%%%%%%%%%%%%%%%%%%%%%%%%%%%%%%%%%%%%%%%%%%%%%%%%%%%%%%%%%%%%%%%%%%%%%%%%%%%%%%%%%%%%%%%%%%%%%%%%%%%%%
\section{Synthèse des prérequis}
%%%%%%%%%%%%%%%%%%%%%%%%%%%%%%%%%%%%%%%%%%%%%%%%%%%%%%%%%%%%%%%%%%%%%%%%%%%%%%%%%%%%%%%%%%%%%%%%%%%%%%%%%%%%%%%%%%%%%%%%%%%%%%%%%
%%%%%%%%%%%%%%%%%%%%%%%%%%%%%%%%%%%%%%%%%%%%%%%%%%%%%%%%%%%%%%%%%%%%%%%%%%%%%%%%%%%%%%%%%%%%%%%%%%%%%%%%%%%%%%%%%%%%%%%%%%%%%%%%%

On rappelle ici quelques prérequis utiles pour la preuve.

\subsection{Définitions}
%%%%%%%%%%%%%%%%%%%%%%%%%%%%%%%%%%%%%%%%%%%%%%%%%%%%%%%%%%%%%%%%%%%%%%%%%%%%%%%%%%%%%%%%%%%%%%%%%%%%%%%%%%%%%%%%%%%%%%%%%%%%%%%%%

Le modèle que nous allons étudier est le modèle de polymérisation-dépolymérisation suivant 
sur un compact $\Omega = [0,L] \times [0,\tau]$. 
Sans perdre de généralité, 
on pose que le coefficient de polymérisation est égal à 1, 
soit $\pol =1$.


On souhaite caractériser le comportement asymptotique des solutions du problème de Cauchy :

\begin{equation}
		\label{eq:poldep}
		\begin{cases}
			\displaystyle \frac{\partial y}{\partial t}+ v(t) \frac{\partial y} {\partial x}  = 0 & (x,t) \in [0,L] \times [0, \tau] \\
             y(x,0) = y_{0} (x) & x\in[0,L]\\
			 v(t) = M - \int_0^L x y(x,t)dx - \dep & t \in [0,\tau]\\
			 v(t)y(0,t)\mathbb{I}_{v(t) > 0} = 0 \\
			 v(t)y(L,t)\mathbb{I}_{v(t) < 0} = 0 \\
		\end{cases}
\end{equation}

Et on suppose que la condition initiale $y_0$ est à support compact.

\begin{equation}
	\label{hyp:compact}
	\exists l>0, \; supp(y_0) \, \subset \, [0,l]
\end{equation}

\begin{lemme}
	Soit $y_0$ vérifiant l'hypothèse~\eqref{hyp:compact},
	et $y$ la solution associée à $y_0$ de ~\eqref{eq:poldep} 
	alors 
 \[\forall t \in [0,\tau], \; supp \, (y(\cdot, t)) \subset [0,L]\subset [0,l-\int_{0}^t v(s)ds] \]
 \end{lemme}

On peut donc remplacer la condition aux limites en $x=L$ par $y(L,t)=0$, et ce quelque soit le signe de la vitesse.


%%%%%%%%%%%%%%%%%%%%%%%%%%%%%%%%%%%%%%%%%%%%%%%%%%%%%%%%%%%%%%%%%%%%%%%%%%%%%%%%%%%%%%%%%%%%%%%%%%%%%%%%%%%%%%%%%%%%%%%%%%%%%%%%%
\subsection{Equations des moments et conséquences}
%%%%%%%%%%%%%%%%%%%%%%%%%%%%%%%%%%%%%%%%%%%%%%%%%%%%%%%%%%%%%%%%%%%%%%%%%%%%%%%%%%%%%%%%%%%%%%%%%%%%%%%%%%%%%%%%%%%%%%%%%%%%%%%%%


%%%%%%%%%%%%%%%%%%%%%%%%%%%%%%%%%%%%%%%%%%%%
\begin{lemme}
	\label{lemmesol}
	Soit $\tau>0$, $\mass$ et $\dep$ deux constantes de $\mathbb{R}^+$, 
	$l>0, L>0$, $y_0 \in L^2([0,L])$.
	Soit $y$ solution classique de l'équation ~\eqref{eq:poldep} et l'hypothèse~\eqref{hyp:compact} vérifiée.
	
	On suppose que la solution $(x,t) \to y(x,t)$ existe, qu'elle est unique. 
	D'après la méthode des caractéristiques, cette solution peut s'écrire,
	 $\forall (x,y) \in [0,L] \times [0, \tau]$, 
	\[y(x,t) = y_0(x-\int_{0}^t v(s)ds) \]
	
	alors $\forall n \geq 0, \; \forall t>0$, on a 
	\[\mu_n(t) = \int_0^L x^n y_0(x-\int_{0}^t v(s)ds) \, dx \]
	
	et pour une solution classique de~\eqref{eq:poldep}, on a la relation : 
	\begin{equation}
		\label{eq:mmt}
		\begin{cases}
			\displaystyle \frac{\mathrm{d} \mu_n }{\mathrm{d}t} = v(t) n \mu_{n-1} & n>0 \\
			\displaystyle \frac{\mathrm{d} \mu_0 }{\mathrm{d}t} = v(t) y_0(max(0,-\theta(t))
		\end{cases}
	\end{equation}	
	
\end{lemme}
%%%%%%%%%%%%%%%%%%%%%%%%%%%%%%%%%%%%%%%%%%%%
  
  
%%%%%%%%%%%%%%%%%%%%%%%%%%%%%%%%%%%%%%%%%%%%%%%%%%%%%%%%%%%%%%%%%%%%%%%%%%%%%%%%%%%%%%%%%%%%%%%%%%%%%%%%%%%%%%%%%%%%%%%%%%%%%%%%%
%%%%%%%%%%%%%%%%%%%%%%%%%%%%%%%%%%%%%%%%%%%%%%%%%%%%%%%%%%%%%%%%%%%%%%%%%%%%%%%%%%%%%%%%%%%%%%%%%%%%%%%%%%%%%%%%%%%%%%%%%%%%%%%%%
%%%%%%%%%%%%%%%%%%%%%%%%%%%%%%%%%%%%%%%%%%%%%%%%%%%%%%%%%%%%%%%%%%%%%%%%%%%%%%%%%%%%%%%%%%%%%%%%%%%%%%%%%%%%%%%%%%%%%%%%%%%%%%%%%
\section{Proposition}
%%%%%%%%%%%%%%%%%%%%%%%%%%%%%%%%%%%%%%%%%%%%%%%%%%%%%%%%%%%%%%%%%%%%%%%%%%%%%%%%%%%%%%%%%%%%%%%%%%%%%%%%%%%%%%%%%%%%%%%%%%%%%%%%%
%%%%%%%%%%%%%%%%%%%%%%%%%%%%%%%%%%%%%%%%%%%%%%%%%%%%%%%%%%%%%%%%%%%%%%%%%%%%%%%%%%%%%%%%%%%%%%%%%%%%%%%%%%%%%%%%%%%%%%%%%%%%%%%%%
 


%%%%%%%%%%%%%%%%%%%%%%%%%%%%%%%%%%%%%%%
\begin{proposition}
	\label{prop:v}
	Soit $\tau>0$, $\mass$ et $\dep$ deux constantes de $\mathbb{R}^+$, 
	$l>0, L>0$, $y_0 \in L^2([0,L])$.
	Soit $y$ solution classique de l'équation ~\eqref{eq:poldep} et l'hypothèse~\eqref{hyp:compact} vérifiée.
	
	Si $\mass-\dep < 0 $,
	\begin{itemize}
		\item  alors $\forall t \in [0,\tau] \;, \;$ on a $v(t) <0$. 
		De plus, si $\tau> \displaystyle \frac{L}{\dep-\mass}$, alors il existe $t^* \in [0,\tau]$ tel que $\mu_0(t^*)=0$ et $\mu_1(t^*) = 0$. 
	\end{itemize}
	
	Sinon, $0 < \mass-\dep$,
	\begin{itemize}
	 \item soit $v(0)<0$, alors $\forall t>0 $, on a $v(t)<0$ et $\mathrm{lim}_{t \rightarrow \infty} v(t) = 0$. 
	 \item soit $v(0)>0$ alors $\forall t \in [0,\tau] \;$ on a $v(t) >0$ et $\mathrm{lim}_{t \rightarrow \infty} v(t) = 0$.
	 \end{itemize}
\end{proposition}
%%%%%%%%%%%%%%%%%%%%%%%%%%%%%%%%%%%%%%%


\begin{preuve}
	Soit l'unique solution classique $y$ du problème de Cauchy~\eqref{eq:poldep}, 
	vérifiant l'hypothèse~\eqref{hyp:compact},
	alors la vitesse $v$ de réaction s'écrit pour tout temps $t \in [0,\tau]$ par définition :
	\[v(t) = \mass -\int_0^L x y(x,t)dx -\dep\]

	\vspace{0.5cm}
	\underline{Si $\mass-\dep < 0 $} :
	
 	Sachant la définition de la vitesse $v$, on a l'implication :
 	\[ v(t) = \mass -\dep - \mu_1(t) <0 \; \iff \; \; \mass-\dep < \int_0^L x y(x,t)dx  \]
	
	Comme $y$ est une densité, elle est positive et cette hypothèse est toujours vérifiée.
	
	De plus, $\forall t>0$, $v(t)> \mass-\dep$, donc $|\theta (t)|> t^*\times
	|\mass-\dep|$.
	Donc pour $t^* = \frac{L}{\mass-\dep}>0$, par hypothèse $t^* \in [0,\tau]$, et on a $ \mu_1(t^*) = \int_{-\theta(t)}^L xy_0(x)dx = 0$ et donc $\mu_0(t^*) =0$.
	
	
	\vspace{0.5cm}
	\underline{Sinon, pour $0<\mass-\dep$} :
	
	\[ \frac{\mathrm{d} v }{\mathrm{d}t} =  - \frac{\mathrm{d} \mu_1 }{\mathrm{d}t} = - v(t) \mu_0(t) \]
	
	\textbf{v(0)<0} :
	
	Sachant qu'à $t=0$ la vitesse $v(0)<0$, $\mu_0 >0$, la vitesse est négative, 
	alors $\mu_1$ est une fonction décroissante et $v$ est une fonction strictement croissante de $[0,\tau]$.
	L'intégrale de la vitesse $v$ (la fonction $\theta$) est donc une fonction négative décroissante.
	\[ 
	\begin{split}
		\mu_1(t) & = \int_0^L x y(x,t)dx \\
		      & = \int_0^L x y_0(x-\theta(t)) dx \\
		      & = \int_{-\theta(t)}^{L} (x+\theta(t)) y_0(x) dx \\
	\end{split}
	\]
	La vitesse $v$ ne change pas de signe tant que $\mu_1 > \mass- \dep >0$. 
	Donc
	\[ \mu_1 > \mass- \dep >0 \implies \mu_1 \searrow \]
    Donc $\mu_1$ est une fonction décroissante strictement et minorée, 
	et elle converge vers son minorant.
	
	Supposons qu'il exite $t^*>0$ tel que $\mu_1(t^*)= \mass -\dep $, 
	alors $v(t^*)=0$, $\frac{\mathrm{d} v }{\mathrm{d}t} (t^*)= 0$.
	
	Si $\mu_0(t^*)>0$ alors $\forall t<t^*$, $0<\mu_0(t^*)<\mu_0(t)$ et on note $ \mu_0(t^*)= \alpha$, et il vient :
	
	\[ \frac{\mathrm{d} |v| }{\mathrm{d}t} = \mu_0(t)|v(t)| \geq \alpha |v(t)| \]
	
	Donc d'après le lemme de Grönwall :
	\[ |v(t)| \geq |v(0)| e^{-\alpha t} \]
	
	Ceci contredit le fait que $v$ s'annule en $t=t^*$.
	Donc nécessairement $\mu_0(t^*) = 0$, 
	or, $x \to xy (x,t)$ et $x \to y(x,t)$ sont des fonctions positives de même support,
	donc $\mu_1(t^*) =0 \implies \mu_0(t^*)=0$ .
	
	En effet, le système est entièrement dépolymérisé et le système vérifie :
	
	$\forall t>t^*$:
	\begin{itemize}
		\item $\theta(t)= \theta(t^*) \geq L $
		\item $y(x,t)=y(x,t^*)= 0$
	\end{itemize}
	
	Donc $v(t^*) = \mass - \dep - \mu_1(t^*) = \mass- \dep > 0$.
	
	C'est une contradiction.	
	
	
	\vspace{0.3cm}
	\textbf{v(0)>0}
 	
	Alors $\mu_0>0$ est une constante et 
 	\[ v(t) = ( \mass - \mu_1 (t)) - \dep = v(0)e^{- \mu_0 t} \]
 	On voit donc que la vitesse ne change jamais de signe sur $[0,\tau]$.
	Cette formule explicite garantie de plus que et $\mathrm{lim}_{t \rightarrow \infty} v(t) = 0$.
	
\end{preuve}







%%%%%%%%%%%%%%%%%%%%%%%%%%%%%%%%%%%%%%%%%%%%%%%%%%%%%%%%%%%%%%%%%%%%%%%%%%% REFERENCES


\end{document}

%% PRE EDITION
\documentclass[a4paper]{article}
\usepackage[utf8]{inputenc}
\usepackage[T1]{fontenc}
\usepackage[french]{babel}
\usepackage{soul}
\usepackage[pdftex]{graphicx}

\usepackage{amsfonts}
\usepackage{amsthm}
\usepackage{amsmath}
\usepackage{amssymb}
\usepackage{mathrsfs}
\usepackage{booktabs}
\usepackage{siunitx}
\usepackage{thmtools}

%% LAYOUT
\declaretheoremstyle[
    bodyfont=\normalfont\color{red},
    headfont=\color{red}
]{styleattention}

\declaretheoremstyle[
    spacebelow=1em
]{styleremarque}

\declaretheoremstyle[
    spaceabove=-6pt, 
    spacebelow=6pt, 
    headfont=\normalfont\bfseries, 
    bodyfont = \normalfont,
    postheadspace=1em, 
    qed=$\Box$, 
    headpunct={$\rhd$}
]{mystyle} 

\declaretheorem[thmbox=M,numberwithin=section,title=Définition]{definition}
\declaretheorem[thmbox=M,sibling=definition]{proposition}
\declaretheorem[thmbox=M,sibling=definition]{corollaire}
\declaretheorem[thmbox=M,sibling=definition,title=Théorème]{theoreme}
\declaretheorem[thmbox=M,sibling=definition]{lemme}
\declaretheorem[thmbox=M,sibling=definition,title=Propriété]{propriete}
\declaretheorem[thmbox=M,sibling=definition,title=Propriétés]{proprietes}
\declaretheorem[style=styleremarque,sibling=definition,title=Remarque]{remarque}
\declaretheorem[style=styleattention,title=À revoir]{Arevoir}
\declaretheorem[name={}, style=mystyle, unnumbered]{preuve}


\renewcommand\qedsymbol{$\blacksquare$}

%% NEW COMMAND
\newcommand{\mass}{\mathrm{M}}
\newcommand{\pol}{p}
\newcommand{\dep}{d}
\newcommand{\Y}{\mathscr{U}}
\newcommand{\Z}{\mathscr{Z}}

\usepackage{geometry}
\geometry{hmargin=3cm,vmargin=2.5cm}

\usepackage{tabularx}
\usepackage{float}

\title{Examen : Probèmes inverses et dynamique des population}
\author{Cécile Della Valle}

%% DEBUT DE REDACTION
\begin{document}

\maketitle

%%%%%%%%%%%%%%%%%%%%%%%%%%%%%%%%%%%%%%%%%%%%%%%%%%%%%%%%%%%%%%%%%%%%%%%%%%%%%%%%%%%%%%%%%%%%%%%%%%%%%%%%%%%%%%%%%%%%%%%%%%%%%%%%%
%%%%%%%%%%%%%%%%%%%%%%%%%%%%%%%%%%%%%%%%%%%%%%%%%%%%%%%%%%%%%%%%%%%%%%%%%%%%%%%%%%%%%%%%%%%%%%%%%%%%%%%%%%%%%%%%%%%%%%%%%%%%%%%%%
%%%%%%%%%%%%%%%%%%%%%%%%%%%%%%%%%%%%%%%%%%%%%%%%%%%%%%%%%%%%%%%%%%%%%%%%%%%%%%%%%%%%%%%%%%%%%%%%%%%%%%%%%%%%%%%%%%%%%%%%%%%%%%%%%
\section{Introduction}
%%%%%%%%%%%%%%%%%%%%%%%%%%%%%%%%%%%%%%%%%%%%%%%%%%%%%%%%%%%%%%%%%%%%%%%%%%%%%%%%%%%%%%%%%%%%%%%%%%%%%%%%%%%%%%%%%%%%%%%%%%%%%%%%%
%%%%%%%%%%%%%%%%%%%%%%%%%%%%%%%%%%%%%%%%%%%%%%%%%%%%%%%%%%%%%%%%%%%%%%%%%%%%%%%%%%%%%%%%%%%%%%%%%%%%%%%%%%%%%%%%%%%%%%%%%%%%%%%%%


 
\begin{equation}
		\label{eq:poldep}
		\begin{cases}
			\displaystyle \frac{\partial y}{\partial t}+ v(t) \frac{\partial y} {\partial x}  = 0 & (x,t) \in [0,L] \times [0, \tau] \\
             y(x,0) = y_{0} (x) & x\in[0,L]\\
			 v(t) = M - \int_0^L x y(x,t)dx - \dep & t \in [0,\tau]\\
			 v(t)y(0,t)\mathbb{I}_{v(t) > 0} = 0 \\
			 v(t)y(L,t)\mathbb{I}_{v(t) < 0} = 0 \\
		\end{cases}
\end{equation}



%%%%%%%%%%%%%%%%%%%%%%%%%%%%%%%%%%%%%%%%%%%%%%%%%%%%%%%%%%%%%%%%%%%%%%%%%%%%%%%%%%%%%%%%%%%%%%%%%%%%%%%%%%%%%%%%%%%%%%%%%%%%%%%%%
%%%%%%%%%%%%%%%%%%%%%%%%%%%%%%%%%%%%%%%%%%%%%%%%%%%%%%%%%%%%%%%%%%%%%%%%%%%%%%%%%%%%%%%%%%%%%%%%%%%%%%%%%%%%%%%%%%%%%%%%%%%%%%%%%
%%%%%%%%%%%%%%%%%%%%%%%%%%%%%%%%%%%%%%%%%%%%%%%%%%%%%%%%%%%%%%%%%%%%%%%%%%%%%%%%%%%%%%%%%%%%%%%%%%%%%%%%%%%%%%%%%%%%%%%%%%%%%%%%%
\section{Etude du problème direct}
%%%%%%%%%%%%%%%%%%%%%%%%%%%%%%%%%%%%%%%%%%%%%%%%%%%%%%%%%%%%%%%%%%%%%%%%%%%%%%%%%%%%%%%%%%%%%%%%%%%%%%%%%%%%%%%%%%%%%%%%%%%%%%%%%
%%%%%%%%%%%%%%%%%%%%%%%%%%%%%%%%%%%%%%%%%%%%%%%%%%%%%%%%%%%%%%%%%%%%%%%%%%%%%%%%%%%%%%%%%%%%%%%%%%%%%%%%%%%%%%%%%%%%%%%%%%%%%%%%%

On se propose d'étudier le système linéaire issu du modèle de Lifshitz-Slyosov où la vitesse de réaction totale, 
somme de la vitesse de polymérisation et dépolimérisation, ne dépend que du temps. 
Les coefficients de polymérisation $\pol$ et de dépolymérisation $\dep$ associés aux réactions sont constants 
et ne dépendent pas de la taille des polymères notée $x$. 

Soit $y$ la distribution en taille des polymères, 
par conservation de la masse,
la vitesse se déduit de la mesure du moment d'ordre 1, noté $\mu_1$, des polymères.

Ainsi le modèle de Lifshitz-Slyosov s'écrit pour $y(x,t)$ la concentration de polymères de taille $x$ en temps $t$ :

\begin{equation}
\label{eq:general}
\begin{cases}
 \displaystyle \frac{\partial y}{\partial t}+ v(t) \frac{\partial y} {\partial x}  = 0  \\
 y(x,0) = y_{0} (x) 
\end{cases}
\end{equation}

où la vitesse $v(t)$ est calculée par conservation de la masse totale $\mass$, des coefficients de polymérisation $\pol$ et $\dep$:

\[
v(t) = \pol(\mass - \int_0 ^\infty x y(x,t) dx)-\dep
\]

et 

\[
\mu_1 = \int_0 ^\infty x y(x,t) dx
\]

 On note que le problème~\eqref{eq:general} est incomplet et qu'il peut être nécessaire d'imposer des conditions aux limites quand cette vitesse est positive ou négative, en fonction du domaine $\Omega$ sur lequel on souhaite résoudre cette équation.
Sans perdre de généralité, 
on pose que le coefficient de polymérisation est égal à 1, 
soit $\pol =1$.

Soit $L>0$ et $\tau>0$, l'objectif est de reconstruire $\hat{y_0}$ 
la condition initiale de l'équation~\eqref{eq:general} 
à partir de mesures de moments $\Psi_n$ pour $n \geq 0$:

 \begin{equation}
	\begin{array}{ccccc}
	\Psi_n & : & L^2([0,L]) & \to & L^2([0,\tau]) \\
	 & & y_0 & \mapsto & t \to \int_0^L x^n y_0(x-\theta(t)) dx\\
	\end{array}
\end{equation}

Nous souhaitons étudier sous quelles conditions ce problème est dit observable.


%%%%%%%%%%%%%%%%%%%%%%%%%%%%%%%%%%%%%%%%%%%%%%%%%%%%%%%%%%%%%%%%%%%%%%%%%%%%%%%%%%%%%%%%%%%%%%%%%%%%%%%%%%%%%%%%%%%%%%%%%%%%%%%%%
%%%%%%%%%%%%%%%%%%%%%%%%%%%%%%%%%%%%%%%%%%%%%%%%%%%%%%%%%%%%%%%%%%%%%%%%%%%%%%%%%%%%%%%%%%%%%%%%%%%%%%%%%%%%%%%%%%%%%%%%%%%%%%%%%
%%%%%%%%%%%%%%%%%%%%%%%%%%%%%%%%%%%%%%%%%%%%%%%%%%%%%%%%%%%%%%%%%%%%%%%%%%%%%%%%%%%%%%%%%%%%%%%%%%%%%%%%%%%%%%%%%%%%%%%%%%%%%%%%%
\section{Etude du problème inverse, premier cas : $c(t)$ connu, on mesure $\mu_0$}
%%%%%%%%%%%%%%%%%%%%%%%%%%%%%%%%%%%%%%%%%%%%%%%%%%%%%%%%%%%%%%%%%%%%%%%%%%%%%%%%%%%%%%%%%%%%%%%%%%%%%%%%%%%%%%%%%%%%%%%%%%%%%%%%%
%%%%%%%%%%%%%%%%%%%%%%%%%%%%%%%%%%%%%%%%%%%%%%%%%%%%%%%%%%%%%%%%%%%%%%%%%%%%%%%%%%%%%%%%%%%%%%%%%%%%%%%%%%%%%%%%%%%%%%%%%%%%%%%%%

\subsection{Question 4}

\textbf{a}

Soit $\Y = L^2(0,L)$ et $\Z = L^2(\mathbb{R}^+)$ deux espaces de Hilbert. 
On définit l'application $\Psi$ :

 \begin{equation}
	 \Psi \; : \; \left\vert
	\begin{array}{ccc}
	\Y & \to & \Z \\
	u^{in} & \mapsto & t \to \int_0^L u(x,t) dx\\
	\end{array} \right.
\end{equation}

Alors l'application $\Psi$ appartient à $L(\Y,\Z)$.

\begin{definition}
Le problème inverde que nous allons étudier se définit ainsi : 
étant donné $z \in \Z$, on cherche $u_{\epsilon}^{in} \in \Y$ 
tel que $\Psi u^{in} = z$.
\end{definition}

\textbf{b}
Soit $u^{in} \in \Y$, et $t \geq 0$ :

\[ \begin{split}
\Psi u^{in}(t) &= \mu_0 (t) \\
               &= \int_0^L u(x,t) dx \\
			   &= \int_0^L u^in( x - a\int_0^tc(s)ds + bt) dx \\
\end{split}\]

\textbf{b}

L'application $\Psi$ est linéaire par linéarité de l'opérateur intégrale.
De plus $\Psi $ est positive dès que $u^{in}$ l'est, par intégration d'une fonction positive.

Montrons que $\Psi$ est continu en montrant que :

\[ \forall (u_1^{in}, u_2^{in}) \in \Y, \; \| \Psi u_1^{in} - \Psi u_2^{in} \|_{\Z} \leq  C \| u_1^{in} - u_2^{in} \|_{\Y}\]

Soit la fonction $\theta$ telle que :

\[\theta \; : \; \left\vert
	\begin{array}{ccc}
	\mathbb{R}^+ & \to & \mathbb{R}^+\\
	t & \mapsto & \int_0^t (ac(s)-b)ds \\
	\end{array} \right.
\]

Puisque pour tout $t \geq 0 $ on a $c(t)< \frac{b}{a}$ alors $ac(t)-b<0$ et $\theta$ est une fonction strictement décroissante.

De plus, $\theta (0) =0$ donc $\theta$ est négatif pour tout $t \geq 0$,
et il existe $\tau$ tel que $\theta(\tau) = - L$.

On pose $u^{in} = u_1^{in} - u_2^{in}$, et $u$ l'unique solution de ~\eqref{eq:general} pour la condition initiale $y^{in}$.

On pose le changement de variable $x'=x - a\int_0^tc(s)ds + bt) = x - \theta (t)$ il vient pour tout $t \geq 0$ :
\[ \begin{split}
\|\Psi u^{in}\|_{\Z}^2 &= \int_0^\infty |u(x,t)|^2 dx \\
			   &= \int_0^\infty \int_0^L |u^in( x - \theta(t))|^2 dx \\
			   &= \int_0^\infty \int_{-\theta(t)}^{L} |u^{in}(x')|^2 dx' \\
			   & \leq int_0^\tau \int_{-\theta(t)}^L |u^{in}(x')|^2 dx'\\
			   & \leq \tau \| u^{in} \|^2
\end{split}\]
avec $\tau = \theta ^{-1}(-L) =$ constante.

On vient donc par cette inégalité de montrer la continuité de $\Psi$. De plus la norme de $\Psi$ est majorée par $\tau = \theta ^{-1}(-L)$.

\textbf{c}


	
\end{document}

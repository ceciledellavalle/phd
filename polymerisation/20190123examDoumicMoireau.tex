%% PRE EDITION
\documentclass[a4paper]{article}
\usepackage[utf8]{inputenc}
\usepackage{ucs} 
\usepackage[T1]{fontenc}
\usepackage[french]{babel}
\usepackage{soul}
\usepackage[pdftex]{graphicx}

\usepackage{amsfonts}
\usepackage{amsthm}
\usepackage{amsmath}
\usepackage{amssymb}
\usepackage{mathrsfs}
\usepackage{booktabs}
\usepackage{siunitx}
\usepackage{thmtools}

%% CHEMICAL EQUATION
\usepackage{chemist}
\usepackage{etex}
\usepackage{m-pictex,m-ch-en}

%% LAYOUT
\declaretheoremstyle[
    bodyfont=\normalfont\color{red},
    headfont=\color{red}
]{styleattention}

\declaretheoremstyle[
    spacebelow=1em
]{styleremarque}

\declaretheoremstyle[
    spaceabove=-6pt, 
    spacebelow=6pt, 
    headfont=\normalfont\bfseries, 
    bodyfont = \normalfont,
    postheadspace=1em, 
    qed=$\Box$, 
    headpunct={$\rhd$}
]{mystyle} 

\declaretheorem[thmbox=M,numberwithin=section,title=Définition]{definition}
\declaretheorem[thmbox=M,sibling=definition]{proposition}
\declaretheorem[thmbox=M,sibling=definition]{corollaire}
\declaretheorem[thmbox=M,sibling=definition,title=Théorème]{theoreme}
\declaretheorem[thmbox=M,sibling=definition]{lemme}
\declaretheorem[thmbox=M,sibling=definition,title=Propriété]{propriete}
\declaretheorem[thmbox=M,sibling=definition,title=Propriétés]{proprietes}
\declaretheorem[style=styleremarque,sibling=definition,title=Remarque]{remarque}
\declaretheorem[style=styleattention,title=À revoir]{Arevoir}
\declaretheorem[name={}, style=mystyle, unnumbered]{preuve}


\renewcommand\qedsymbol{$\blacksquare$}

%% NEW COMMAND
\newcommand{\mass}{\mathrm{M}}
\newcommand{\pol}{a}
\newcommand{\dep}{b}
\newcommand{\Y}{\mathscr{Y}}
\newcommand{\Z}{\mathscr{Z}}
\newcommand{\yea}{y_{\epsilon, \alpha}}
\newcommand{\zea}{z_{\epsilon, \alpha}}

\usepackage{geometry}
\geometry{hmargin=3cm,vmargin=2.5cm}

\usepackage{tabularx}
\usepackage{float}

\title{Examen : Probèmes inverses et dynamique des population}
\author{Cécile Della Valle}

%% DEBUT DE REDACTION
\begin{document}

\maketitle

%%%%%%%%%%%%%%%%%%%%%%%%%%%%%%%%%%%%%%%%%%%%%%%%%%%%%%%%%%%%%%%%%%%%%%%%%%%%%%%%%%%%%%%%%%%%%%%%%%%%%%%%%%%%%%%%%%%%%%%%%%%%%%%%%
%%%%%%%%%%%%%%%%%%%%%%%%%%%%%%%%%%%%%%%%%%%%%%%%%%%%%%%%%%%%%%%%%%%%%%%%%%%%%%%%%%%%%%%%%%%%%%%%%%%%%%%%%%%%%%%%%%%%%%%%%%%%%%%%%
%%%%%%%%%%%%%%%%%%%%%%%%%%%%%%%%%%%%%%%%%%%%%%%%%%%%%%%%%%%%%%%%%%%%%%%%%%%%%%%%%%%%%%%%%%%%%%%%%%%%%%%%%%%%%%%%%%%%%%%%%%%%%%%%%
\section{Etude du problème direct}
%%%%%%%%%%%%%%%%%%%%%%%%%%%%%%%%%%%%%%%%%%%%%%%%%%%%%%%%%%%%%%%%%%%%%%%%%%%%%%%%%%%%%%%%%%%%%%%%%%%%%%%%%%%%%%%%%%%%%%%%%%%%%%%%%
%%%%%%%%%%%%%%%%%%%%%%%%%%%%%%%%%%%%%%%%%%%%%%%%%%%%%%%%%%%%%%%%%%%%%%%%%%%%%%%%%%%%%%%%%%%%%%%%%%%%%%%%%%%%%%%%%%%%%%%%%%%%%%%%%

%%%%%%%%%%%%%%%%%%%%%%%%%%%%%%%%%%%%%%%%%%%%%%%%%%%%%%%%%%%%%%%%%%%%%%%%%%%%%%%%%%%%%%%%%%%%%%%%%%%%%%%%%%%%%%%%%%%%%%%%%%%%%%%%%
\subsection*{Question 1. Positivité, borne supérieure}
%%%%%%%%%%%%%%%%%%%%%%%%%%%%%%%%%%%%%%%%%%%%%%%%%%%%%%%%%%%%%%%%%%%%%%%%%%%%%%%%%%%%%%%%%%%%%%%%%%%%%%%%%%%%%%%%%%%%%%%%%%%%%%%%%


 \textbf{a.} Les constantes $a$ et $b$ sont les constantes de polymérisation et de dépolymérisation. 
 Ce modèle est la limite du modèle de Becker-Döring, 
 correspondant aux réactions chimiques entre des monomères $C_1$ 
 et des polymères de taille $i>2$ noté $P_i$ :
 
\begin{equation}
    \begin{cases}
	P_{i} \; + \; C_1 \overset{a}{\longrightarrow} P_{i+1}\\
	P_{i} \overset{b}{\longrightarrow}  P_{i-1} + \;C_0
    \end{cases}
\end{equation}

Quand le nombre de polymères tend vers $+\infty$, 
on peut poser $x=\epsilon i$, 
qui permet d'adimensionner les équations détaillées. 
Enfin, en faisant tendre $\epsilon$ vers $0$,
on retrouve le système (1).

Ainsi, la quantité $c(t)$ représente 
la concentration de monomères en fonction du temps, et $u(x,t)$ la concentration de polymère de taille $x$ en fonction du temps.

Enfin, les quantités $\mu_0$ et $\mu_1$ représentent respectivement les moments d'ordre $0$ et d'ordre $1$ de la concentration des polymères au cours du temps.

\textbf{b.}
Soit $u$ solution classique du système (1), 
on pose le problème de Cauchy suivant :
\begin{equation}
\begin{cases}
	\label{eq:caract}
	\displaystyle \frac{\mathrm{d}}{\mathrm{d}s} X(t,x_0)
	= (\mass - \dep) - c(t) 
	= v(t)\\
	X(0,t) = x_0
	\end{cases}
	\end{equation}
	
	Ce problème admet une unique solution, de plus elle est de classe $C^1$.
	On dérive $u$ le long de la courbe caractéristique définie par $X$.
	On pose $U(X(s))=u(s, X(s,x_0))$ où l'on suppose que $u$ est une solution classique du système (1).
	Lorsque l'on dérive $U$ le long d'une courbe caractéristique on obtient :
	\[ 
	\begin{split}
		\frac{\mathrm{d}}{\mathrm{d}s} U(X(s)) & = \frac{\mathrm{d} X }{\mathrm{d}s} \frac{\partial}{\partial x}u(s,X(s,x_0)) + \frac{\partial}{\partial t}u(s,X(s,x_0)) \\
		                                 & = v(s) \frac{\partial}{\partial x} u(s,X(s,x_0)) + \frac{\partial}{\partial t} u(s,X(s,x_0))\\
										 & =0 
	\end{split}
		\]
		
	La solution classique $u$ est donc constante le long d'une courbe caractéristique et $u(s,X(s,x_0)) = u(0,x_0)=u^{in}(x_0)$.
	
	De plus on peut obtenir une expression analytique explicite de cetet courbe :
	
	\[ X(t,x_0) = x_0 + \int_0^t v(s)ds \]
	
	D'après cette expression pour tout $t>0$ et $x = x_0 + X(0,t)$ 
	alors $x \in \mathbb{R} = supp(u^{in})$. 
	Le prolongement de $u^{in}$ sur $\mathbb{R}$ nous assure que la quantité
	$u^{in}(x)$ est bien définie pour tout $t$ et $x = x_0 + X(0,t)$.
	
	Ainsi pour tout $t>0$ et toute taille 
	$x$, $u(t,x) = u^{in}(x-\int_0^t v(s)ds)\geq 0$.

\textbf{c.}
Intégrer (1) contre le poids $x$ donne le système d'équation suivant :

\[	
\begin{split}
\displaystyle \frac{\mathrm{d}}{\mathrm{d} t} \int_0^{+\infty} x u(t,x)dx
	                        &= - (ac(t)-b)\int_0^{+\infty}
							x\frac{\partial}{\partial x}u(t,x)dx\\
							&= - (ac(t)-b) [
							[xu(t,x)]_0^{+\infty} 
							- \int_0^{+\infty} u(t,x)dx]\\
							& = (ac(t)-b) \int_0^{+\infty} u(t,x)dx
\end{split}
\]

On en déduit donc :
\[ \displaystyle \frac{\mathrm{d}}{\mathrm{d} t} \int_0^{+\infty} x u(t,x)dx
+ \frac{\mathrm{d}}{\mathrm{d} t} c(t) = 0 \]

Donc la quantité $\mu_1(t) +c(t) = \rho_0$ est constante au cours du temps.
Cette relation correspond à la conservation de la masse.

\textbf{d.} 
	Puisque :
	\[\displaystyle \frac{\mathrm{d}}{\mathrm{d} t} c(t) = -(ac(t)-b)\mu_0(t)\]
	Sachant que pour tout $t>0$ et pour tout $x$ on a $u(t,x)\geq 0$,
	on en déduit que de même $m_0(t) \geq 0$ et $\mu_1(t)\geq 0$
	Or, 
	\[ ac(t)-b = a(\rho_0 - \mu_1(t))-b = a(\rho_0)-b - a\mu_1(t) <0 \]
	Donc la fonction $c$ est strictement croissante dès que $\mu_0(t)>0$.Or, à $t=0$,   $c(0)=0$,
	donc si $\mu_0(0)>0$, pour tout $t>0$, $c(t)> 0$.

\textbf{e.}
On a d'une part pour tout $t\geq 0$, $c(t)+\mu_1(t) = \rho_0$, et d'autre part $c(t)\geq 0$ et $mu_1(t)\geq 0$.

Donc on obtient les deux inégalités :
\[
\begin{cases}
	0 \leq c(t) \leq \rho_0 \\
	0 \leq \mu_1(t) \leq rho_0
\end{cases}
\]


%%%%%%%%%%%%%%%%%%%%%%%%%%%%%%%%%%%%%%%%%%%%%%%%%%%%%%%%%%%%%%%%%%%%%%%%%%%%%%%%%%%%%%%%%%%%%%%%%%%%%%%%%%%%%%%%%%%%%%%%%%%%%%%%%
\subsection*{Question 2. Comportement asymptotique de la solution}
%%%%%%%%%%%%%%%%%%%%%%%%%%%%%%%%%%%%%%%%%%%%%%%%%%%%%%%%%%%%%%%%%%%%%%%%%%%%%%%%%%%%%%%%%%%%%%%%%%%%%%%%%%%%%%%%%%%%%%%%%%%%%%%%%

\textbf{a.}
A $t=0$, $c(0) = 0$ donc $v(0) = ac(0) -b = -b <0$.
Or, on a démontré que :
\[\displaystyle \frac{\mathrm{d}}{\mathrm{d} t} c(t) = -(ac(t)-b)\mu_0(t)\]

Démontrons par l'absurde que la vitesse $v$ ne change pas de signe.

La vitesse $v$ est continue (et même absoluement continue)
puisque c'est une fonction intégrale d'une fonction de $L^1$.

Par le théorème des valeurs intermédiaires, si $v$ change de signe,
alors il exite $t^*>0$ tel que $v(t^*)= 0$, 
	dans ce cas $\mu_1(t^*)= \rho_0 -\dep $, 
	$\frac{\mathrm{d} v }{\mathrm{d}t} (t^*) = 0$.
	
	Si $\mu_0(t^*)>0$ alors $\forall t<t^*$, $0<\mu_0(t^*)<\mu_0(t)$ et on note $ \mu_0(t^*)= \alpha$, et il vient :
	
	\[ \frac{\mathrm{d} |v| }{\mathrm{d}t} = \mu_0(t)|v(t)| \geq \alpha |v(t)| \]
	
	Donc d'après le lemme de Grönwall :
	\[ |v(t)| \geq |v(0)| e^{-\alpha t} \]
	
	Ceci contredit le fait que $v$ s'annule en $t=t^*$.
	Donc nécessairement $\mu_0(t^*) = 0$, 
	or, $x \to xy (x,t)$ et $x \to y(x,t)$ sont des fonctions positives de même support,
	donc $\mu_1(t^*) =0 \implies \mu_0(t^*)=0$ .
	
	En effet, le système est entièrement dépolymérisé et le système vérifie :
	
	$\forall t>t^*$:  $y(x,t)=y(x,t^*)= 0$.
	
	Donc $v(t^*) = \mass - \dep - \mu_1(t^*) = \mass- \dep > 0$.
	
	C'est une contradiction.	

En utilisant maintenant l'hypothèse (2), on peut écrire 
en posant $\delta = -(rho_0 a -b) $ et dans ce cas :
\[ ac(t)-b \leq \delta <0 \] 

Donc les courbes caractéristiques ont une pente $v$ strictement négative.

\textbf{b.}
Pour tout $t>0$, $x \in[0,+\infty)$, on a :
\[ u(x,t) = u^{in}(x- \int_0^t v(s)ds) \]

Donc pour $T = \displaystyle \frac{L}{\delta}$, et $t>T$, on a $x- \int_0^t v(s)ds > x+\delta t >L $. Or, la fonction $u^{in}$ est de support sur $[0,L]$ 
donc pour $t>T$, $u(x,t)=0$.
On a naturellement comme borne supérieure $\displaystyle \frac{L}{\delta}$.

\textbf{c.}
Par le même raisonnement, à $t>0$ fixé, 
on suppose que $supp(u^{in})= [l_1,l_2]$,
alors  d'après l'expression de $u$ issue de la méthode des caractéristiques :
\[supp (u(t,x)) = [l_1 + \int_0^t v(s) , l_2 + \int_0^t v(s) ds] \cap [0,L]\] 

\textbf{d.}
Pour $t>T$, la fonction $u(x, \cdot)=0$, 
et on a $c(t) = \rho_0 - \mu_1(t) = \rho_0$.
La fonction $c$ converge en un temps fini vers une constante.




%%%%%%%%%%%%%%%%%%%%%%%%%%%%%%%%%%%%%%%%%%%%%%%%%%%%%%%%%%%%%%%%%%%%%%%%%%%%%%%%%%%%%%%%%%%%%%%%%%%%%%%%%%%%%%%%%%%%%%%%%%%%%%%%%
\subsection*{Question 3. Equations des moments}
%%%%%%%%%%%%%%%%%%%%%%%%%%%%%%%%%%%%%%%%%%%%%%%%%%%%%%%%%%%%%%%%%%%%%%%%%%%%%%%%%%%%%%%%%%%%%%%%%%%%%%%%%%%%%%%%%%%%%%%%%%%%%%%%%

\textbf{a.}
Pour $k>1$ :
\[	
\begin{split}
	\displaystyle \frac{\mathrm{d}}{\mathrm{d} t} \int_0^{+\infty} x^k u(t,x)dx
	                        &= - (ac(t)-b)\int_0^{+\infty}
							x^k \frac{\partial}{\partial x}u(t,x)dx\\
							&= - (ac(t)-b) [
							[x^k u(t,x)]_0^{+\infty} 
							- \int_0^{+\infty} u(t,x)dx ]\\
							& = (ac(t)-b) \int_0^{+\infty} kx^{k-1} u(t,x)dx
\end{split}
\]

Donc $ \displaystyle \frac{\mathrm{d}}{\mathrm t}\mu_k = k (ac(t)-b) \mu_{k-1}$


\textbf{b.}
Pour $k =0$ :
\[
\begin{split}
	\displaystyle \frac{\mathrm{d}}{\mathrm{d} t} \int_0^{+\infty} u(t,x)dx
	                        &= - (ac(t)-b)\int_0^{+\infty}\frac{\partial}{\partial x}u(t,x)dx\\
							&= - (ac(t)-b) [
							[u(t,x)]_0^{+\infty} \\
							&= 	(ac(t)-b)u(t,0)
\end{split}
\]

Donc $\displaystyle \frac{\mathrm{d}}{\mathrm{d} t} \mu_0 = (ac(t)-b)u(t,0)$.





%%%%%%%%%%%%%%%%%%%%%%%%%%%%%%%%%%%%%%%%%%%%%%%%%%%%%%%%%%%%%%%%%%%%%%%%%%%%%%%%%%%%%%%%%%%%%%%%%%%%%%%%%%%%%%%%%%%%%%%%%%%%%%%%%
%%%%%%%%%%%%%%%%%%%%%%%%%%%%%%%%%%%%%%%%%%%%%%%%%%%%%%%%%%%%%%%%%%%%%%%%%%%%%%%%%%%%%%%%%%%%%%%%%%%%%%%%%%%%%%%%%%%%%%%%%%%%%%%%%
%%%%%%%%%%%%%%%%%%%%%%%%%%%%%%%%%%%%%%%%%%%%%%%%%%%%%%%%%%%%%%%%%%%%%%%%%%%%%%%%%%%%%%%%%%%%%%%%%%%%%%%%%%%%%%%%%%%%%%%%%%%%%%%%%
\section{Etude du problème inverse, premier cas : $c(t)$ connu, on mesure $\mu_0$}
%%%%%%%%%%%%%%%%%%%%%%%%%%%%%%%%%%%%%%%%%%%%%%%%%%%%%%%%%%%%%%%%%%%%%%%%%%%%%%%%%%%%%%%%%%%%%%%%%%%%%%%%%%%%%%%%%%%%%%%%%%%%%%%%%
%%%%%%%%%%%%%%%%%%%%%%%%%%%%%%%%%%%%%%%%%%%%%%%%%%%%%%%%%%%%%%%%%%%%%%%%%%%%%%%%%%%%%%%%%%%%%%%%%%%%%%%%%%%%%%%%%%%%%%%%%%%%%%%%%




%%%%%%%%%%%%%%%%%%%%%%%%%%%%%%%%%%%%%%%%%%%%%%%%%%%%%%%%%%%%%%%%%%%%%%%%%%%%%%%%%%%%%%%%%%%%%%%%%%%%%%%%%%%%%%%%%%%%%%%%%%%%%%%%%
\subsection*{Question 4. Formulation du problème inverse, cadre statique}
%%%%%%%%%%%%%%%%%%%%%%%%%%%%%%%%%%%%%%%%%%%%%%%%%%%%%%%%%%%%%%%%%%%%%%%%%%%%%%%%%%%%%%%%%%%%%%%%%%%%%%%%%%%%%%%%%%%%%%%%%%%%%%%%%


\textbf{a}.

Soit $\Y = L^2(0,L)$ et $\Z = L^2(\mathbb{R}^+)$ deux espaces de Hilbert. 
On définit l'application $\Psi$ :

 \begin{equation}
	 \Psi \; : \; \left\vert
	\begin{array}{ccc}
	\Y & \to & \Z \\
	u^{in} & \mapsto & t \to \int_0^L u(x,t) dx\\
	\end{array} \right.
\end{equation}

Alors l'application $\Psi$ appartient à $L(\Y,\Z)$.

\begin{definition}
Le problème inverde que nous allons étudier se définit ainsi : 
étant donné $z \in \Z$, on cherche $u_{\epsilon}^{in} \in \Y$ 
tel que $\Psi u^{in} = z$.
\end{definition}

\textbf{b.}
Soit $u^{in} \in \Y$, et $t \geq 0$ :

\[ \begin{split}
\Psi u^{in}(t) &= \mu_0 (t) \\
               &= \int_0^L u(x,t) dx \\
			   &= \int_0^L u^{in}( x - a\int_0^tc(s)ds + bt) dx \\
\end{split}\]

\textbf{c.}

L'application $\Psi$ est linéaire par linéarité de l'opérateur intégrale.
De plus $\Psi $ est positive dès que $u^{in}$ l'est (d'après la question 1.b), par intégration d'une fonction positive.

Montrons que $\Psi$ est continu en montrant que :

\[ \forall (u_1^{in}, u_2^{in}) \in \Y, \; \| \Psi u_1^{in} - \Psi u_2^{in} \|_{\Z} \leq  C \| u_1^{in} - u_2^{in} \|_{\Y}\]

Soit la fonction $\theta$ telle que :

\[\theta \; : \; \left\vert
	\begin{array}{ccc}
	\mathbb{R}^+ & \to & \mathbb{R}^+\\
	t & \mapsto & \int_0^t (ac(s)-b)ds \\
	\end{array} \right.
\]

Puisque pour tout $t \geq 0 $ on a $v(t)<0$,
alors cela implique aue $c(t)< \frac{b}{a}$ alors $ac(t)-b<0$ et $\theta$ est une fonction strictement décroissante, de plus $\theta$ est bijective puisque sa dérivée ne s'annule pas sur $\mathbb{R}^+$, comme montré à la question 2.a..

De plus, $\theta (0) =0$ donc $\theta$ est négatif pour tout $t \geq 0$,
et il existe $T$ tel que $\theta(T) = - L$.

On pose $u^{in} = u_1^{in} - u_2^{in}$, et $u$ l'unique solution du système (1) pour la condition initiale $u^{in}$.

On pose le changement de variable $x'=x - a\int_0^tc(s)ds + bt) = x - \theta (t)$ il vient pour tout $t \geq 0$ :
\[ \begin{split}
\|\Psi u^{in}\|_{\Z}^2 &= \int_0^\infty |u(x,t)|^2 dx \\
			   &= \int_0^\infty \int_0^L |u^in( x - \theta(t))|^2 dx \\
			   &= \int_0^\infty \int_{-\theta(t)}^{L} |u^{in}(x')|^2 dx' \\
			   & \leq int_0^T \int_{-\theta(t)}^L |u^{in}(x')|^2 dx'\\
			   & \leq T \| u^{in} \|^2
\end{split}\]
avec $T = \theta^{-1}(-L) =$ constante.

On vient donc par cette inégalité de montrer la continuité de $\Psi$. De plus la norme de $\Psi$ est majorée par $T = \theta ^{-1}(-L)$.

\textbf{d.}
On introduit la notation $\theta(t)= \int_0^t v(s)ds$
Soit $u_1^{in}$, $u_2^in$ de $\mathscr{Y}$,
calculons la quantité :
\[
\begin{split}
	\langle \Psi u_1^{in}, \Psi u_2^in \rangle_{\mathscr{Z}}  
	                                  &= \int_{t=0}^{+\infty} 
									  (\int_{x_1=0}^L u_1(t,x_1)dx_1)
									  (\int_{x_2=0}^L u_2(t,x_2) dx_2)
									  dt\\
	                                  &= \int_{t=0}^{T} 
									  \int_{x_1=-\theta(t)}^L \int_{x_2=0}^L
									  u_1^{in}(x_1)
									   u_2(t,x_2) 
									  dx_2dx_1dt\\
	                                  &= \int_{x_1=0}^L
									  (u_1^{in}(x_1)
									  \int_{t=0}^{\theta^{-1}(-x_1)} 
									   \int_{x_2=0}^L
									   u_2(t,x_2) 
									  dx_2 dt )dx_1\\
	                                  &= \langle u_1^{in},
									  \Psi^* \Psi u_2^in \rangle_{\mathscr{Y}}
\end{split}
\]

On définit donc l'adjoint :

\begin{equation}
	 \Psi^* \; : \; \left\vert
	\begin{array}{ccc}
	\Z & \to & \Y \\
	\mu_0 & \mapsto & x \to \int_0^{\theta(-x)} \mu_0(t)dt \\
	\end{array} \right.
\end{equation}

%%%%%%%%%%%%%%%%%%%%%%%%%%%%%%%%%%%%%%%%%%%%%%%%%%%%%%%%%%%%%%%%%%%%%%%%%%%%%%%%%%%%%%%%%%%%%%%%%%%%%%%%%%%%%%%%%%%%%%%%%%%%%%%%%
\subsection*{Question 5. Résolution "directe" du problème inverse}
%%%%%%%%%%%%%%%%%%%%%%%%%%%%%%%%%%%%%%%%%%%%%%%%%%%%%%%%%%%%%%%%%%%%%%%%%%%%%%%%%%%%%%%%%%%%%%%%%%%%%%%%%%%%%%%%%%%%%%%%%%%%%%%%%

\textbf{a.}
D'après la question 3.b., on a obtenu la relation, pour $\mu_0 \in \mathscr{Z}$ 
et $u^{in} \in \mathscr{Y}$ :

\[ \displaystyle \frac{\mathrm{d}}{\mathrm{d} t} \mu_0 
= (ac(t)-b)u^{in}(-\int_0^tv(s)ds)\]

Ce problème est mal posé d'ordre 1, en effet $\mu_0$ est une fonction de $L^2$ et non de $H^1$. 

\textbf{b.}
On remarque que :
\[ Y_0(t) = bt - a\int_0^t c(s)ds = -\int_0^t v(s)ds = -\theta (t)\]

Or reprend le même raisonnement qu'à la question 2.c., puisque pour tout $t \geq 0 $ on a $v(t)<0$,
alors cela implique que $Y_0$ est une fonction strictement croissante de $\mathbb{R}^+$, elle est donc bijective.
On peut donc écrire pour tout $x $ :

\[ u^{in}(x) = \displaystyle \frac{1}{v(Y_0^{-1}(x))} \frac{\mathrm{d}}{\mathrm{d} t} \mu_0 (Y_0^{-1}(x))\]

La fonction $f(x) = \frac{1}{v(Y_0^{-1}(x))} $ est négative elle est donc majorée par $0$ et minorée par $-1/\delta$ d'après la question 2.a..

\textbf{c.}
Pour définir le pseudo-inverse de Moore-Penrose, il faut que l'opérauteur $\Psi$ soit borné, ce point est bien vérifié pour toute solution $u$ du système (1) d'après la question 4.c..
L'image de $\Psi$ sont les fonctions $H^1[0,+\infty)$ à support sur $[0,Y_0^{-1}(L)]$.

\[ Im(\Psi) = \left\{\mu \in H^1[0,+\infty) \;|\; supp(\mu) \subset[0,Y_0^{-1}(L)] \right\} \]

On peut donc définir le pseudo inverse par cette formule :

\begin{equation}
	 \Psi^\dagger \; : \; \left\vert
	\begin{array}{ccc}
	\Z & \to & \Y \\
	\mu_0 & \mapsto & \displaystyle \frac{1}{v(Y_0^{-1}(x))} \frac{\mathrm{d}}{\mathrm{d} t} \mu_0 (Y_0^{-1}(x)) \\
	\end{array} \right.
\end{equation}

et $D(\Psi^\dagger) = Im(\Psi) + Im(\Psi)^\perp$

%%%%%%%%%%%%%%%%%%%%%%%%%%%%%%%%%%%%%%%%%%%%%%%%%%%%%%%%%%%%%%%%%%%%%%%%%%%%%%%%%%%%%%%%%%%%%%%%%%%%%%%%%%%%%%%%%%%%%%%%%%%%%%%%%
\subsection*{Question 6. Résolution du problème inverse par la méthode de Tikhonov généralisée}
%%%%%%%%%%%%%%%%%%%%%%%%%%%%%%%%%%%%%%%%%%%%%%%%%%%%%%%%%%%%%%%%%%%%%%%%%%%%%%%%%%%%%%%%%%%%%%%%%%%%%%%%%%%%%%%%%%%%%%%%%%%%%%%%%

\textbf{a.}
Ce problème inverse revient à résoudre la question suivante : peut-on estimer $y=u^{in}$ la condition initiale par la mesure du moment d'ordre 0 d'une solution du système (1) ?
Comme dans le cadre du cours (chapitre 2) sur l'exemple du problème inverse de l'intégrale, ce problème est mal posé d'ordre 1 pour la norme sur $L^2$ sur $\mathscr{Z}$.

\textbf{b.}

On note à titre préliminaire :
\[
\begin{cases}
	\zea(t) = \Psi \yea(t) \\
	\zea'(t) = - Y_0'(t)\yea(Y_0(t))\\
	\zea(T) = \int_{Y_0(T)}^L \yea(x)dx = 0
\end{cases}
\]

Il vient ainsi :
\[
\begin{split}
	\int_0^T (Y_0'(t)\yea(Y_0(t)))(\Psi\yea - \alpha Y_0'(t)\yea(Y_0(t)))dt
	 & = \int_0^T (Y_0'(t)\yea(Y_0(t))) \Psi \yea(t)dt
	 - \int_0^T \alpha (Y_0'(t)\yea(Y_0(t)))^2 dt \\
	 &= - \int_0^T \zea'(t) \zea(t)dt
	 - \alpha \int_0^T \zea'(t)^2 dt\\
	 &= - \displaystyle \frac{1}{2}[\zea(t)^2]_0^T
	 - \alpha \int_0^T \zea'(t)^2 dt\\
	 & = \displaystyle \frac{1}{2}\zea(0)^2 
	 - \alpha \int_0^T \zea'(t)^2 dt
\end{split}
\]

D'autre part :

\[
\int_0^T (Y_0'(t)\yea(Y_0(t))) z_{\epsilon}(t)dt 
= - \int_0^T \zea'(t)z_{\epsilon}(t) dt
\]

Donc :
\[
\displaystyle \frac{1}{2}\zea(0)^2 
	 - \alpha \int_0^T \zea'(t)^2 dt
	 = - \int_0^T \zea'(t)z_{\epsilon}(t) dt
\]

De plus, pour obtenir l'inégalité, posons le changement de variable $s = Y_0(t)$,
et $ds = Y_0'(t)dt$ :

\[
\begin{split}
\int_0^T \zea'(t)^2 dt & = \int_0^T Y_0'(t)\yea(Y_0(t))^2 (Y_0'(t)dt) \\
                       & = \int_0^L Y_0'(Y_0^{-1}(s)) \, \yea(s)^2 ds\\
\end{split}
\]

Donc :
\[
\begin{split}
\| \yea \|_{\Y}^2 &= \int_0^L \yea(x)^2 dx \\
                  &\leq \displaystyle \frac{1}{v_{min}}
				      \int_0^T \zea'(t)^2 dt \\ 
				  &\leq \displaystyle \frac{1}{\alpha v_{min}}
				      (\displaystyle \frac{1}{2}\zea(0)^2 
				      + \int_0^T \zea'(t)z_{\epsilon}(t) dt)\\
					  &\leq \displaystyle \frac{1}{\alpha v_{min}}
					      (\displaystyle \frac{1}{2}\zea(0)^2 
					      + (\int_0^T \zea'(t)^2 dt \in_0^Tz_{\epsilon}(t)^2 dt)^{1/2}\,)\\
				  &\leq \displaystyle \frac{1}{\alpha v_{min}}
				      (\frac{1}{2}  \| \yea \|_{\Y}^2
				      +  v_{max} \| \yea \|_{\Y}
					  \|z_{\epsilon} \|_{\Z})
\end{split}
\]

Soit, pour $0< 1-\displaystyle \frac{1}{2\alpha v_{min}} $  :

\[
\| \yea \|_{\Y}^2 (1-\displaystyle \frac{1}{2\alpha v_{min}})
  \leq \frac{v_{max}}{\alpha v_{min}}  \| \yea \|_{\Y} \|z_{\epsilon} \|_{\Z})
\]				  

Il vient donc :
\[
\| \yea \|_{\Y}  \leq f(\alpha) \|z_{\epsilon} \|_{\Z}
\]

Avec $f(\alpha) = \displaystyle \frac{2v_{max}}{2\alpha v_{min -1}}$
	
\end{document}
